\documentclass[11pt]{amsart}
\usepackage[marginratio=1:1]{geometry}  % See geometry.pdf to learn the layout options. There are lots.
%\geometry{letterpaper} % ... or a4paper or a5paper or ... 
%\geometry{landscape}  % Activate for for rotated page geometry
\usepackage[parfill]{parskip}    % Activate to begin paragraphs with an empty line rather than an indent
%\usepackage{amsfonts}
\usepackage{palatino}
\usepackage{natbib}
\usepackage{hyperref} 
% \usepackage{paralist}

\title[Structural modeling]{Structural modeling in accounting research}



\author{Reviewer}
%\date{}   % Activate to display a given date or no date

\begin{document}

\section{Introduction}
The methods for estimating causal effects described above (i.e., instrumental variables, RDD, and natural experiments) have been characterized as the experimentalist approach. 
We argue above that the experimentalist approach is unlikely to yield more than a tiny fraction of the research in top accounting journals. One alternative approach to empirical research in economics is labeled the structural approach (Wolpin 2013).

The structural approach ``requires that a researcher explicitly specity a model of economic behavior, that is, a theory." (Wolpin, 2013, p.2).

\end{document}
