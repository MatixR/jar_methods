\documentclass[11pt]{amsart}
\usepackage[marginratio=1:1,margin=1in]{geometry}  % See geometry.pdf to learn the layout options. There are lots.
%\geometry{letterpaper} % ... or a4paper or a5paper or ... 
%\geometry{landscape}  % Activate for for rotated page geometry
% \usepackage[parfill]{parskip}    % Activate to begin paragraphs with an empty line rather than an indent
%\usepackage{amsfonts}
\usepackage{palatino}
\usepackage[longnamesfirst]{natbib}
\usepackage{hyperref} 
% \usepackage{paralist}
\usepackage{tikz}
\usepackage{pgf}
\usepackage{attrib}
\usepackage{caption}
\usepackage{subcaption}
\usepackage{pdflscape}
\usepackage{setspace}
\usepackage{ragged2e}
\raggedbottom

\setlength\parindent{1cm}

\newtheorem{theorem}{Theorem}
\newtheorem{lemma}[theorem]{Lemma}
\theoremstyle{definition}
\newtheorem{definition}{Definition}

\tikzset{every node/.style = 
		    	{shape = rectangle, rounded corners, fill = black!30!white,
		   		text width = 3cm, minimum height = 1.5cm, align = center, text = black},
		    every edge/.style = {draw, ->, line width=2pt, black}}

% The following are for Peter's tables		    
\newcommand\T{\rule[0em]{0pt}{1.5em}} % Top strut
\newcommand\B{\rule[-1em]{0pt}{0pt}} % Bottom strut

% To eliminate unnecessary space before bullet points
\usepackage{enumitem}
\setlist{nolistsep}

\title[Causal Inference]{Causal Inference in Accounting Research}

\author{Ian D. Gow \and David F. Larcker \and Peter Reiss}
\thanks{We thank Eugene Soltes, Dan Taylor and Charlie Wang for helpful discussions.}
%\date{}   % Activate to display a given date or no date

\begin{document}
\usetikzlibrary{automata, shapes, calc, positioning}

\bibliographystyle{chicago}
% Quick LaTeX Guide for Dave (originally for Suraj).

% - Percent signs (%) mark comments. To get a percent sign, escape it by putting a backslash in front.
%  & is another special character in LaTeX. Use \& to get &.
% Note that each part of the document is in a separate file (so we can edit in parallel).
% Citations are automatic with the correct key. 
% LaTeX doesn't pay attention to multiple spaces. Also adjacent lines get collapsed into single paragraphs.
% Insert a blank line between lines that are part of two separate paragraphs.
% \section, \subsection, and \subsubsection have the obvious meanings.
% Note that there is a file jar_methods.bib in the list of files to the right that this pulls bibliographic information from.
\maketitle


% Why do we need the abstract on a separate page? Will this need to be anonymized for review? (We are identified on the conference website!)
% It is quite easy to change the format of the document without doing it "manually".
% \clearpage
% {Causal Inference in Accounting Research}

\begin{abstract}
	In this paper, we examine the approaches used by accounting researchers to draw causal inferences from analyses of archival or observational data. The vast majority of empirical papers place causal interpretations on their statistical results.
	Unfortunately, the research design e.g., (``natural experiments") and econometric approaches (e.g., instrumental variables and regression discontinuity designs) do not generally provide a reasonable basis for causal inferences.
	We argue that observational accounting research would benefit from more focus on (i) in-depth descriptive studies to provide detailed insights into the actual institutional mechanisms that generation observational data and (ii) structural modeling using these institutional insights and data to estimate causal theoretical models.  
	Although the structural approach does not completely resolve the philosophical controversies associated with drawing causal inferences from observational data, it provides a much better basis for understanding causal mechanisms and developing appropriate counterfactuals.
\end{abstract}


\clearpage

 
\section{Introduction}

\begin{quotation}
	There is perhaps no more controversial practice in social and biomedical research than drawing inferences from observational data.
	Despite \dots problems, observational data are widely available in many scientific fields and are routinely used to draw inferences about the causal impact of interventions.
	The key issue, therefore, is not whether such studies should be done, but how they may be done well.
\attrib{\citealt{Berk:1999uz}}
\end{quotation}

% Dave: It's actually helpful to put every sentence on a separate line. 
% You need two line breaks to indicate a paragraph.

Most empirical research in accounting relies primarily on observational data, i.e., data produced by observation of processes completely outside the control of the researcher. The goal of this paper is to evaluate the approaches used by accounting researchers to causal inference using observational (or non-experimental) data and, drawing on developments in fields such statistics, epidemiology, and political science, identify opportunities for improvement.

The importance of causal inference in accounting research is clear from the research questions that accounting researchers seek to answer. Using papers published in 2014 in the \textit{Journal of Accounting Research}, \textit{The Accounting Review}, or \textit{Journal of Accounting and Economics} as a sample, we find that most original research papers use observational data and that that about 90\% of such papers seek to draw causal inferences.\footnote{
 ``Original" research excludes papers that are surveys or discussions of other papers.
We also exclude experimental and field-based research papers, although most of these papers also seek to draw causal inferences. 
We recognize that some authors might deny making causal inferences and there is some subjectivity in our characterizations.
For example, a researcher might argue that a paper that claimed that ``theory predicts X is associated Y and, consistent with that theory, we show X is associated with Y'' is merely a descriptive paper that does not make causal inferences.
However, by stating that ``consistent with . . . theory, X is associated with Y," the clear purpose is to argue that the
evidence tilts the scale, however slightly, in the director of believing the theory is a valid description of the real world: in other words, inference.
Since a theory inevitable involves causal relations, such inference is inherently causal inference.}
That accounting researchers focus on causal inference is consistent with the view that ``the most interesting research in social science is about questions of cause and effect" \cite[p. 3]{Angrist:2008vk}.
Most long-standing questions in accounting research are causal: 
Does conservatism affect the terms of loan contracts?
Do higher quality earnings lead to lower information asymmetry? 
Did IFRS cause an increase in liquidity in the jurisdictions that adopted it?
Additionally, inferences that support or refute broader theories are arguably generalizable, whereas atheoretical correlations provide no basis for predictions about what would happen in different circumstances.

It is well understood by accounting researchers that the use of observational data for causal inference is problematic.
In an experimental setting, the validity of causal inference usually relies on random assignment to differing treatment groups (e.g., ``treatment" and ``control"), where randomization can be effected through manipulation by the researcher. %TODO: Cite some classic reference here.
However, with observational data, treatment assignment is outside the control of the researcher and quite likely driven by forces that confound attempts to draw causal inferences using approaches that would appropriate using experimental data.

Our survey of published research in accounting in 2014 suggests that most papers seeking to draw causal inferences use traditional methods such as regression analysis with potential confounding variables included as controls.
We first examine the use of these traditional methods and the appropriateness of their use for causal inference.
To structure our discussion, we use the technology of causal diagrams \citep{Pearl:2009kh}.
We argue that causal diagrams help to clarify thinking about the causal relations assumed by the researcher and, using straightforward mathematical reasoning, provide clarity about the causal inferences that can (and cannot) be drawn from observational data.
With regard to regression approaches, causal diagrams provide guidance regarding such practical issues as which covariates should be included as controls in regressions and, an often overlooked issue, which should not be included.
Nonetheless, the fact that treatment is not randomly assigned leads many researchers to be skeptical of any efforts to use regression analyses for causal inference.

Recently, some social scientists have held out hope that better research designs and statistical methods can increase the credibility of causal inferences.
For example, \citet{Angrist:2010jv} claim that ``empirical microeconomics has experienced a credibility revolution, with a consequent increase in policy relevance and scientific impact.''  
\citet[p. 26]{Angrist:2010jv} argue that such ``improvement has come mostly from better research designs, either by virtue of outright experimentation or through the well-founded and careful implementation of quasi-experimental methods."
These quasi-experimental methods are used to some degree in accounting research (our survey of research published in 2014 finds four studies claiming to study natural experiments and ten studies using instrumental variables).
We then examine and critically evaluate the use of the quasi-experimental methods discussed in \citet{Angrist:2010jv}, \citet{Roberts:2013cz} and others in accounting research.
We find that, upon closer examination, it is clear that these approaches are not used in a way that enhances the credibility of the causal inferences drawn.
We argue that more transparent reasoning, perhaps with the aid of causal diagrams, would have revealed the flaws in these studies.
While regression discontinuity designs rely on weaker assumptions than other methods, the estimates they provide often will not be of the effects of primary interest to researchers.
Finally, we argue that quasi-experimental methods are unlikely to apply to the vast majority of accounting research.
The randomization required by natural experiments and instrumental variables is likely to be very rare and few studies can avail themselves of the sharp breaks in treatment assignment required for regression discontinuity designs.

%One especially promising path is use of  field experiments with randomized treatments to measure causal effects (a good example -- Roberts QJE paper)} Perhaps the most active area are the use of quasi-experiments with observational data. These suggestions include
% This makes it  difficult, if not impossible, to claim that changes in accounting variables such as ``tone management'' \citep{Huang:2014cs}, ``occurrence of a material restatement'' \citep{Chen:2014ji}, or ``adoption of fair value reporting model'' \citep{Liang:2014ea} provide unbiased estimates of causal effects for their selected outcome variables.
 
%TODO: IDG: Discuss how we will use this as a framework for evaluating the 2014 accounting research

% this all goes back to Blalock, Duncan, Goldberger and other famous sociologists, which we should probably highlight.  Interestingly, this is the essence of path model and latent variable models that I have little success getting into the accounting literature.


Having argued that causal inference is the primary focus of accounting research, but quasi-experimental methods have limited applicability in accounting research, the message of the first part of our paper might be perceived as pessimistic and as not offering a realistic path forward for accounting researchers.
To address this concern, in the second part of the paper we seek to identify approaches that provide a viable path forward. 
We identify emerging approaches in accounting research and also draw on other disciplines to offer a vision for how accounting research might successfully address causal questions even when clever instruments or natural experiments do not provide sharp, straightforward answers to the questions of interest to the field.
In particular, we discuss developments in thinking about causal inference in economics, statistics, political science, sociology, and epidemiology.

% Structural model does not solve endogeneity, but makes it explicit and gives it a theoretical and institutional basis.

Ultimately we believe that accounting researchers need to move away from the ``false promises" of causal inference using quasi-experimental methods being applied to somewhat contrived research topics.
A more compelling direction is twofold:  
\begin{itemize}
\item Increased emphasis on the study of causal mechanisms.
	Specifically, we argue that there is a greater role for in-depth descriptive research to understand how putative causes actually have the effect we conjecture they have based on evidence of associations. 
	This approach is likely to also help us to formulate new hypotheses.
\item Increased use of structural models.
	As discussed above, a well-formulated causal diagram can be viewed as a non-parametric structural model. 
	But, guided by theory, a structural model can be formulated in a parametric fashion and actually taken to data.
\end{itemize}

We view these two approaches a complementary. 
Study of mechanisms will provide the raw materials for well-founded models and structural models will often have an account of the underlying causal mechanisms that may not be observed in large-sample archival data.
%TODO: Obviously this paragraph needs work, but we want to highlight where we are going in the paper -- seems like the descriptive discussion maps into your idea that many of our research ideas are stupid and have nothing to do with the real world

\section{Causal inference: An overview}

% I'm not sure about this. I think we want to establish the fact that accounting researchers are trying to do causal inferences ... then evaluate the approaches taken. The "..." is where I think we introduce the tools to evaluate.
%
% One "wrinkle" is the fact that causal inference using "OLS" is still the dominant approach. 90 papers do causal inference: 14 use IV or "natural experiments" (all BS), leaving 74 papers doing OLS, etc. I guess some are doing D-in-D, etc., but what does that achieve, really? It's just "OLS". Perhaps we need to make this explicit. In that case, survey basics/descriptives, then Pearl, then "OLS" then quasi-experimental methods. Causal graphs for OLS should be easy. The OLS section could be where we discuss generic endogeneity issues (as in our original paper outline).

% add the classic Pearl references here

%THIS SEEMS TO BELONG HERE   
% Do you like this idea? I think that many issues in accounting research would disappear if researchers were more explicit and careful about their reasoning.
% The SCM of Pearl is (so Pearl claims) a generalization of the Rubin Causal Model, and structural models in economics.
% Causal graphs can be interpreted as "non-parametric structural models" and structural models therefore are a special case.
% I think by explicating the approach to causal graphs, we can be more positive and seem to be introducing something helpful, as opposed to seeming very negative and not having answers.
% This also provides a natural bridge to the structural model stuff at the back.
% The reason for doing it first is to use it in the first part.

\subsection{Causal inference using observation data in accounting research}

% Add a footnote to the guy doing something similar is AOS -- claims that only 3% of papers are causal

To get a sense for the importance of causal questions in accounting research,
we conducted a survey of all papers published in 2014 in the  \textit{Journal of Accounting Research}, \textit{The Accounting Review}, or the \textit{Journal of Accounting and Economics}).
We counted 139 papers, of which, 125 are original research papers (a further 14 papers survey or discuss other papers).

We assign a category to each original research paper based on the methods used in the paper: ``theoretical''  (7), ``experimental'' (12), ``field" (3), or ``archival"  (103). 
For our discussion below, we collect the field and archival papers into a single category as they all use observational data.
For each non-theoretical paper, we examined the paper to determine the primary research questions asked and whether the primary or secondary research questions in each paper are
``causal" in nature.
To do this, it generally sufficed to examine the title and abstract for evidence of causal inference. 
Often the title reveals a causal question, with words such as  ``effect of \dots" or ``impact of \dots"  
\citep[e.g.][]{Cohen:2014jl,Clorproell:2014cv} making it clear that a causal question was being asked. 
Often language in the abstract reveals a goal of causal inference. 
For example, \citet{deFranco:2014ct} asks ``how the tone of sell-side debt analysts' discussions about debt-equity conflict events \emph{affects} the informativeness of debt analysts’ reports in debt markets.''

Of the 106 original papers using observational data, we coded 91 as seeking to draw causal inferences.
Of the remaining empirical papers, we coded 7 papers as having a goal of ``description'' (including two of the three field papers). 
For example, \citet{Soltes:2013ba} uses data collected from one firm to provide insights into when analysts privately interact with management the nature of these interactions.
We coded 5 papers as having a goal of ``prediction.'' 
For example, \citet{Czerney:2014bv} examine whether the inclusion of ``explanatory language" in unqualified audit reports can be used to predict the detection of financial misstatements in the future.
We coded 3 papers as having a goal of ``measurement.'' 
For example, \citet{Cready:2014ji} examine whether inferences about traders based on trade size are reliable and suggest improvements to the measurement of variables used by accounting researchers. 

\subsection{Causal inference: A framework}
We will use causal diagrams as a framework for discussing issues in causal inference throughout the paper.

Of the 91 papers in accounting research in 2014 seeking to draw causal inference from archival data, as discussed below, only a small minority uses quasi-experimental methods for inferences.
The more common approach we observed is the use of estimation approaches such as ordinary least-squares regression or propensity-score matching for estimating coefficients that the authors use to draw causal inferences.
The typical concern in these papers is with the inclusion of controls.


For example, consider \cite{Hollander:2010jg} that explores the determinants of managers' decisions to disclose or withhold information response to analyst questions in earnings conference calls, and identify four ``explanatory variables" (namely, disclosure agency problems, proprietary information, firm performance, and litigation.\footnote {The role of firm performance as a causal variable of interest is clouded by the fact that the authors state " we control for performance" (p. 544)} \cite{Hollander:2010jg} state that ``prior research indicates the several other company characteristics determine investor demand for disclosure. In fact, some of our main variables are correlated with these characteristics.
To alleviate concerns about correlated omitted variable bias, we control for these characteristics in our analyses."
Clearly, by identifying ``concerns about correlated omitted variable bias," the authors are affirming their interest in estimating structural or causal effects.

\subsection{Causal inference: Recent developments}

% I think we could do a brief overview of Pearl's SCM here.
% We could use this to:
% - explain that inferring causation is not a purely statistical exercise.
% - introduce the technology of causal graphs
% - other things?

% I think we want to emphasize the importance of stepping back from statistics and econometrics and thinking about causation more carefully.
% I think reading the first 10 pages or so of Pearl's survey paper sent yesterday (2014-03-12) to get some flavor of the ideas here.

% \subsection{Endogeneity}
Recent decades have seen a great deal of research on causal inference fields as diverse as epidemiology, sociology, statistics, and computer science. 
Work by \citet{Rubin:1974im,Rubin:1977dv} and Holland (1986) formalized ideas from the potential-outcome framework of Neyman (1923) to develop the Rubin causal model. %TODO: Get reference.
Other fields have used path analysis, as initially studied by geneticist Sewell Wright (1921), as an organizing framework.
%TODO: Get reference.

In economics and econometrics, while the status of causal notions has occasionally been unclear, early proponents of structural models were quite clear about the causal interpretation of these models.
As discussed by \citet{Heckman:2015ez}, \citet{Haavelmo:1943cl,Haavelmo:1944jq} studies a structural model ``based on a system of structural equations that define causal relationships among a set of variables."
%standard econometric texts generally avoid explicit discussion of causation.
%For example, Greene (2003) does not discuss causality except for Granger causality, which is widely recognized as a purely statistical notion quite distinct from notions of one variable causing another.
% However, some economists have explic
\citep[p. 979]{Goldberger:1972cq} explores a similar notion: ``By structural equation models, I refer to stochastic models in which each equation represents a causal link, rather than a mere empirical association \dots
Generally speaking the structural parameters do not coincide with coefficients of regressions among observable variables, but the model does impose constraints on those regression coefficients."
\citep[p.979]{Goldberger:1972cq} focuses on linking such approaches to the path analysis of Wright.

More recently, \citet{Pearl:2009kh} has sought to integrate these perspectives into a single analytical framework based on graph theory and probability.
Pearl's framework, which he calls the structural causal model, uses directed acyclic graphs (DAGs) to describe causal relationships.
\citet{Pearl:2009kh} shows that the structural causal model provides a framework for analyzing observational data and the valid causal inferences that can be derived from a given causal graph.
\citet[p. 698]{Pearl:2011jd} points out that his framework has been ``adapted warmly" by epidemiologists, sociologists, and statisticians.

Recent developments have led to increased recognition that causal reasoning is distinct from statistical reasoning.
To see this, consider the following simple model
\[ y = x \beta + \epsilon \]
with $x \sim N(0, \sigma_x^2)$ and $\epsilon  \sim N(0, \sigma_{\epsilon}^2)$.
If we have reason to believe that $\mathbb{E}[x \cdot \epsilon] = 0$, then OLS regression will yield an unbiased estimate of $\beta$, to which we might give a causal interpretation.
However, simple algebra allows to rewrite the model above as 
\[ x = y \alpha + \nu \]
with $y \sim N(0, \sigma_y^2)$ and $\nu  \sim N(0, \sigma_{\nu}^2)$.
Given the assumptions we made above, $\mathbb{E}[y \cdot \nu] = 0$ and OLS regression will yield an unbiased estimate of $\alpha$, to which we might give a causal interpretation.

But either $x$ causes $y$ or $y$ causes $x$. 
How do we distinguish these two possibilities?
Clearly we cannot do so on purely statistical grounds, as there is no basis for distinguishing between these two models on such grounds.
Instead, we would use our understanding of the phenomenon, including institutional knowledge and existing theory to cast one model or the other as the plausible one.\footnote{
Note that we would have already used such information to motivate the assumptions that $\mathbb{E}[x \cdot \epsilon] = 0$ and $\mathbb{E}[y \cdot \nu] = 0$.
Additionally, if $x$ and $y$ are jointly determined, then these assumptions become implausible.}
In this way, the structural model we put forth embeds our assumptions about what causes what, so $y = x \beta + \epsilon$ can be viewed in the sense used by \citep[p.979]{Goldberger:1972cq}, as meaning that $x$ causes $y$.

Another difference between a structural, or causal, model and more statistical view is that there is no issue in principle with having a correlation between $X$ and $\epsilon$ ($\mathbb{E} [x \cdot \epsilon] \neq 0$) in the structural model.
This fact may imply that our ability to obtain an unbiased estimate of $\beta$ from observational data is compromised, but does not imply that $y = x \beta + \epsilon$ is somehow not a valid structural model.

\subsection{Causal graphs: A primer}
One of the products of the research of \citet{Pearl:2009kh} and others is the causal graph.
\citet{Pearl:2009kh} shows how graphs can be used to encode causal assumptions and how such causal graphs can be viewed as a non-parametric structural model.
% One way to assess the reasonableness of causal inferences in accounting research is to use the framework developed by \citet{Pearl:2009kh}.
Pearl identifies straightforward criteria that can be applied to a causal diagram to allow a researcher to deduce what causal inferences can be drawn from a given research design.
Given a correctly specified causal graph, these criteria can be used to verify conditioning strategies, instrumental variable designs, and mechanism-based causal inferences.\footnote{While \citet[p.248]{Pearl:2009kh} defines an instrument in terms of conditional independence criteria applied to causal graphs, additional assumptions are often needed to estimate causal effects using an instrument \citep{Angrist:1996p7456}.}

\subsubsection{Causal diagrams: Some terminology}
A causal graph is a directed, acyclic graph (DAG) consisting of nodes and edges.
A node represents a random variable and edges connect variables.
An edge in causal graph is directed, with an arrow pointing from one node to another and representing a causal relation between these variables running in the direction of the arrow.\footnote{
That arrows have a direction accounts for the ``D" in DAG, and that there are no cycles (e.g., $X \rightarrow Y \rightarrow Z \rightarrow X$) accounts for the ``A" element.}
A distinction is made between observed and unobserved random variables.
In some cases, an unobserved joint determinant of two random variables will not be explicitly represented, but replaced by a dashed, undirected edge between those two random variables.

\citet{Pearl:2009kh} shows that, if we are interested in assessing the causal effect of $X$ on $Y$, that we may be able to do so by conditioning on a set of variables, $Z$, that satisfies the ``back-door criterion" \citep[p.79]{Pearl:2009kh}.
While conditioning is much like the standard notion of ``controlling for" variables by including them as additional regressors in OLS regression, there are critical differences.
First, reflecting the non-parametric nature of causal diagrams in their most general form, conditioning in principle means estimating effects for each distinct level of the set of variables in $Z$.
Second, the inclusion of a variable in $Z$ may cause the back-door criterion to be violated. 

\begin{definition}[Back-door criterion]
A set of variables $Z$ satisfies the \emph{back-door criterion} relative to a an ordered pair of variables $(X, Y)$ in a 
	DAG $G$ if:
	\begin{itemize}
		\item no node in $Z$ is a descendant of $X$; and
		\item $Z$ blocks every path between $X$ and $Y$ that contains an arrow into $X$.\footnote{The ``arrow into $X$" is the portion of the definition that is explains the ``back-door" terminology.}
	\end{itemize}
\end{definition}
Given this criterion, \citet{Pearl:2009kh} proves the following result.
\begin{theorem}[Back-door adjustment]
	If a set of variables $Z$ satisfies the back-door criterion relative to $(X, Y)$, then the causal effect of $X$ on $Y$ is identifiable and is given by the formula 
	\[ P(y | x) = \sum_{z} P(y | x, z) P(z), \]
where $P(y|x)$ stands for the probability that $Y = y$, given that $X$ is set to level $X=x$ by external intervention.
% Need to think if there's a more intuitive way to write the P( | ) notation, or perhaps just omit it altogether.
\end{theorem}

% OF COURSE -- THE THEOREM ASSUMES THE MODEL FOR X,Y, AND Z IS ACTUALLY THE CORRECT MODEL!


%CHANGE FIGURE 1 TO REPLACE "C" WITH "Z"

As the back-door criterion is relatively abstract, we use Figure \ref{fig:basic} to illustrate on its application and to demonstrate what is meant by the term ``block" in the definition above.
In Figure \ref{fig:basic} we assume very simple causal graphs in which we are interested in estimating the causal effect of $X$ on $Y$ in the presence of a third variable, $Z$ that is related to $X$ and $Y$ in some fashion.
These causal diagrams are straightforward as all variables are observable and there are just three variables and in all cases, there is a hypothesized causal link between $X$ and $Y$.
The only difference between the three graphs is in the direction of arrows linking either $X$ and $Z$ or $Y$ and $Z$.

Applying the back-door criterion to Figure \ref{fig:confound} is straightforward. The set of variables $\{Z\}$ or simply $Z$ satisfies the criterion, as $Z$ is not a descendant of $X$ and $Z$ blocks the back-door path $X \leftarrow Z \rightarrow Y$.
So by conditioning on $Z$, we can estimate the causal effect of $X$ on $Y$.
This situation is a generalization of linear model in which $Y = X \beta + Z \gamma + \epsilon_Y$ and $\epsilon_Y$ is independent of $X$ and $Z$, but $X$ and $Z$ are correlated.
In this case, it is well known that omission of $Z$ would result in a biased estimate of $\beta$, the causal effect of $X$ on $Y$, but by including $Z$ in the regression, we get an unbiased estimate of $\beta$.
In this situation, $Z$ is a \emph{confounder}.


%  I DO NOT UNDERSTAND THE "PHI" SET -- WHERE DOES THIS COME FROM? I AM PROBABLY MISSING SOMETHING OBVIOUS HERE.

Turning to Figure \ref{fig:mech}, we see that $Z$ does not satisfy the back-door criterion, because $Z$ is a descendant of $X$.
However, $\emptyset$ (i.e., $\{\}$ does satisfy the back-door criterion.
Clearly, the empty set contains no descendant of $X$.
Furthermore, the only path other than $X \rightarrow Y$ that exists is $X \rightarrow Z \rightarrow Y$, which does not have a back-door into $X$.
Note that the back-door criterion not only implies that we need not condition on $Z$ to obtain an unbiased estimate of the causal effect of $X$ on $Y$, but that we must not condition of $Z$ to get such an estimate.

% MIGHT BE UESFUL TO COMPARE TO THE WELL-KNOWN DIRECT AND INDIRECT EFFECTS IN PATH ANALYSIS

Finally in Figure \ref{fig:collider}, we have $Z$ acting as what Pearl refers to as a ``collider" variable.\footnote{
The two arrows from $X$ and $Y$ ``collide" in $Z$.} % TODO: Reference?
Again, we see that $Z$ does not satisfy the back-door criterion, because $Z$ is a descendant of $X$.
However, $\emptyset$ (i.e., $\{\}$ does satisfy the back-door criterion.
Again, the empty set contains no descendant of $X$.
Furthermore, the only path other than $X \rightarrow Y$ that exists is $X \rightarrow Z \leftarrow Y$, which does not have a back-door into $X$.
Again, the back-door criterion not only implies that we need not condition on $Z$, but that we must not condition of $Z$ to get an estimate of the causal effect of $X$ on $Y$.

% Next relate the Pearl stuff to an accounting setting.
 

WE NEED TO SHOW HOW TO APPLY THE PEARL FRAMEWORK IN AN ACCOUNTING SETTING.

HOLLANDER SEEMS ODD HERE -- I MOVED THIS TO WHERE WE DISCUSS THE SURVEY EVIDENCE

DO WE HAVE SOME IDEA ABOUT WHERE HOLLANDER SHOULD HAVE CONTROLLED FOR SELECTED VARIABLES AND OTHER CASES WHERE THEY SHOULD NOT?  THIS SEEMS TO BE THE PART OF PEARL THAT WE HAVE PULLED INTO THE TEXT

HOW ABOUT USING THIS HERE?

\citet{Kelly:2012ih} treat mergers of brokerage firms that covered treatment firms as an exogenous (i.e., as-if random) source of variation in analyst coverage.%\footnote{Seems to me that we should highlight some of these issues with this instrument -- not clear why this is outside our scope.} 
They then examine the effect of such changes in analyst coverage on information asymmetry.

The causal graph for \citet{Kelly:2012ih} is depicted in Figure \ref{fig:bbkl}.
The identifying assumption is that $\textit{Brokerage closure}_t$ affects $\textit{Analyst coverage}_t$, but otherwise has no effect, direct or otherwise, on $\textit{Information asymmetry}_t$.
These three boxes, along with the arrows (or lack of arrows) between them are the causal graph underlying \citet{Kelly:2012ih}.

An assumption in \citet{Kelly:2012ih} is that analyst coverage affects contemporaneous information asymmetry, which again affects contemporaneous liquidity.
These relations are represented in Figure \ref{fig:bbkl} by the three nodes in the right-most column and the nodes between them.

Then discuss why this identification strategy  breaks down


MAYBE ALSO USE SOME OF THE HOLLANDER AS WELL?


We focus on the two explanatory variables for which \citet{Hollander:2010jg} find evidence of effects are firm performance, measured as the change in sales, and disclosure agency problems, measured using the level of stock price-based compensation.



Second, we note that the authors are quite explicit about the concerns they have, which can be understood in terms of Figure \ref{fig:basic}.
In that figure we provide three alternative sets of relations between the treatment variable, $X$, the outcome variable $Y$, and other variables correlated with $X$ and $Y$, which we label $C$.
<<<<<<< HEAD
The typical approach in accounting research is to identify any potential correlated variables and include them in the regression.
One control that 
=======
The typical approach in accounting research is to identify any variables.
One control that % TO BE CONTINUED.



>>>>>>> a8ebe4eb7530ff8261393ce10c656ffc79722ee3

%TODO: Dave: We want some general "endogeneity" talk here. Then we can introduce the SCM as a tool for evaluating and understanding causal issues. We can discuss Rubin Causal Model, etc., as I believe Pearl relates those to his framework.




% \subsection{Where do research questions come from?}
% One striking aspect of 
% Sources of research questions
%
% - theory
% - prior research: replications
% - prior research: resolution of debates
% - prior research
% - practice





\section{Quasi-experimental methods in accounting research}

% TODO: somewhere we want to note that there are some real field experiments -- John Roberts with the impact of consulting firms in India and a variety of related studies in Economics.  There was also the SEC decision to randomly assign short selling restrictions.

\subsection{Natural experiments}
Natural experiments occur when observations are assigned by nature (or some other force outside the control of the researcher) to treatment and control groups in a way that is random or ``as if'' random \citep{Dunning:2012tt}. 
Truly (as if) random assignment to treatment and control provides a sound basis for causal inference, enhancing the appeal of natural experiments for social science research.
However, argues that this appeal ``may provoke conceptual stretching, in which an attractive label is applied to research designs that only implausibly meet the definitional features of the method'' \citep[p.3]{Dunning:2012tt}.

Our survey of accounting research identified six papers that exploited either a ``natural experiment'' or ``exogenous shock'' to identify causal effects \citep{Lo:2013jk,Aier:2014ii,Kirk:2014gx,Houston:2014hv}. %TODO: IDG: What about Hail:2014fq?
But closer examination suggests that most of these papers misapply the fundamental idea of natural experiments  In fact, these papers seem to be good examples of ``conceptual stretching''.

\cite{Aier:2014ii} exploit a 1991 Delaware court that ``expanded the scope of directors' fiduciary duties to include creditors when a Delaware incorporated firm is in the `vicinity of insolvency.'" as a ``natural experiment'' for the purpose of understanding the causal effect of debtholders' demand for conservatism (the treatment variable) on financial reporting conservatism (the outcome of interest). But it is completely unclear how this ``natural experiment'' sorted firms into differing levels of the treatment variable, let alone why this assignment is appropriately considered to as-if random.

\citet{Kirk:2014gx} ``exploit the natural experiment setting created by the exogenous shock of Reg FD to examine the effect of Reg FD on firms with an established IR [investor relations] program.'' 
Given that the treatment of interest in \citet{Kirk:2014gx} is the establishment of an IR program, only a event that randomly assigned firms to having or not having such a program would qualify as a natural experiment for this research setting.

A plausible explanation for the ease with which conceptual stretching has occurred derives from the ambiguity of the word ``exogenous,'' which not only denotes  ``of, relating to, or developing from external factors'' (Oxford Dictionary), but is also the antonym of ``endogenous.''
For example, the fact that Reg FD was perhaps not driven by factors related to firms' IR programs and firm-level capital market outcomes, it does not immediately imply random assignment of firms into IR treatment and control groups and thus does not help resolve the endogeneity of IR programs with such capital market outcomes.

\cite{Houston:2014hv} analyzes ``whether the political connections of listed firms in the United States affect the cost and terms of loan contracts.'' They argue that ``the recent financial crisis can be viewed as a major exogenous shock, the effects of which may vary depending on whether the firm is politically connected.'' 

GIVE A BETTER EXPLANATION THAN THE SMOKING STORY

The "natural experiment" of the financial crisis did not randomly assign firms into politically connected and non-connected treatments.  The shock may well have increased the power of some statistical tests, but it does not help with endogeneity.


%But this is not what is needed for a valid natural experiment. To see this, an analogy is perhaps helpful. Suppose we wanted to study the effects of smoking on life expectancy. A long-standing concern in studies of such effects is the existence of other differences in the lifestyles of smokers and non-smokers. A ``natural experiment'' analogous to that in \cite{Houston:2014hv} might be one that created gas leaks in the homes of smokers and non-smokers alike. Because gas leaks are likely to have more deleterious consequences on smokers (e.g., instant immolation when lighting a cigarette), the reasoning of \cite{Houston:2014hv} might suggest that gas leaks are a helpful ``exogenous shock,'' contrary to common sense. %TODO: IDG: Tone this down.


GENERAL EVALUATION for the use of NATURAL EXPERIMENTS for causal inferences in accounting 

Are there any good examples of natural experiments in accounting?  Should we put some "good examples" of natural experiments from other areas -- economics and finance?  Are some of the accounting natural experiments better than others?

% IV starts here!
\subsection{Instrumental variables}
\citet[p.114]{Angrist:2008vk} describe instrumental variables (IV) as ``the most powerful weapon in the arsenal of [statistical tools]" in econometrics. 
Accounting researchers have long used instrument variables to address concerns about endogeneity \citep{Larcker:2010fq} and continue to do so.  Our survey of research published in 2014 identifies 10 papers using instrumental variables \citep{Cannon:2014im,Cohen:2014jl,Kim:2014fm,Vermeer:2014bs,Fox:2014io,Guedhami:2013cj,Houston:2014hv,deFranco:2014ct,Erkens:2014hj,Correia:2014fp}. 

However, much has been written on the challenges for researchers in using instrumental variables (IV) as the basis for causal inference \citep[e.g.,][]{Roberts:2013cz}. 
With respect to accounting research, \citet{Larcker:2010fq} lament that ``some researchers consider the choice of instrumental variables to be a purely statistical exercise with little real economic foundation'' and call for 
``accounting researchers \dots to be much more rigorous in selecting and justifying their instrumental variables.'' 
\citet[p.117]{Angrist:2008vk} argue that ``good instruments come from a combination of institutional knowledge and ideas about the process determining the variable of interest."
One study that illustrates this is \cite{Angrist:2008vk}.
In that setting, the draft lottery is well understood as random and the process of mapping from the lottery to draft eligibility is well understood.
Furthermore, there are good reasons to believe that the draft lottery does not affect anything else directly except for draft eligibility.\footnote {Of course, this seemingly "ideal" instrument has been subject to considerable criticism. I think the idea is that even if you had a low draft number it was not clear that you actually went to the army.  In fact, upper class kids did not go (and the probably had much better skills) than the white hillbilly trash and minorities that ended up going to VN}
%TODO: ADD



\subsubsection{Evaluating IVs is \emph{not} a statistical exercise}
It is evident that many researchers in accounting view causal inference as a purely statistical exercise.
Most accounting papers using IV methods provide little justification for the validity of their chosen instruments (especially why they are expected to satisfy the important exclusion restriction) and tend not to evaluate other critical features such as whether the instruments are weak and overidentifying restrictions. Although perhaps obvious, the standard statistical tests associated with instrumental variable applications provide little insight into the quality of instruments.  

%\citet{Correia:2014fp} is relatively thorough. \citet{Correia:2014fp} tests for weak instruments and also uses a test of over-identifying restrictions to examine the validity of her chosen instruments. 

It is easy to illustrate this point with a simple simulation exercise.
Suppose that we are interested in a model such as $y = X \beta + \epsilon$, but with $X$ and $\epsilon$ having correlation $\rho(X, \epsilon) = 0.2$ (i.e., $X$ is endogenous) and $\beta = 0$ (i.e., there is no causal relation between $X$ and $y$). 
Now, suppose we "construct" the following three instruments 
$z_1 = x +\eta_1$, $z_2 = \eta_2$, and $z_3 = \eta_3$, with $\sigma_{\eta_1} = \sigma_{\eta_2} = \sigma_{\eta_3} \sim N(0, 0.09)$ and independent. 
That is, $z_1$ is $X$ plus noise, while $z_2$ and $z_3$ are random noise.  Obviously, these "instruments" are silly choices and completely inappropriate.

We finally  Assume that $X$ and $\epsilon$ are bivariate-normally distributed with variance of $1$. We then run 1000 simulations and  estimate the IV regression using these instruments on the simulated data in each case. Doing so, we find a mean estimated coefficient on $X$ of $0.201$, which is statistically significant at the 5\% level 100\% of the time.\footnote{Note that this coefficient is close to $\rho(X, \epsilon) = 0.2$, which is to be expected given how our data were generated.} Based on a test statistic of 30, which easily exceeds the thresholds suggested by Stock et al. (2002), the null hypothesis of weak instruments is rejected 100\% of the time. 
The test of overidentifying restrictions fails to reject a null hypothesis of valid instruments (at the 5\% level) 95.7\% of the time.
In other words, it is easy for completely spurious instruments to deliver bad inferences, yet easily pass tests for weak instruments and endogeneity.\footnote{It is also common for accounting researchers to claim that they have established the validity of their instruments by implementing some type of Hausman test.  It is very clear that these types of overidentifying tests require the researcher to actually have one valid IV.  In general, there is no test that enables a researcher to verify that their IVs satisfy the exclusion restriction.}


I AM TEMPTED TO ADD THE CORE JAR PAPER THAT BORROWS FROM THE AER PAPER IN THE SECTION BELOW.  MAYBE USE THE PEARL GRAPH ON THIS ONE?

\subsubsection{Using instruments from other papers should be done with care}
One popular source of instruments in accounting, finance, and economics is prior research in economics and finance.
For example, \citet{Balakrishnan:2014js} use the ``exogenous shocks" to analyst coverage in \citet{Kelly:2012ih} as an instrument for changes in voluntary disclosure practices. 
\citet{Kelly:2012ih} seek to understand the effect of changes in analyst coverage on information asymmetry. 
The challenge faced by \citet{Kelly:2012ih} is that changes in analyst coverage are generally not random and may be correlated with information asymmetry due to omitted correlated variables.
\citet{Kelly:2012ih} treat mergers of brokerage firms that covered treatment firms as an exogenous (i.e., as-if random) source of variation in analyst coverage.%\footnote{Seems to me that we should highlight some of these issues with this instrument -- not clear why this is outside our scope.} 
They then examine the effect of such changes in analyst coverage on information asymmetry.


SEEMS LIKE WE NEED MORE THAN JUST FIGURE 3, AS WE ARE TALKING ABOUT THE ABSENCE OF SOME LINKS AND WHAT HAPPENS IF YOU ADD A LINK.  MAYBE DO AN "EVOLUTION" OF CURRENT FIGURE 3? MAYBE JUST THE KL DIAGRAM IN THE PEARL SECTION ABOVE, WHAT IS BEING ASSUMED BY BALA ET AL, AND THEN SOME EXAMPLES WHERE THERE ARE OBVIOUS LINKS AND WHAT HAPPENS ALA PEARL.  MAKE SENSE?


The causal graph for \citet{Kelly:2012ih} is depicted in Figure \ref{fig:bbkl}.
The identifying assumption is that $\textit{Brokerage closure}_t$ affects $\textit{Analyst coverage}_t$, but otherwise has no effect, direct or otherwise, on $\textit{Information asymmetry}_t$.
These three boxes, along with the arrows (or lack of arrows) between them are the causal graph underlying \citet{Kelly:2012ih}.

Turning to \citet{Balakrishnan:2014js}, as the authors note, the critical identifying assumptions are that ``lagged coverage shocks a) lead to more disclosure, b) not affect liquidity directly, and c) not correlate with some omitted variable that in turn affects liquidity." 
The link between coverage shocks and disclosure is the assumption that ``managers respond to exogenous shocks to their information environments" (i.e., the increase in information asymmetry shown by  \citet{Kelly:2012ih}.
These link can be expressed by adding to the causal graph in Figure \ref{fig:bbkl} a node for \textit{Disclosure}, driven by $\textit{Information assymetry}_t$, which is linked to $\textit{Liquidity}_{t+1}$.

As \citet{Balakrishnan:2014js} point out, it is necessary for $\textit{Information assymetry}_t$ not to affect $\textit{Liquidity}_{t+1}$ through any channel other than \textit{Disclosure}.
For example, it is well known that information asymmetry is a major driver of contemporaneous liquidity, so an assumption is that neither $\textit{Information assymetry}_t$ nor $\textit{Liquidity}_t$ affects $\textit{Liquidity}_{t+1}$ except through their effect on $\textit{Disclosure}$. 
This assumption is represented in Figure \ref{fig:bbkl} by the omission of an arrow between $\textit{Information assymetry}_t$ and $\textit{Information assymetry}_{t+1}$.

An assumption in \citet{Kelly:2012ih} is that analyst coverage affects contemporaneous information asymmetry, which again affects contemporaneous liquidity.
These relations are represented in Figure \ref{fig:bbkl} by the three nodes in the right-most column and the nodes between them.
However, if analyst coverage is sticky (i.e., $\textit{Analyst coverage}_t$ affects $\textit{Analyst coverage}_{t+1}$ as represented in Figure \ref{fig:bbkl}), then the identification strategy in \citet{Balakrishnan:2014js} breaks down, as there is then a back-door path from $\textit{Liquidity}_{t+1}$ to $\textit{Disclosure}$ via this link.
That such stickiness in coverage should exist seems to be assumed by \citet{Balakrishnan:2014js}, who argue that ``firms [unable to respond  to the coverage shock through increased disclosure] suffer a \emph{permanent} reduction in liquidity" (p. 2239), which presumably requires the shock to analyst coverage to be persistent.

The point of the discussion above is not to impugn the precise findings of \citet{Balakrishnan:2014js}, but rather to illustrate the merit of more careful analysis of a paper's identification strategy and, we argue, the value of causal graphs in doing so.


A review of these papers suggests that researchers have paid little heed to the suggestions and warnings of  \citet{Larcker:2010fq} and \citet{Roberts:2013cz}.

Several papers provide little or no justification for the validity of their instruments. For example, to address endogeneity \citet{Cohen:2014jl} use ``two instrumental variables. The first is the natural log of industry size, measured as the number of companies within each two-digit SIC. The second measures industry competition using the Herfindahl-Hirschman index, which is well-established as a measure of competitive industries. Our untabulated results using this approach are qualitatively similar to our main analysis, thus indicating that endogeneity is not a concern when assessing the reliability of our findings.''
\citet{Vermeer:2014bs} ``use Maddala's (1988) two-stage procedure'' in order to ``control for endogeneity'' without providing any explanation at all and in fact seem to be assuming the each of three endogeneous variables can used as an instrument for the other two.
\citet[p.48]{Fox:2014io} state in a footnote that they ``instrumented for the price index employing a two stage least squares estimator'' without further details, simply noting that their ``conclusions are robust with respect to these concerns.''
\citet{Cannon:2014im} uses ``industry-level capacity unit cost and selling price changes'' as instruments for firm-level capacity unit cost changes with no more justification than the fact that these ``are outside management's control.'' But being outside management control does not make a variable a an adequate instrumental variable. without some assessment for how this fits into the economic setting. %ADD HERE

%  THIS IS WHAT WAS IN THE SENTENCE -- in an econometric sense.

Other papers provide limited, but seemingly flawed, justification for their chosen instruments. 
 \citet{Guedhami:2013cj} use $\textit{CAPITAL}$, an indicator for a firm being located in a capital city, as an instrument for political connectivity in a study looking at the effect of political connections on the use of a Big 4 auditor ($\textit{BIG 4}$).
 The only justification \citet{Guedhami:2013cj} provide in support of the required exclusion restriction  is that ``importantly, the correlation between $\textit{CAPITAL}$ and $\textit{BIG 4}$ is small in our data set $(\rho = 0.05)$, helping to justify the validity of this exclusion restriction.''
 \citet{Guedhami:2013cj} cite \citet{Larcker:2010fq} as a reference for this approach, even though \citet{Larcker:2010fq} carefully explain why simple tests like this cannot be used to justify instruments.
 
 \citet{Houston:2014hv} use variables variables that are related to the location of the company's headquarters as instruments for political connection and argue that ``these instruments should not be conceptually related to loan spreads. The key insight here is that the geographic locations of headquarters for companies are predetermined and are unlikely to affect banks' financing decision on loan costs. In summary, our identification assumption is that the costs of bank loans are not directly related to the companies' geographic locations, after controlling for a series of firm and loan characteristics'' (p.228). In justifying the relevance of the instrument, the authors seem eager to justify a connection, suggesting that ``the presumption is that the company's geographic location affects the company's ability to attract politically connected directors.'' But it far from clear why a company's geographic location would not also affect the its ability to attract directors with connections to \emph{financial institutions}, which plausibly affects financing terms directly \citep{Guner:2008tp}.\footnote{\citet{Houston:2014hv} also use firm age as an instrument, arguing that ``firm age affects a firm's incentive and capability in building up political connections''; but it is not clear why firm age would not also affect a firm's ``incentive and capability in building up'' financial connections.} 
 
Researchers tend to very unclear about the determinants of their selected instruments.  In many cases, it seems that the instruments are also endogenous.  This makes it very difficult to to rule out the possibility that the instrument directly affects (or is correlated with) variables other than the endogenous variable of interest.
 \citet{Erkens:2014hj} ``use the following three instrumental variables that capture the extent to which lenders are more likely to serve on a firm's board, which is studied for its potential effect on accounting conservatism. We use \emph{Industry importance to primary lender} because industry specialization increases the importance of acquiring information about a firm's industry, \emph{Primary lender within 50 mile radius} because physical proximity to lenders' headquarters reduces the cost of serving on the board, and \emph{Number of commercial banks within 50 mile radius} because the close proximity of multiple banks increases competition for board seats from other lenders.'' If industry specialization affects information-acquisition incentives, it seems it would do so through channels outside of board membership. With respect to the second instrument, it's quite likely that proximity affects information-gathering independent of service on the board. With respect to the third instrument, it is also implausible that the only direct effect of this variable is one on the service of bankers on the board (for example, this may lead to lower search costs in choosing potential lenders).
 
In some cases, arguments that seek to justify a set of instruments seem to provide reasons to believe that they \emph{not} valid. \citet{Kim:2014fm} use director age as an instrument for director tenure in a study examining the effect of the latter on firm performance. 
``Importantly, research finds little or no association between age and performance \dots and a small negative association between age and executive functions \dots. 
Related to directors, Ferris et al. (2003) suggest that any positive effects from director experience increasing with age may be offset by older directors having less energy, posing a last-period risk, and viewing directorships as lucrative part-time jobs for their retirement years.'' 
But these arguments seem to invalidate age as an instrument for tenure. 
For age to be a valid instrument, there should be no unblocked causal path between age and performance except for the path via tenure.
 That possible positive effects may be offset by negative effects and thus detecting an association between age and performance is not a valid basis for claiming age to be a valid instrument. \citet{deFranco:2014ct} ``find that the number of covenants is positively related to the interest rate, likely due to endogeneity between the interest rate and covenants.'' To address this using they use ``the number of covenants by calendar year indicators as the instrument'' for the number of covenants. Apart from the issues with using an average as an instrument discussed in \citet{Reiss:2007ej}, the authors justify their instrument by suggesting that ``the strictness of covenant packages significantly deteriorated during the years of the credit boom that preceded the financial crisis.'' But it seems likely that the credit boom would have a direct effect on interest rates on bond issues. 

SEEMS LIKE WE NEED TO COMBINE THE AVG AS AN IV TOGETHER IN THESE TWO PARAGRAPHS

\citet{Correia:2014fp} uses ``average level of political contributions made by the other firms in the same industry'' as an instrument for political contributions by a firm, as well as two additional instruments: ``the percentage of sales made to the government, and the number of years in the previous five years in which there was a close election involving two candidates in the firm's state.'' With regard to the first instrument, the reasons that cause political contributions to be endogenous may well affect all firms in an industry (or at least be correlated across firms within an industry). 
As such \citet{Reiss:2007ej} suggest that there is little reason to view such averages as valid instruments.

% I agree with this!!! (IDG)
GENERAL EVALUATION for the use of IV for causal inferences in accounting.  

The classic textbook solution to endogeneity and causal inference does not seem like a viable approach for most observational accounting research.  Most of the selected instruments are unlikely to satisfy the necessary exclusion restriction.  The prospects for reasonable use of IVs in accounting research seem quite low.

\subsection{Regression discontinuity designs} 
%\textbf{TBD.} Discuss RD d, how it works, but issues in applying it and the fact that it has limited applicability in general (i.e., need a discontinuity).

In discussing the recent ``flurry of research" using regression discontinuity (RD) designs, \citet[p.282]{Lee:2010hya} point out that they ``require seemingly mild assumptions compared to those needed for other nonexperimental approaches \dots and that causal inferences from RD designs are potentially more credible than those from typical `natural experiment' strategies." 
Recently, RD designs have attracted the interest of accounting researchers, as a number of phenomena of interest to accounting researchers involve discontinuities. For example, whether an executive compensation plan is approved is a discontinuous function of shareholder support \citet{Armstrong:2013io} and whether a firm had to comply with provisions of Sarbanes-Oxley Act in 2004 \citep{Iliev:2010ic}, and shareholder voting for approval of management proposals AGL (20xx)

While RD designs make relatively mild assumptions, in practice these assumptions may be violated. In particular, manipulation of the running variable may occur.

It is also important to note that various ``quasi-RD" designs bear little resemblance to RD designs. We need to explain and expand on this important point.
% Listokin, McCrary, etc.  

Point out that the estimated treatment effect is very local, which may or may not be the estimate desired.  Most of the accounting research is looking for average effects across the entire sample, not local effects.  However, the local effect is useful if RDD enables us to make statements that are close to being causal.

Seems like we want to include in this section:  plot the data and if you can't see it in the data, it probably is not actually there (Imbens), do not use the high level polynomial approach, and other similar issues.  Probably something on whether the magnitude of the results is actually believable -- Yonca's, the prize winning JF paper, and others results seem implausibly large for governance topics.


I believe that you have some type of simulation where the RDD does not work when the threshold is endogenously selected (75 million dollar cap and other similar thresholds).  This would be good to add similar to the simple IV example above.

GENERAL EVALUATION for the use of RDD for causal inferences in accounting 


\subsection{Propensity score matching}

Should we discuss this too? PSM is not really a ``quasi-experimental'' method in the sense of the above methods except under assumptions that are about as restrictive as those needed to deliver causal estimates using OLS (basically OLS without linearity). That is, it does \emph{not} solve endogeneity, but many believe that it does.

I like the first stage results of PSM -- this can be informative, but you have to be careful not to "toss in the kitchen sink"  Show we critique the Defond working paper on why PSM is stupid in an auditing context?

It seems like we should highlight that PSM (or for that matter any matching approach) does not give causal interpretations when there are "correlated omitted variables."  PSM corrects for the observed determinants, but not the unobserved determinants.  I believe it would be good to discuss briefly the bounding approach here -- Manski-style, Rosenbaum, Frank methods that are in the AJL JAR paper.  I realize that we do not have precise criteria for deciding whether results are questionable if some omitted variable correlation causes things to change, but I believe such bounding is useful for evaluating possible causal interpretations.
% I don't think bounding should be conflated with PSM.

%TODO: general evaluation of the use of PSM for causal inferences in accounting 


\subsection{Overall evaluation} 
\textbf{TBD.} Quasi-experimental methods are very poorly applied by accounting researchers. Even if accounting researchers knew what they were doing, the reality is that instruments, natural experiments, and discontinuities do not grow on trees. So quasi-experimental methods are unlikely to deliver anything like the 100+ papers that accounting researchers crank out every year (in economics---a much broader field---it's the same handful of examples brought out over and over).

The next part of the paper has a more positive take (structural modeling, etc.).

Likely effective in settings, but of limited applicability and care is needed to do it well and to interpret the results. Discuss examples from shareholder voting and the \$75 million threshold for SOX.

\subsection{Overall assessment}

We agree that the revolution in econometric methods for causal inference has certainly been an exciting development.\footnote{Not sure about ``revolution''; need to look at \emph{Mostly Harmless Econometrics} to get the right term here.} However, we have two grave concerns with regard to this methods in accounting research. First, it is evident that accounting researchers understand these methods only poorly and frequently seek to apply them inappropriately. Second,  it is far from clear that these methods, properly applied, can support more than a tiny fraction of accounting research. Of more than a 100 papers published in the top three accounting journals in 2014, we identified just a small fraction that applied these approaches and the vast majority of these did so in ways that seem difficult to classify as appropriate. Accounting researchers license to run IV should be revoked.

\section{Alternative approaches}

\begin{quotation}
In complex fields like the social sciences and epidemiology, there are only few (if any) real life situations where we can make enough compelling assumptions that would lead to identification of causal effects.
\attrib{Judea Pearl, cited in \citealt[p.\,287]{Freedman:2004ix}}
\end{quotation}

In the first half of the paper, we have argued that, while causal inference is the goal of most accounting research using observational data, there exist no research designs and statistical methods that yield output that can be viewed as unbiased estimates of causal effects using observational data in all but exceptional circumstances.
This perspective may cause a reader to ask: 
So, what should researchers do? Do we stop doing research? Do we need to give up on causal inference? 
We believe that this is too pessimistic and that there are viable paths forward that do not rely on researchers identifying ``clever" identification strategies to answer questions of interest.
The objective of the second part of this paper is to discuss these paths forward.

\subsection{Causal mechanisms: Some examples}

Accounting research is not alone in relying primarily on observational data.
Other fields also seek to draw causal inferences, but need to grapple with the reality of observational data. 
Yet in many cases, these fields have successfully drawn causal inferences.
In the following, we briefly discuss case studies of plausible causal inference in other fields and highlight features that enhanced the credibility of inference.

\subsubsection{John Snow and cholera}
A widely cited case of causal inference involves John Snow's work on cholera.
As there are many excellent accounts of Snow's work, we will focus on the barest details.
As discussed in Freedman %TODO: Get reference to book.
``John Snow was a physician in Victorian London.
 In 1854, he demonstrated that cholera was an infectious disease, which could be prevented by cleaning up the water supply. 
The demonstration took advantage of a natural experiment.
 A large area of London was served by two water companies. 
 The Southwark and Vauxhall company distributed contaminated water, and households served by it had a death rate`between eight and nine times as great as in the houses supplied by the Lambeth company, ' which supplied relatively pure water."

But there was much more to Snow's work than the use of a convenient natural experiment.
First, Snow's reasoning (much of which was surely done before ``the arduous task of data collection" began) was about the  mechanism through which cholera spread. Existing theory suggested ``odors generated by decaying organic material."
Snow reasoned qualitatively that such a mechanism was implausible.
Instead, drawing on his medical knowledge and the facts at hand, Snow conjectured that ``A living organism enters the body, as a contaminant of water or food, multiplies in the body, and creates the symptoms of the disease. Many copies of the organism are expelled with the dejecta, contaminate water or food, then infect other victims."
With a hypothesis at hand, Snow then needed to collect data to prove it.
His data collection involved a house-to-house survey in the area surrounding the Broad Street pump operated by  Southwark and Vauxhall.
As part of his data collection, Snow needed to account for anomalous cases (such as the brewery workers who drank beer, not water).
It is important to note that this qualitative reasoning and diligent data collection were critical elements establishing (to a modern reader) the ``as-if random" nature of the treatment assignment mechanism provided by the Broad Street pump.
This contrasts with the speculative guesses often used to justify natural experiments by modern researchers.

But another important feature of the case of John Snow and cholera is that widespread acceptance of Snow's hypothesis did not occur until compelling evidence of the mechanism was provided.
``However, widespread acceptance was achieved only when Robert Koch isolated the causal agent (\emph{Vibrio cholerae}, a comma-shaped bacillus) during the Indian epidemic of 1883." %TODO: Add cite to Freedman here.
Only once persuasive evidence of a plausible mechanism was provided (i.e., direct observation of microorganisms now known to cause the disease) did Snow's ideas become widely accepted.

\subsubsection{Smoking and heart disease}
A more recent illustration of plausible causal inference is discussed by \citet{Gillies2011-GILTRT-3}.
 \citet{Gillies2011-GILTRT-3} points discusses the paper by \citet{Doll:1976aa}, which studies the mortality rates of male doctors between 1951 and 1971.
 The data that \citet{Doll:1976aa} had ``a striking correlation between smoking and lung cancer" \citep[p.\,111]{Gillies2011-GILTRT-3}.
 \citet{Gillies2011-GILTRT-3} argues that ``this correlation was accepted at the time by most researchers (if not quite all!) as establishing a causal link between smoking and lung cancer. Indeed Doll and Peto themselves say explicitly (p.\,1535) that the excess mortality from cancer of the lung in cigarette smokers is caused by cigarette smoking."
In contrast, while \citet{Doll:1976aa} also had highly statistically significant evidence of an association between smoking and heart disease, they were cautious about drawing inferences of direct causal explanation for the association.
\citet[p.\,1528]{Doll:1976aa} say ``To say that these conditions were related to smoking does not necessarily imply that smoking caused \dots them. The relation may have been secondary in that smoking was associated with some other factor, such as alcohol consumption or a feature of the personality, that caused the disease.''
 
 \citet{Gillies2011-GILTRT-3} then discussed extensive research into atherosclerosis between 1979 and 1989 and concludes that ``by the end of the 1980s, it was established that the oxidation of LDL was an important step in the process which led to atherosclerotic plaques."  Later research established evidence of much higher levels of a new measure (levels of $F_2$-isoprostanes in blood samples) of the relevant oxidation in the body ``provides compelling evidence that smoking causes oxidative modification of biologic components in humans. This conclusion is greatly strengthened by the finding that levels of $F_2$-isoprostanes in the smokers fell significantly after two weeks of abstinence from smoking" \citep[pp.\,1201--2]{Morrow:1995gz}.  \citet[p.\,120]{Gillies2011-GILTRT-3} points out that this evidence did not establish a confirmed mechanism linking smoking with heart disease, because the required oxidation needs to occur in the artery wall, not in the blood stream. But later research established a link: ``Smoking produced oxidative stress. This increased the adhesion of leukocytes to the \dots artery, which in turn accelerated the formation of atherosclerotic plaques" \citep[p.\,123]{Gillies2011-GILTRT-3}.
Thus, a causal link or mechanism between smoking and atherosclerosis was established.

%TODO: Get an example from the social sciences, e.g., political science.

\subsubsection{Implications of cases on mechanism}
 \citet{Gillies2011-GILTRT-3} avers that the process by which a causal link between smoking and atherosclerosis was established illustrates the ``Russo-Williamson thesis."
 \citet[p.\,159]{Russo:2007iz} suggest that ``mechanisms allow us to generalize a causal relation: while an appropriate dependence in the sample data can warrant a causal claim `$C$ causes $E$ in the sample population,' a plausible mechanism or theoretical connection is required to warrant the more general claim `$C$ causes $E$.' Conversely, mechanisms also impose negative constraints: if there is no plausible mechanism from $C$ to $E$, then any correlation is likely to be spurious. Thus mechanisms can be used to differentiate between causal models that are underdetermined by probabilistic evidence alone."

 The Russo-Williamson thesis was arguably also at work in the case of Snow and cholera, where the establishment of a mechanism (\emph{Vibrio cholerae}) was essential before the causal explanation offered by Snow was widely accepted and also in the case of smoking and lung cancer, which was initially conjectured based on associational evidence, but was widely accepted by 1976.\footnote{
 The persuasive force of Snow's natural experiment, coming decades before the work of Neyman and Fisher, might be considered greater today.}

A reader may wonder what relevance the cases above have for accounting research.
Our view is that accounting researchers can learn from fields such as epidemiology and political science. 
These fields grapple with the reality of observational data.
While randomized controlled trials are a gold standard of sorts in epidemiology, in many cases it is unfeasible or unethical to use such trials.
And in political science, it is not possible to randomly assign countries to treatment conditions such as \emph{democracy} or \emph{socialism}.
Yet these fields have often been able to draw plausible causal inferences by establishing clear mechanisms, or causal pathways, from putative causes to putative effects.

We argue that the value of the study of mechanisms is perhaps underestimated, perhaps due to a belief that the path to credible causal inferences involves clever identification strategies.
As an example, many published papers have suggested that managers adopt conditional conservatism as a reporting strategy to obtain benefits such as reduced debt costs \citep{Ahmed:2002aa,Zhang:2008bc}.
But, as \citet[p\,317]{Beyer:2010cj} points out, an ex ante commitment to such a reporting strategy ``requires a mechanism that allows managers to credibly commit to withholding good news or to commit to an accounting information system that implements a higher degree of verification for gains than for losses," yet research has only recently begun to focus on the mechanisms through which such commitments are made \citep[e.g.,][]{Erkens:2014hj}.



\section{Structural Modeling}
Structural empirical models consist of two main elements:
% One alternative approach to empirical research in economics is labeled the structural approach (Wolpin 2013).
% The structural approach ``requires that a researcher explicitly specify a model of economic behavior, that is, a theory." (Wolpin, 2013, p.2).

\begin{enumerate}
\item a theoretical (economic) model of the phenomena
of interest; and,
\item a stochastic model that links the theoretical model to the observed
data.
\end{enumerate}
The theoretical model minimally describes who makes decisions, the objectives of decisions makers,
and constraints on decisions and behavior. 
In developing and analyzing the theoretical model, the researcher decides what conditions (variables) matter, 
what is endogenous and exogenous, and what conditions impact behavior. 
The goal in formulating the theoretical model is to derive a set of equations or inequalities that describe the determinants of decisions. 
Not all these determinants may be observed by the researcher.
The stochastic component of the structural model accounts for determinants the researcher does not observe, as well as why the theoretical model may not perfectly explain the data. 
Additionally, the unobservables may also rationalize why the process that generated the data differs from that envisioned by the theoretical model.

Empiricists employ structural models for several nonexclusive reasons. 
First, structural models are a means by which a researcher can understand what determines behavior and market outcomes. 
Second, structural models make it clear what data are needed to identify unobserved theoretical quantities or objects, such as an individual's degree of risk aversion. 
Third, structural models provide a foundation for estimation and
inference. 
Finally, structural models facilitate counterfactual analyses.
Counterfactual analyses predict what might happen under conditions not observed in the data. 
For example, they might show how accounting reports would change if a new accounting standards were adopted. 
 
We now explore these benefits of structural models in more detail, as well as discuss some of the costs that arise in formulating and estimating structural models.

\subsection{The Structural Approach}

The term structural model originated with economists and statisticians working at
the Cowles Foundation in the 1940s and 1950s.
The earliest structural models showed how price and quantity data could be used to recover unobserved demand and supply curves. 
This literature introduced the issue of identification, and more specifically the idea that (instrumental) variables beyond price and quantity were needed to identify demand and supply schedules. 
The impact of these early models on empirical work in economics encouraged and other social scientists to begin using theoretical models to help interpret data.
Indeed, structural models have been used to study: educational choices, voting, contraception, addiction, and financing decisions. 
More modern definitions and applications of structural models can be found in Reiss and Wolak (2007) and Reiss (2011). 
%TODO: Add references.

Generically, structural empirical models consist of equations or inequalities that describe the optimizing behavior of individuals or organizations.
These equations come from decision-making models that describe what factors (variables) impact agents' decisions.
Let $y^*$ denote the endogenous choices of agents. 
The researcher may observe these choices, or some transformation of them.
Let $y=y(y^*)$ denote the choices the researcher observes. 
The exogenous factors affecting decisions observed by the researcher are denoted by the vector $x$. 
Unobserved factors generally are divided into two types: those that vary across sample observations, $\xi$, and those that are constant across observations, $\theta$. 
The $\theta$s are referred to as parameters.

Mathematically, the theoretical model delivers either equalities (``structural equations")
$$  y = g(y, x,\xi ,\theta)$$ 
or inequalities, e.g.,
$$  y \le g(y, x,\xi ,\theta)$$
that relate these quantities. 
In most structural models, the functional form of $g(\cdot)$ is known and is a consequence of specific functional forms used in the theoretical model. 
Usually the ultimate goal of the structural model is to show how the unknown parameters $\theta$ can be recovered from data on $y$ and $x$. 

To illustrate these ideas, under specific economic and behavioral
assumptions, the behavior of demanders and suppliers can be described by a demand and supply system of the form:
$$\begin{array}{lcl}
q^D & = & \theta_1 + \theta_2 \, p + \theta_3 \, x_1 \\[.5em]
q^S & = & \theta_4 + \theta_5\, p + \theta_6 \, x_2 \\[.5em]
q^D  & = & q^S 
\end{array}
$$
Here $y^*$ equals: quantity demanded, $q^D$;  quantity supplied, 
$q^S$; and price $p$.
The exogenous variables in $x$ include demand and cost shifters such as consumer incomes ($x_1$) and suppliers' input prices ($x_2$).
It is important to notice that this structural model consists of three equations. The first two equations are a consequence of the optimizing behavior by individual demanders and suppliers. 
The third condition is an equilibrium condition (i.e., ``demand equals supply"), one that is imposed by the modeler to close the model. 
That is, map the unobserved quantities in $y^*$ to an observed $y$, which is simply ``quantity" ($q$) and price.

In principle these theoretical relations can be taken directly to data.
The problem with taking them directly to data is that the data will prove them flawed. 
For example, in the above demand and supply system the
only object besides $x$ that explain variation in $y$ is the vector $\theta$. 
With many observations on prices and quantities, it would be highly unusual for the six unknown parameters to explain the variation in price and quantity not rationalized by $x$.
For this reason, the researcher typically adds unobservables, such as $\xi$ and $\epsilon$, whose variation across sample observations (along with the other variables) is capable of rationalizing 
all values of $y$.
It is important to realize that these unobservables need not be motivated by the model or related to agent behavior.\footnote{Measurement errors are but one example.}

With the addition of an $\epsilon$, the above demand and supply model becomes 
$$\begin{array}{lcl}
q & = & \theta_1 + \theta_2 \, p + \theta_3 \, x_1 + \epsilon_1 \\[.5em]
q & = & \theta_4 + \theta_5\, p + \theta_6 \, x_2 + \epsilon_2\\
\end{array}
$$
The critical challenge for the researcher at this point is to explain how the observational data on $x$ and $y$ identify $\theta$. 
It is important to realize that just because the structural model contains distinct parameters or quantities, this does not mean that they can be unambiguously estimated. 
When they can, the quantities are ``identified." 
For example, $y$ might be CEO compensation and one
element of $\theta$ might be a coefficient representing a corporate CEO's degree of relative risk aversion.
In order to recover the degree of relative risk aversion from data, the researcher must be prepared to show how risk aversion is recovered from variations in CEO compensation and the other variables.

\subsection{Are Structural Models Relevant to Accounting Research? A Misstatement Illustration}

This high-level discussion of structural modeling may not seem directly relevant to accounting
research. 
Indeed, in general there is a divide between theoretical and empirical research in accounting.
While there are exceptions, few theoretical accounting researchers explain how to map the specifics of their models to data. Similarly, few empirical researchers have attempted to take theoretical models directly to data.
Does this mean that structural models are not applicable to accounting questions? Or, is there unrealized potential? Our view is that structural models have more of a role to play.
To be clear, we do not believe that all accounting researchers need  estimate structural models. 
Indeed, no structural modeling exercise should go forward unless the researcher is highly convinced that the benefits of a structural model would outweigh the substantial costs entailed in developing and estimating a structural model. 

While we have already discussed some of the advantages of structural models, it is 
important to keep in mind the challenges researchers face in developing structural models. 
To begin, structural models can be both technically demanding and time-intensive to develop. 
Additionally, when constructing a theoretical model that can be taken the data, the
empirical researcher will typically be forced to make simplifications that a pure theorist would never make.
Similarly, in order to have an empirical model that is linked to a theory, the empiricist may have to live with assumptions that other empiricists criticize as
unrealistic. 
Finally, just because a researcher can write down a theoretical model and estimate it does not make the empirical model ``right."  There is no guarantee that the causal connections contemplated are correct. 
Further, there is no guarantee that after all the effort that went into developing the model, that the estimates will make sense or that the model will otherwise be validated. 
Despite these challenges, we believe that the benefits of structural models can outweigh their costs.
The purpose of this section is to illustrate how a structural model can deliver accounting insights. 

To illustrate how one might go about developing a structural model for accounting data, we turn to the topic of accounting misstatements. 
There is now an extensive accounting literature exploring why misstatements are made and how easy they are to detect.
This literature is of importance to investors, managers and boards.
Many predictive or explanatory model of misstatements regress indicators for misstatements on a host of accounting, firm and market variables. 
These variables include measures of the complexity of the accounting statements, accrual quality, off-balance sheet activities, firm performance, firm and auditor characteristics, and manager compensation variables. 

If we take this literature as starting point, empirically we would like to have a model that explains why misstatements occur. 
The first problem a structural model must confront is the empirical reality that we do not observe misstatements, but typically only restatements. 
Restatements occur after the firm and its auditor agree on how to report results, and as such are triggered by investigations by third parties or new 
audits.
For simplicity we shall refer to these collectively as ``investigations." 
Thus, any inferences about misstatements have to be seen through the lens of what is detected, or restatements.
If investigations are perfect and detect all misstatements that initially get past the auditor, then there is a one-to-one correspondence of misstatements and restatements. 
Realistically, investigations are not perfect, meaning that we will have to recognize the difference between the two in the structural model.

\subsection{A First Model: Nonstrategic Auditing}

To simplify the initial model, we imagine that misstatements  are
deliberate. 
Though this is clearly a strong assumption, we make it here because it 
allows us to deliver clear predictions about the unobserved rate of misstatements.\footnote{
While this strong assumption can be seen as a weakness of the model, it also can be
seen as a strength. 
If we, or others, do not like the end model, we know what assumption(s)
to revisit.} 
We also simplify matters by assuming that a single agent, the `CEO', is responsible for deciding whether or not to misstate results.
The CEO is assumed to be rational in the sense that they trade off the expected benefits and costs of misstatements when deciding whether to misstate.

Suppose that the CEO receives a benefit of $B^*$ from the \emph{successful} manipulation of earnings, 
i.e., from a misstatement that is not detected by the firm's auditors or subsequent investigations. 
Manipulations are successful only if they are not caught initially by the firm's auditors before a report is released and if they are not caught subsequently during further scrutiny.
We assume the firm's auditors independently catch misstatements at a constant rate $p_A$ and that the (conditional) probability of subsequent investigations catching a misstatement is $p_I$.
Given these assumptions, the probability of misstatement getting past the firm's 
auditor and subsequent investigation is $(1-p_A) \times (1 - p_I)$.
The CEO's benefit from a successful misstatement is then
$$ B^* = (1-p_I) \times (1-p_A) \times B$$
where $B$ is a gross benefit to the manager of a misstatement. 

To misstate performance, the CEO must exert costly effort, which is a fixed $C_M$. 
Putting together the manager's benefits of misstating with their costs gives
\begin{equation}\label{bencost}
y_M^* = \begin{cases}\mbox{Misstate} & \mbox{if }\, (1-p_I) \times (1-p_A) \times B - C_M \ge 0\\
\mbox{Don't Misstate} & \mbox{Otherwise}.\end{cases}.\end{equation}
This (structural) inequality describes the unobserved misstatement process. 
In general, researchers will not observe $B$ or $C_M$. 
In some instances, they may not observe $p_A$ or $p_I$.

To complete the structural model, the researcher must relate these objects to the observed data.
Because we observe a zero-one indicator variable for restatements $y$ and not misstatements, we need to link the two. 
In this model, restatements are the result of three stochastic processes:

\begin{enumerate}
\item The manager decides to misstate (or not).
\item The firm auditor randomly audits and detects (or not) .
\item A post-report investigation happens and detects (or not).
\end{enumerate}

Mathematically, this sequence can be modeled as
\begin{equation}\label{restateeqn}
 y = I(\mbox{Restate}) = I(y^*_M \ge 0) \, \times\, (1 - I(y^*_A \ge 0)) \, \times\, I(y^*_I \ge 0)
\end{equation}
where $I(\cdot)$ is a zero-one indicator function equaling one when the condition in parentheses is true.
The unobserved variables $y^*_A$ and $y^*_I$ respectively reflect the likelihood that the firm's
auditor and the investigation process detect a misstatement. Notice equation (\ref{restateeqn})
uses $(1 - I(y^*_A \ge 0))$, the indicator that the firm's auditor misses the misstatement.

Equation (\ref{restateeqn}) somewhat resembles a traditional binary discrete choice model. The easiest
way to see this is to take expectations (from the researcher's standpoint)
\begin{equation} \label{equilpr}
\begin{array}{lcl}
 E(y) & = & E\, \left[\; I(y^*_M \ge 0) \, \times\, (1 - I(y^*_A \ge 0)) \, \times\, I(y^*_I \ge 0) \; \right]\\[1em]
 & = &  \mbox{Pr(Misstate)} \times \mbox{Pr(Auditor Misses)} \times
\mbox{Pr(Investigation Finds)}\\[1em]
& = & \beta^* \times (1-p_A) \times p_{I} = \mbox{Pr(Restate)}
\end{array}\end{equation}
From equation (\ref{bencost}), $\beta^*$ is the (researcher's) forecasted probability that a misstatement occurs, or
\begin{equation}\label{betaplus}
\beta^*= \mbox{Pr}\left(\, (1 - p_A)(1 - p_I) B - C_M \ge 0 \,\right)
\end{equation}

At this point, the theory has delivered a structure for relating the \emph{unobserved} probability of a misstatement, $\beta^*$, to the potentially estimable probability of a restatement.
Now we face a familiar structural modeling problem, which is that the equations delivered by theory do not necessarily anticipate all the reasons why
in practice why CEO behavior, auditor behavior or outside scrutiny might vary across accounting reports.
For example, the theory so far does not point to different reasons why CEO's might differ in their cost-benefit analyses of misstatements. 
To move theoretical relations closer to the data, researchers typically introduce observable reasons into them. Often there is some ad hoc or ``nonstructural" element to these
additions. 
Empiricists are willing to do this, however, because they believe that it is important to account for heterogeneity they believe the theory does not explicitly recognize.

To illustrate this approach, here we assume that CEO's unobserved costs and benefits do vary systematically with observables.
In addition, because these observables do not perfectly represent the observed and unobserved benefits, it is important to allow for unobservable differences in the costs and benefits of misstatements. 
One specification that does this is to assume
\begin{equation}\begin{array}{lcl}\label{eqns1}
B & = & b_0 + b_1 \, \mbox{BONUS} + X_B\beta\\[.5em]
C_M & = & m_0 + m_1 \, \mbox{SALARY} + X_C\gamma + \xi\\[.5em]
\end{array}
\end{equation}
where BONUS is the fraction of a CEO's total pay that is stock-based compensation, 
the $X_B$ are other observable factors that impact the manager's benefits from misstatements,
SALARY is the CEO's annual base salary, and the $X_C$ are observable factors impacting the CEO's perceived costs of misstatements.

Why this linear specification and why these variables? 
We have no strong theoretical reason for the linear assumption. 
Instead, its motivation is practical.
We shall shortly see that it facilitates estimation of the model unknowns.
As for the variables in the benefit and cost specifications, here we rely in part on theoretical and empirical observations in the academic literature on misstatements. 
The BONUS variable is included in benefits because it is thought to capture a classic moral hazard problem: the more CEO are rewarded for performance, the
greater their incentive to misstate results so as to increase (perceived) 
performance.
%TODO: Get refs
Thus, we would expect the unknown coefficient $b_1$ to be positive.
Similarly, the SALARY variable is included in costs because previous studies have hypothesized that the higher a CEO's guaranteed base pay, the more the CEO perceives ex ante that it is risky to make misstatements.
Thus, we would expect the unknown coefficient $m_1$ also to be positive. For now, we leave the other $X$ variables unnamed.

One key variable in the above model is the unobserved cost $\xi$.
While it makes sense to say that the researcher cannot measure all misstatement
costs, why not also allow for unobserved benefits as well.
The answer here is that adding an unobserved benefit would not really add to the model as it is the net difference that the model is trying to capture.\footnote{
The sense in which it could matter is if we thought we observed the probabilities
$p_A$- and $p_I$.
In this case, we might be able to distinguish between the cost and benefit unobservables based on their variances.}
To see this, observe that the additive error in costs becomes the additive error in the net benefit to a misstatement. 
Further, the probability of a restatement becomes
\begin{equation}\label{restate1}
\mbox{Pr(Restate)} = \theta_0 \mbox{Pr}\left(\, \theta_1 + \theta_2 \mbox{BONUS}
+ \theta_3 \mbox{SALARY}  \ge \xi \,\right)
\end{equation}
where $\theta_0=(1-p_A) \times p_{E}, \theta_1 = (1 - p_A)(1 - p_I) b_0 - m_0, 
\theta_2 = (1 - p_A)(1 - p_I) b_1,$ and $\theta_3 = - m_1.$ 

Apart from the scalar multiple $\theta_0$, which can be absorbed into the probability statement (and is thus not identified), this probability model has the form of  a familiar binary choice (e.g., a probit or logit model).
Thus, the value of the structure imposed so far is that it can motivate the application of a familiar statistical model, as  well as explain how the estimated coefficients are potentially connected to quantities that impact the probability of a misstatement.

\subsection{Estimation}

To illustrate the application of this structural to data, we assembled a dataset containing 5,000 firm-year observations on whether or not financial results were restated in a given year.
Table \ref{tab:desc} describes the variables in our data set.
We discuss the source of the data in more detail after we estimate the structural model.

The data include variables that have previously been used to model restatements.%TODO: \footnote{[Refs to be included]} 
The variable BIG4 is included  because it is believed that Big 4 auditing firms have more expertise and are therefore more likely to catch misstatements. 
Similarly, the corporate governance literature suggests that board oversight from directors with accounting or finance backgrounds reduces the likelihood that CEOs will make misstatements. 
%TODO: Get refs.
Finally, the variables INT and 
SEG are included to capture the complexity and costs of audits. 
%TODO: Get reference for audit model.
Specifically, international companies and companies with more business segments are thought to raise the costs of auditing. 

Table \ref{tab:desc} reports descriptive statistics for the sample. CEOs are on average receive about three-quarters of a million dollars in base pay and their incentive-related pay averages 26\% of their total pay.
Three-quarters of the sample has a Big 4 accounting firm as its auditor. The fraction of directors with financial expertise is less than ten percent. 
The average firm has about four and one-half business SEG and is primarily based in the United States.

Absent a theoretical model of mis- or re-statements, most empirical
analyses of these data would summarize them by regressing the restatement
indicator on the list of predictors in Table 1. Such a regression can either be described
as showing how the probability of a restatement co-varies with the right hand side variables,
or it can be described as a prediction equation. Our structural model allows us to say more,
particularly when it comes to the signs of the SALARY and BONUS variables. By including
the other variables on the right hand side, we are in essence maintaining they belong in
$X_C$, $X_B$, or both sets of regressors.

Table \ref{tab:logit} reports the results of logit regressions in which the dependent
variable is the restatement indicator variable. 
The table contains both a simple 
specification containing an intercept along with the two CEO pay variables, 
and a more intricate specification involving the other variables we have in our data.
For each specification we report the estimated coefficients of the logit and the 
corresponding marginal effects evaluated at the sample averages of the exogenous
variables.%TODO: Explain marginal effects.
The results for the pay coefficients in both specifications run counter those the previous accounting literature might predict and counter to those predicted by the structural model.
Specifically,  more base pay is associated with more restatements, while more incentive pay is associated with fewer restatements.
Besides the intercepts and BONUS coefficients, the only other coefficients  that are statistically significant are INTional companies and companies with more SEG to audit. 
While we can say (descriptively) that they are associated
with higher restatement rates, unless we take a position on how they enter $X_C$ or $X_B$, it is difficult to interpret whether their signs make sense.

The question we now address is what to make of the fact that the coefficients
on the CEO pay variables did not turn out as either informal arguments or our
structural model would predict. 
We consulted colleagues in the profession for their opinions. 
%TODO: Really?
One set attributed the outcome to data errors or sampling issues. 
Another set said the logit model was obviously flawed because it did not include all relevant accounting variables. (High on the list were measures
of recent accrual activity, use of off balance sheet transactions and firm
performance variables.) 
Their thinking was, had we included these, the signs on the pay coefficients might be different. 
Mentioned less often was that the idea that an important difference between restatements and misstatements is the role of subsequent investigations. 
There is nothing in the model or variable list that would help predict  the intensiveness of outside scrutiny (or indeed of scrutiny by the firm's external auditors).

\subsection{An alternative model}

The point we would like to make here is that the structural model can help us understand the potential influence and importance of each of these points. 
To illustrate this, we will focus attention on the model's potentially overly simplistic view of the auditor's role in detecting misstatements.  
To make the model richer, suppose the firm's  auditors are more likely to look for misstatements when they perceive they are more likely to find misstatements. 
This reasoning naturally
leads to a strategic decision making model where the managers' and the auditors' decisions are interdependent.
%TODO: Get references (e.g. Tshibano (198X) and Bresnahan and Reiss (1991)).
In such a model, auditors too presumably trade off the costs of audit effort against the reputational losses they might incur should they miss a managerial misstatement and the misstatement is subsequently detected.\footnote{
Here we have in mind the findings of \citet{Dyck:2010kh} who show that many egregious forms of misstatements are detected subsequently by employees, directors, regulators, and the media.} 

In the previous model, the firm's auditor impacted the manager's misstatement benefits through $p_A$. 
Suppose that $p_A$ is in fact a choice variable for the firm's auditor. 
To make matters simple, suppose that the auditor detects manipulation with probability $p_{AH}$ if they exert high effort and  otherwise they detect manipulation with the lower probability $p_{AL}$. 
Let the cost of high effort be a fixed cost $C_A > 0$. 
Without loss of generality suppose the cost of low effort is zero. 
When deciding whether to audit with high or low effort, the auditor perceives a cost to its reputation, $C_R$, to not detecting a misstatement that is subsequently caught by an external investigations. 
This structure implies that the total cost of high effort to the auditor is $C_A + (1-p_{AH}) \times p_E \times C_R$ -- the cost of high effort plus the expected cost of missing a misstatement that is subsequently caught with probability $p_E$. 
The total expected cost of
low effort is similarly, $(1-p_{AL}) \times p_E \times C_R$. 

We complete this new model, we need to make an (equilibrium) assumption about how the CEO and firm auditor interact. Following the literature, we assume that the two simultaneously
and independently make decisions, and that their strategies form a Nash equilibrium.
That is, we assume the players' strategies are such that they optimize their objectives 
taking the actions of the other players as fixed. This means that in a Nash equilibrium, 
the players are taking actions that they cannot unilaterally improve upon.

It is well known that in this type of auditing game, that the CEO and the auditor 
best actions are to play a mixed (randomized) strategy.\footnote{Explain.}
That is, the auditor will independently exert high effort with probability $\alpha^*$ 
and the manager independently misstates with probability $\beta^*$. These probabilities 
are such that each has no incentive to change their randomized strategy; that is:\\
\begin{quote}
(i) the manager is indifferent between misstating and not misstating, or:\\
\begin{equation}\label{manager}
(1 - p_A^*)(1 - p_I) B - C_M = 0 
\end{equation}
where $p_A^* = \alpha^* p_{AH }+ (1-\alpha^*) p_{AL}$ is the equilibrium 
probability a misstatement is detected; and,\\
\end{quote}
\begin{quote} (ii) the auditor must be
indifferent between exerting high and low effort, or\\
$$ \beta^* (1-p_{AH}) p_G C_R + C_A = \beta^* (1-p_{AL}) p_I C_R .$$
\end{quote}

\vglue 5pt
Solving these two equations for the equilibrium probabilities $\alpha^*$ and $\beta^*$
yields:\\
\begin{equation}\label{equilstrat}
\begin{array}{lcl}
  \alpha^* &= & \dfrac{ ( 1 - p_{AL}) (1 - p_I) B- C_M}{ (1 - p_I) (p_{AH}-p_{AL}) B}\\[1.5em]
  \beta^* &= & \dfrac{C_A}{(p_{AH}-p_{AL}) p_I C_R}  
\end{array}
\end{equation}
From these equations, we can calculate the equilibrium probability of a restatement\footnote{
As part of the solution, we require $\alpha^*$ and $\beta^*$ to be probabilities between
zero and one. This is true provided $C_R$ and $B$ satisfy the inequality\\
$$  C_R > \frac{C_A}{(p_H-p_L)p_I} $$
and \\
$$ B > \frac{C_M}{1 - p_I}  $$}
\begin{equation} \label{equilpr1}
\begin{array}{lcl}
\mbox{Pr(Restate)} & = &  \mbox{Pr(Misstate)} \times \mbox{Pr(Auditor Misses)} \times
\mbox{Pr(Investigation Finds)}\\[1em]
& = & \beta^* \times (1-p_A^*) \times p_{E}
\end{array}\end{equation}
This equation tells us how the observed (or measurable) probability of a restatement is related to
the unobserved frequency of misstatements. In particular, if we knew the frequency with which auditors and investigations caught misstatements, we could easily link the two. Otherwise,
we would have to estimate these probabilities (or make assumptions about them).

Substituting the equilibrium strategies (\ref{equilstrat}) into (\ref{equilpr1}) yields
\begin{equation} \label{equilpr2}
\begin{array}{lcl}
\mbox{Pr(Restate)}& = &  \dfrac{C_AC_M(1-p_{IL})}{(p_{IH}-p_{IL})(1-p_I)C_RB}.
\end{array}\end{equation}
We now are in a position to use the theory to help interpret the conflicting logistic regression results
in Table 3. 

 
Equation (\ref{equilpr2}) shows that the presence of a strategic external auditor
changes how the CEO's incentives impact the probability of a restatement.\footnote{Notice that
the probability statement in equation  (\ref{equilpr2})  differs from that in equation (\ref{restate1}).
The probability statement in equation  (\ref{equilpr2}) reflects the randomness of the
strategies, whereas in equation (\ref{restate1}) it reflects unobservables the researcher 
does not observe.} Partial derivatives of equation (\ref{equilpr2}) show that the restatement probability is:\\

\begin{itemize}
\item Decreasing in the benefit $B$ that the manager enjoys from misstatement.
\item Increasing in the personal cost of manipulation $C_M$ incurred by the manager.
\item Decreasing in the reputational cost $C_R$ incurred by the external auditor.
\item Increasing in the cost of high effort $C_A$ incurred by the external auditor.
\end{itemize}

Thus, in contrast to the nonstrategic model, increasing the benefit that managers enjoy from misstatement,
or decreasing the misstatement cost, leads to fewer restatements being observed by researchers.
These two effects explain the negative sign on BONUS and the positive sign on SALARY observed in the previous logit results. Thus, this structural model has the potential to rationalize patterns observed in the data.

To have a better sense of how one might connect the strategic auditor theory to the
logistic models in Table \ref{tab:logit}, suppose, similar to ways we motivated (\ref{eqns1}), that 
\begin{equation}\begin{array}{lcl}\label{eqns2}
B & = & b_0 + b_1 \, \mbox{BONUS} \\[.5em]
C_M & = & m_0 + m_1 \, \mbox{SALARY} \\[.5em]
C_A & = & a_0 + a_1 \, \mbox{INT} + a_2 \, \mbox{SEG}\\[.5em]
C_R & = & r_0, \quad B  =  b_0, \quad p_{AH}   =  p_0, \; \mbox{ and } \; p_{AL}  =  v_0 \\[.5em]
\end{array}
\end{equation}
where $ a_0, a_1, a_2, r_0, b_0, p_0$ and $v_0$ are constant parameters. 
Inserting these expressions into the expected restatement rate (\ref{equilpr2}) gives
\begin{equation*} \label{equilpr3}
\begin{array}{lcl}
\mbox{Pr(Restate)}& = &  \dfrac{C_AC_M(1-p_{IL})}{(p_{IH}-p_{IL})(1-p_I)C_RB}\\[2em]
& = & \dfrac{(1-v_0)(a_0 + a_1 \, \mbox{INT} + a_2 \, \mbox{SEG})(m_0 + m_1 \, \mbox{SALARY})}
{(p_0-v_0)r_0(b_0 + b_1 \, \mbox{BONUS})}\\[2em]
\end{array}
\end{equation*}
\begin{equation}\label{equilpr4}
 =  \dfrac{\theta_0 + \theta_1\mbox{\small INT} + \theta_2\mbox{\small SEG} + \theta_3\mbox{\small SALARY}
+ \theta_4\mbox{\small INT} \times \mbox{\small SALARY}+ \theta_5\mbox{\small SEG} \times \mbox{\small SALARY}}
{1 +  \theta_6\mbox{\small BONUS}}
\end{equation}
Notice that the $\theta$s absorb unknown quantities such as $r_0$ and $p_0$, and that the denominator intercept
is normalized to one. This last restriction is required to identify the ratio of the two linear functions.

Although this model does not have a logit form, it is potentially estimable using 
generalized method of moments (GMM).
This method attempts to match so-called sample moments to what the structural model implies they should be. 
For example, an obvious sample moment would be the average restatement rate in the sample.
The corresponding theoretical moment would be the probabilty expression in equation (\ref{equilpr4}).
Because we need at least as many moments as we have $\theta$ parameters to estimate (there are seven $\theta$s in the model), we use seven sample moments, each of the form:
$$ \mathcal{M}_j = \sum_{i=1}^{5,000} \; X_{ji}^\prime\left[\; \mbox{RESTATE}_A - \mbox{Pr(Restate)}_A \; \right]. $$
where Pr(Restate) comes from equation (\ref{equilpr4}).\footnote{
To ensure that the model parameters imply restatement probabilities between zero and one, we add a penalty function to the GMM objective function.
This penalty increases with the number of estimated probabilities below zero or above one.
For most replications this penalty is immaterial to the results obtained.} 
The $X_j$ used in the moments include all explanatory variables. 
Thus, because $X$ includes  a dummy variable for whether the firm is an international company, the corresponding moment equation seeks to match the sample international companies average restatement rate to the model's prediction for that rate.

Table \ref{tab:gmm} reports the results of estimating the new (strategic auditor) structural model on the sample of 5,000 firms. 
The results show that in this particular case, even without sample information on the unobserved probabilities $p_A$ and $p_I$, we can recover estimates of the model parameters up to a normalization.\footnote{
A simple way to see this might be the case is to observe that there are seven $\theta$ coefficients and eight underlying structural parameters.}
For instance, the coefficient ratio $\theta_3/\theta_0$ estimates the ratio of cost parameters $m_1/m_0$.
Since $m_1$ is the cost coefficient on SALARY and $m_0>0$ for costs to make sense, the sign of $\theta_3/\theta_0$ reveals the sign of $m_1$.
From the theory, we expect the sign to be positive, and in the estimates it is. 
Similarly, $\theta_6$ equals the (scaled) misstatement benefit coefficient on the BONUS variable.
Although the descriptive regression coefficients in Table 3 suggest BONUS has a negative effect on restatements, here, because we model misstatements as part of restatements, we find it has a positive effect, as expected.

The one sign that does not make sense given the other coefficient estimates is the negative sign on $\theta_6$, however this coefficient is insignificantly different from zero.
This is indeed true of most of the coefficients, and is perhaps not surprising given the participants use of randomized strategies.
Further, even with a sample size of 5,000, restatements are relatively rare, thus making it difficult for the model to predict them with much accuracy. 

While the coefficient magnitudes do not allow us to estimate the underlying benefits and costs to managers from misstatements, we can illustrate the value of the model by performing a counterfactual calculation.
There are many different counterfactuals that could be considered. 
Here, for illustrative purposes we can ask what would happen to misstatements and restatements if we  do away with incentive pay and nothing else changes.
The value of having a model to analyze this
change is that we can see how the auditing process would adjust to the removal of 
CEO incentives to misstate. 
From the equilibrium strategies in equation (\ref{equilstrat}), we see that removing bonus pay does not
change the equilibrium frequency of misstatements, but does change the frequency of high effort auditing.
From (\ref{equilpr}), the model and the data we find
$$ \dfrac{\mbox{Pr(Restate }\vert \mbox{ No Bonus)}}{\mbox{Pr(Restate }\vert \mbox{ Bonus)}}=\dfrac{\beta^* \times (1-p_A^{**}) \times p_{E}}
{\beta^* \times (1-p_A^{*}) \times p_{E}} = \dfrac{(1-p_A^*)}{(1-p_A^*)} = 1.10.$$
What this says is that the restatement rate increases by 10\% (from 10.24\% to 11.25\%) when the bonuses
are withdrawn. 
The fact that the restatement rate goes up may at first seem somewhat odd given that the benefits to the CEOs have fallen. 
The model, however, shows that the increase  comes about because the auditors exert \emph{less} effort in detecting misstatements, thereby catching fewer, leaving more for outsiders to subsequently catch.  

The discussion above illustrates some of the ways in which a structural modeling exercise might help understand accounting data. 
In particular, the comparative statics of the model shed light on the difference between restatements and misstatements, and what assumptions (e.g., strategic versus nonstrategic auditor) and data were needed to draw  inferences about misstatements from restatements. 
Additionally, we were able to recover some of the primitive parameters impacting incentives for managers to misstate results, as well as perform counterfactual analyses.

While there is the potential for disappointment in the simplistic theory we used, we see room for improving models as an opportunity rather than a defining limitation.

\subsection{Limitations of structural models}
% This discussion may belong near the end. When we talk about structural modeling, we should make clear that some questions are difficult to fit into a structural model, etc.
The vast majority of empirical research papers in accounting do not rely on a formal theoretical model to motivate their hypotheses.
But in many cases, extant theory would not support estimation of a structural model.
For example, \citet{Huang:2014cs} study the effect ``tone management'' on capital market outcomes.
Developing a formal theory of the relation between firm performance, managerial psychological states, and measures of tone would be a complex undertaking involving economics, psychology, and linguistics.
Building on such a (hypothetical) foundation to solve the complex game involving managers and capital markets would be extremely ambitious.
Instead, \citet{Huang:2014cs} does what almost all empirical research papers in accounting do and resorts to more verbal approaches to hypothesis development. 
% Can we find some critique of this approach? Pfleiderer?  YES -- let's review Paul's paper and add some of this and include him in the references

To those who would see the ``glass as half full," the exercise can be seen as yielding insight into an important accounting issue and how accountants might better take advantage of data (and indeed what extra data they might like to collect).

\subsection{Structural models in accounting research}

The use of structural models in accounting research has been very modest to date.  There are certainly examples of regression models being derived (or perhaps influenced) by a theoretical model. For example, Lambert and Larcker (1985) use the traditional Holmstrom model (and a variety of simplifying assumptions) to help specify a regression function linking CEO compensation to firm performance. While somewhat structural in orientation, the approach suppresses the fundamental causal mechanism associated with compensation decisions and does little to actually estimate the primary structural model of Holmstrom.

% This stuff is in Peter's structural section (i.e., before Section 6).
Gerakos and Kovrijnykh (2013) and Nikolaev (2014a,b) provide an analysis misreporting and accounting quality that resembles some features of structural models (e.g., Nikolaev implements method of moment estimation). Both papers develop a dynamic stochastic model for the accounting process and use this structure to separately identify quality earnings from manipulated earnings. While interesting empirical studies, these papers simply assume the fundamental stochastic process for earnings and abstract away from all the underlying (optimal) managerial decisions and accounting choices that ultimately produce observed accounting numbers. 

The recent papers by Zakolyukina (2014) and Bertomeu et al (2015) are more consistent with traditional structural modeling.  Similar to our simple model above, Zakolyukina is concerned with how equity incentives motivate managers to manipulate earnings.  Bertomeu et al examine management forecasts using a formal disclosure model to estimate whether managers strategically withhold information from shareholders.

These two papers model an institutionally rich problem, estimate the derived model, provide estimates for important structural parameters, and also give interesting counterfactuals based on their theoretical models.  
We view these papers as useful initial steps in applying structural approaches to accounting research questions.\footnote{Maybe review some structural finance papers -- Luke Taylor on CEO labor markets, Whited on debt?, Terry jmp about "meet or beat" and macroeconomic research and development.} % Is this footnote for the editors or for us?



% Peter's stuff here. May be worth bullet-pointing this for now.
% I think it would be useful to do a little discussion of efforts to date.
% We could be fairly kind, but highlight the challenges.

\subsection{Deeper institutional understanding}

We believe that accounting research could benefit greatly from increased emphasis on research that enhances our understanding of real-world phenomena and institutions. There are several benefits that would accrue to such efforts.

\begin{enumerate}
\item Better hypothesis development.  
\item % Too many papers in accounting research test ``armchair" hypotheses with no basis in real-world phenomena.
\item Enhanced identification of causal effects.
\item More relevant research results.
\end{enumerate}

One alternative approach to empirical research in economics is labeled the structural approach (Wolpin 2013).

The structural approach ``requires that a researcher explicitly specify a model of economic behavior, that is, a theory." (Wolpin, 2013, p.2).


\subsubsection{Better hypothesis development}
Many papers examine hypotheses that are not motivated by prior theory,  observations of real-world phenomena, or beliefs of practitioners. Instead, researchers often propose and empirically test hypotheses in the same paper.

\subsubsection{Enhanced identification of causal effects}
One point that is often overlooked by researchers seeking to make causal inferences is that a deep understanding of the treatment assignment mechanism is necessary to support claims that such assignment is as-if random. Such understanding is not statistical, but relates to the facts of assignment itself. \textbf{Get some examples from Rubin on this.}

\subsection{Increased emphasis on measurement}
It is often claimed that accounting researchers have a ``comparative advantage in measurement.'' This claim is presumably based on the notion that accounting in practice relates to measurement of the performance and financial position of organizations. However, other disciplines have created extensive literatures studying measurement issues. For example, psychologists have long grappled with issues of measurement of various constructs such as intelligence. This work has given rise to deep statistical techniques.

\section{How to enhance institutional understanding}



Section -- Descriptive Institutional Research 


Accounting is essentially an applied discipline and it would seem that most empirical research studies should be well grounded in institutional facts.  Unfortunately, there is a remarkable lack of in-depth institutional descriptive research in the top accounting journals.  Although it is a matter of taste, a number of the papers in our review seem to ask research questions and use causal mechanisms that are far removed from real world issues.  We believe that accounting research would substantially improve if more in-depth descriptive studies were published.  As we discuss below, this type of research is an essential component of developing structural models and improving our understanding of causal mechanisms.

In the past (find the year), the Journal of Accounting Research would publish some papers in the section entitled Capsules and Comments (I think this is the correct name).  The editor (in this case Nicholas Dopuch) would seem to place selected papers into this section if there was some new institutional data and ideas, but where the perceived quality of these papers was not sufficient to warrant publication as a main article.  This seems like a good idea to revive.


Here are some other “classic” observational studies:
Cyert, Simon, and Trow, “Observation of a Business Decision,” Journal of Business, July, 1956)
Bower, Joseph, Managing the Resource Allocation Process:  A Study of Corporate Planning and Investment, Harvard Business School, 1970
Mintzberg, The Natural of Managerial Work

More recent possibilities:
This might be useful (I like this because it has a direct link to managerial accounting) -- Bloom and Van Reenen, “Measuring and Explaining Management Practices Across Firms and Countries,” QJE (Nov, 2007).  A descriptive study of 732 medium-sized firms and attempting to assess whether management practices are related to productivity.  Practices are operations (e.g., lean mfgr), targets (simple vs complex, stretch), and incentives (perf based).  Use surveys and interviews.  
Ahern (2014) comprehensively examined 183 illegal insider networks using primary source documents from the SEC, DOJ, and various public records.  Almost purely descriptive, but provides rich insights for developing theoretical models of networks and clever methodological designs for empirical studies.  For example, Network relationships are familial (23%), business-related (35%), friendships (35%), or “not clear” (21%).  Insiders are more likely to be an accountant or lawyer, less likely to be a Democrat, and more likely to have a “criminal record.”

Accounting examples:
Soltes (2014) examines the interactions between sell-side analysts and company management in one firm that granted proprietary access.  Fairly unstructured analysis and he is looking for sort of obvious patterns.  Lots of conjectures in prior papers about the context, ability, and information transfer during these meetings with analysts.  Attempts to answer who, when, and why regarding these interactions.  Who – actually similar to Mayhew (2008) results from conference calls – cover fewer firms, less time as an analyst, etc.).   When – interactions through the year, after earnings release. Why – obviously to gain additional insight (within the confines of Reg FD), do not seem to update forecasts immediately after meeting.  Other – analysts meet with companies they do not cover – why?  This type of insight has to be relevant to understanding the mechanisms linking analyst behavior, analyst forecast, etc.

Use Groysberg, Healy, and Maber – compensation of analysts?  I guess this is new insights, but not from a descriptive study.  Just new proprietary data?

Brown et al (2014, JAR) survey – do we think this is useful (deep) descriptive research?  They ask some interesting and useful questions – how analysts view earnings quality (earnings supported by cash flows), they do not believe the “red flags” used by academics, they are generally not attempting to uncover manipulation, forecasts are used to figure out the stock price target and not an end in themselves, etc.  Seems like these are good insights that can shape structural models.

Clearly, there are other similar surveys – Dichev et al, JAE (2013).  These surveys seem pretty odd to me.
I guess my preference would be for there to be more structured or semi-structured interviews as opposed to impersonal surveys.  Looking for commonalities across many different interviews.


What features would make for a good descriptive study:
Select a topic or setting of real interest to accounting researchers – analysts, loan officers, executives making strategic decisions, compensation committee members, etc. How do managers and boards make decisions on “disclosure quality” (an aside – do they even know what disclosure quality means and do they care?  Does this construct have any real relevance in the real world?)  
Related to your point that we work on silly ideas:  Would conditional conservatism show up in conversations with real world managers?  If you had an unstructured conversation, what accounting topics would be “top of mind” for managers, board members, bankers, etc.?
Where present accounting research seems to depart from known institutional details – do executive really behave like Black-Scholes would imply?  How are compensation plans really designed and why?  Why don’t companies take advantage of “academically obvious” tax changes?
Can we develop a list of silly examples from accounting papers that just do not match any that is plausible in the real world?
You need to go out and actually interview these people using structured and semi-structured interviews.  Not just one per company, but several at varying levels.  Need to understand the setting, economics aspects, behavioral aspects.
Map out the mechanism by which decisions of interest are actually made.  Maybe there are multiple mechanisms depending on the situation.  What are the contextual variables?
Use these mechanisms to develop structural models – if you want to explain something observed, maybe it is a good idea to understand the phenomenon of interest first.











\subsection{Greater emphasis on description}
A typical paper in accounting research will include tables of descriptive statistics consisting of summary statistics of dependent and independent variables used in subsequent regression analyses. The paper may also include statistics on sample composition (e.g., split by industry and year). But it seems that there are opportunities to enrich our understanding of the phenomena being studied by description that extends beyond merely providing data for understanding subsequent regression analyses.
% Discuss role of graphical analysis. Good illustrations? RDD?

\subsection{Increase the use of field evidence}

The use of archival data obtained from public databases dominates empirical research in accounting. relies \cite{Soltes:2014gr} discusses the pitfalls of exclusive reliance on archival data. 





\clearpage
\bibliography{jar_methods}

\clearpage

\begin{figure}
	
    \centering
    \caption{Three basic causal diagrams} \label{fig:basic}
    \addtocounter{figure}{-1}
    \begin{subfigure}{\textwidth}
        \caption{$Z$ is a confounder} \label{fig:confound}
        \centering
		\begin{tikzpicture}
		    \node[rectangle] (X) {Treatment variable (X)}; 
		    \node [right = 1 of X] (Y) {Outcome variable (Y)};
		    \node [below = 1 of X] (Z) {``Control" (Z)};
		    \draw (X) edge[->] (Y);
		    \draw (Z) edge[->] (X);
		    \draw (Z) edge[->] (Y);
		\end{tikzpicture}
    \end{subfigure}

    \begin{subfigure}[b]{\textwidth}
        \caption{$Z$ is mediator} \label{fig:mech}
        \centering
		\begin{tikzpicture}
		    %create X and Y node
		    \node[rectangle] (X) {Treatment variable (X)}; 
		    \node [right = 1 of X] (Y) {Outcome variable (Y)};
		    \node [below = 1 of X] (Z) {``Control" (Z)};
		    \draw (X) edge[->] (Y);
		    \draw (Z) edge[<-] (X);
		    \draw (Z) edge[->] (Y);
		\end{tikzpicture}
    \end{subfigure}

    \begin{subfigure}[b]{\textwidth}
    	\caption{$Z$ is a collider} \label{fig:collider}
		\centering
		\begin{tikzpicture}
		    \node[rectangle] (X) {Treatment variable (X)}; 
		    \node [right = 1 of X] (Y) {Outcome variable (Y)};
		    \node [below = 1 of X] (Z) {``Control" (Z)};
		    \draw (X) edge[->] (Y);
		    \draw (Z) edge[<-] (X);
		    \draw (Z) edge[<-] (Y);
		\end{tikzpicture} 

    \end{subfigure}

\end{figure}

\clearpage
\begin{figure} 
    \caption{Causal graph for \citet{Armstrong:2013io}} \label{fig:agl}
    \centering
	\begin{tikzpicture}
		\node[rectangle] (C0) {Compensation$_{t-1}$}; 
		\node[draw,dashed,fill=white] at ([shift={(3,-2.5)}] C0) (EC1) {Shareholder-observable determinants of compensation$_{t+1}$};
		\node [right = 3 of C0] (C1) {Compensation$_{t+1}$};
		\node [below = 3 of C1] (SS0) {Shareholder support$_t$};
		\node [below = 3 of C0] (ISS) {ISS recommendation$_t$};
				
		\draw (C0) edge[->] (ISS);
		\draw (C0) edge[->] (EC1);
		\draw (SS0) edge[->] (C1);
		\draw (C0) edge[->] (C1);
		\draw (ISS) edge[->] (SS0);
		\draw (EC1) edge[->] (C1);
		\draw (EC1) edge[->] (SS0);
	\end{tikzpicture}
\end{figure}


\clearpage
\begin{figure} 
    \centering
	\caption{Identifying effects of analyst coverage changes} 
    \addtocounter{figure}{-1}
    \begin{subfigure}{\textwidth}
        \caption{Causal graph for \citet{Kelly:2012ih}} \label{fig:kl}
        \centering
		\begin{tikzpicture}
			\node[rectangle] (BC) {Brokerage closure$_t$}; 
			\node [right = 1 of BC] (AC0) {Analyst coverage$_t$};
			\node [below = 1 of AC0] (IA0) {Information asymmetry$_t$};
			\draw (BC) edge[->] (AC0);
			\draw (AC0) edge[->] (IA0);
			\draw (IA0) edge[bend right=60, dashed, <->] (AC0);
		\end{tikzpicture}
    \end{subfigure}

    \begin{subfigure}[b]{\textwidth}
        \caption{Causal graph for \citet{Balakrishnan:2014js}} \label{fig:bbkl}
        \centering
		\begin{tikzpicture}
						\node[rectangle] (BC) {Brokerage closure$_t$}; 
			\node [right = 1 of BC] (AC0) {Analyst coverage$_t$};
			\node [right = 1 of AC0] (AC1) {Analyst coverage$_{t+1}$};
			\node [below = 1 of AC0] (IA0) {Information asymmetry$_t$};
			\node [right = 1 of IA0] (IA1) {Information asymmetry$_{t+1}$};
			\node [below = 1 of IA0] (D) {Disclosure};
			\node [right = 1 of D] (LQ) {Liquidity$_{t+1}$};
			\draw (BC) edge[->] (AC0);
			\draw (AC0) edge[->] (IA0);
			% \draw (AC0) edge[->] (AC1);
			\draw (AC1) edge[->] (IA1);
			\draw (IA0) edge[->] (D);
			\draw (IA1) edge[->] (LQ);
			\draw (D) edge[->] (LQ);
			\draw (IA0) edge[bend right=60, dashed, <->] (AC0);
			\draw (IA1) edge[bend right=60, dashed, <->] (AC1);
		\end{tikzpicture}
    \end{subfigure}
    
        \begin{subfigure}[b]{\textwidth}
        \caption{Alternative causal graph for \citet{Balakrishnan:2014js}} \label{fig:bbkl_alt}
        \centering
		\begin{tikzpicture}
						\node[rectangle] (BC) {Brokerage closure$_t$}; 
			\node [right = 1 of BC] (AC0) {Analyst coverage$_t$};
			\node [right = 1 of AC0] (AC1) {Analyst coverage$_{t+1}$};
			\node [below = 1 of AC0] (IA0) {Information asymmetry$_t$};
			\node [right = 1 of IA0] (IA1) {Information asymmetry$_{t+1}$};
			\node [below = 1 of IA0] (D) {Disclosure};
			\node [right = 1 of D] (LQ) {Liquidity$_{t+1}$};
			\draw (BC) edge[->] (AC0);
			\draw (AC0) edge[->] (IA0);
			\draw (AC0) edge[->, line width=2pt] (AC1);
			\draw (AC1) edge[->] (IA1);
			\draw (IA0) edge[->] (D);
			\draw (IA1) edge[->] (LQ);
			\draw (D) edge[->] (LQ);
			\draw (IA0) edge[bend right=60, dashed, <->] (AC0);
			\draw (IA1) edge[bend right=60, dashed, <->] (AC1);
		\end{tikzpicture}
    \end{subfigure}
\end{figure}

\begin{landscape}

\begin{figure}

\caption{Causal diagram for strategic auditor model} \label{fig:audit}

% Figure depicts causal graph for model with strategic auditor discussed in the text. 

\begin{tikzpicture}
    
    % Exogenous stuff
    \node[rectangle] (BONUS) {BONUS};  
    \node [right = 1 of BONUS] (SALARY) {SALARY};  
    \node [left = 1 of BONUS] (SEG) {SEG};    
    \node [left = 1 of SEG] (INT) {INT};    
	
    \node[below = 1 of BONUS,draw,dashed,fill=white] (B) {Managerial incentives ($B$)}; 
    
    % Equilibrium
    \node [right = 1 of B, below = 2 of B,draw,dashed,fill=white] (alpha) {Audit effort};
    \node [right = 1 of alpha,draw,dashed,fill=white] (beta) {Attempted misstatement};
    \node [below = 1 of SALARY,draw,dashed,fill=white] (C_M) {Cost of\\manipulation ($C_M$)};
    \node [below = 1 of beta,draw,dashed,fill=white] (M) {Misstatement};
    \node [left = 1 of B,,draw,dashed,fill=white] (C_A) {Cost of audit effort ($C_A$)};
    \node [below = 4 of C_A,draw,dashed,fill=white] (p_I) {$\mathrm{Pr}$(Detection by subsequent investigation)\\ ($p_I$)};
    \node [below = 1.5 of INT,draw,dashed,fill=white] (C_R) {Auditor\\reputational concerns ($C_R$)};  
    
    % Unobservable outcome

    % Observable outcome
    \node [below = 2 of alpha] (R) {Restatement};
    
    \draw (INT) edge[->] (C_A);
	\draw (SEG) edge[->] (C_A);
    \draw (BONUS) edge[->] (B);
    
    \draw (BONUS) edge[->] (B);
    \draw (SALARY) edge[->] (C_M);
    \draw (B) edge[->] (beta);
    
    \draw (C_R) edge[->] (beta);
    \draw (C_A) edge[->] (beta);
    \draw (C_M) edge[->] (beta);
    \draw (C_M) edge[->] (beta);
    \draw (p_I) edge[->] (beta);
    
    \draw (B) edge[->] (alpha);
	\draw (C_R) edge[->] (alpha);
    \draw (C_A) edge[->] (alpha);
    \draw (C_M) edge[->] (alpha);
    \draw (p_I) edge[->] (alpha);
    
    \draw (alpha) edge[->] (M);
    \draw (beta) edge[->] (M);
    \draw (M) edge[->] (R);
    \draw (p_I) edge[->] (R);
    
    % \draw (IA) edge[bend right=60, dashed, <->] (AC);
\end{tikzpicture}
\end{figure}
\end{landscape}



\clearpage
\begin{table}[t]

\caption{Descriptive statistics.} \label{tab:desc}

\justify{
{\bf RESTATE} is a zero-one indicator for whether the firm made a restatement in a particular year.
{\bf SALARY} is the CEO's annual base salary.  
{\bf EQUITY} is the fraction of a CEO's total pay that is incentive compensation.  
{\bf BIG4} is a zero-one indicator for whether the firm uses a Big 4 auditor.
{\bf FINDIRECT} is the fraction of the board of directors with a finance background. 
{\bf INT} is a zero-one indicator for non-US corporation.  
{\bf SEG} is number of the firm's business segments.\\}

\begin{center}
\begin{tabular}{|l|c|}
\hline
             & {\bf Sample }   \\
{\bf Variable}  &   {\bf Mean }   \\ 
          &  (Std Dev)   \\ \hline
\T {\bf RESTATE} &   0.102  \\
                           &   0.303  \\[.6em]
    {\bf SALARY} &     0.95  \\
                           &     0.15  \\[.6em]
    {\bf EQUITY} &    0.26  \\
                           &     0.29  \\[.6em]
    {\bf BIG4} &   0.76  \\
                           &     0.43  \\[.6em]
    {\bf FINDIRECT} &    0.08  \\
                           &     0.08  \\[.6em]
    {\bf SEG} &    4.41  \\
                           &    3.02  \\[.6em]
    {\bf INT} &     0.30  \\
                           &     0.46  \\[.6em]
\hline
\end{tabular}
\end{center}
\end{table}

\begin{table}[t]
\caption{Logit Regression Results} \label{tab:logit}

\justify{This table presents results from logistic regressions of 

{\bf RESTATE}, a zero-one indicator for whether the firm made a restatement in a particular year, on a proxy for managerial incentives ({\bf EQUITY}) and controls.
{\bf SALARY} is the CEO's annual base salary.  
{\bf EQUITY} is the fraction of a CEO's total pay that is incentive compensation.  
{\bf BIG4} is a zero-one indicator for whether the firm uses a Big 4 auditor.
{\bf FINDIRECT} is the fraction of the board of directors with a finance background. 
{\bf INT} is a zero-one indicator for non-US corporation.  
{\bf SEG} is number of the firm's business segments.\\}

\begin{center}
\begin{tabular}{|l|cc|cc|}
\hline
             & \multicolumn{2}{c|}{\bf Specification 1}  &  \multicolumn{2}{c|}{\bf Specification 2}   \\
 {\bf Coefficient on}  &  {\bf Coefficient} & {\bf Marginal Effect} & {\bf Coefficient} & {\bf Marginal Effect}  \\ 
          &  (Std Dev)  & (Std Dev) &  (Std Dev)  & (Std Dev) \\ \hline
\T {\bf Intercept} &   -2.210*   &   & -2.786*   &    \\
                   &   0.308   &   & 0.349   &   \\[.6em]
 {\bf SALARY}    &   0.118    &  0.010  & 0.010    &  0.001   \\
                   &   0.317   &  0.027 & 0.327   &  0.027 \\[.6em]
 {\bf EQUITY}  &   -0.644*   &  -0.055* & -0.673*   &  -0.056*  \\
                   &   0.185   &  0.016 & 0.201   &  0.017 \\[.6em]
 {\bf BIG4 }  &       &    & 0.104    &  0.009   \\
                   &      &   & 0.130   &  0.011 \\[.6em]
 {\bf FINDIRECT}  &       &    & -0.058    &  -0.005   \\
                   &      &   & 0.740   &  0.062 \\[.6em]
 {\bf INT }  &       &    & 0.630*   &  0.052*  \\
                   &      &   & 0.120   &  0.010 \\[.6em]
 {\bf SEG}  &       &    & 0.086*   &  0.007*  \\
                   &      &   & 0.021   &  0.002 \\[.6em]
\hline
\end{tabular}
\end{center}
\end{table}

%TODO: Add standard errors and table header to Table 3.
%TODO: Include marginal effects.
\begin{table}[t]
\caption{Logit GMM Estimates for the Strategic Auditor Model} \label{tab:gmm}

\justify{This table presents results for GMM estimates of the strategic auditor model of Section \ref{sec:struct}.\\}

\begin{center}
\begin{tabular}{|c|cc|}
\hline
\T  &  {\bf Estimated}  & {\bf Bootstrap Std} \\ 
\B  & {\bf Coefficient} & {\bf Error} \\ \hline
\T $\theta_0= \dfrac{(1-v_0)a_0m_0}{(p_0-v_0)r_0b_0}$ & 0.0342 &\\[1.5em]
 $\theta_1=\dfrac{(1-v_0)a_1m_0}{(p_0-v_0)r_0b_0}$  &  0.0022 & \\[1.5em]
$\theta_2=\dfrac{(1-v_0)a_2m_0}{(p_0-v_0)r_0b_0}$  &  0.0145   & \\[1.5em]
$\theta_3=\dfrac{(1-v_0)a_0m_1}{(p_0-v_0)r_0b_0}$  &  0.0108  & \\[1.5em]
$\theta_4=\dfrac{(1-v_0)a_1m_1}{(p_0-v_0)r_0b_0}$ &  0.1015  &  \\[1.5em]
$\theta_5=\dfrac{(1-v_0)a_2m_1}{(p_0-v_0)r_0b_0}$ &  -0.0064  & \\[1.5em]
\B$\theta_6=\dfrac{b_1}{b_0}$ &  0.4557  & \\[1.6em]
\hline
\end{tabular}
\end{center}
\end{table}


\end{document}
	