\section{Alternative approaches}

In the first half of the paper, we have argued that causal inference remains at the heart of accounting research, but that statistical approaches to obtaining plausible causal inference are unlikely to assist researchers in the vast majority of cases.

The methods for estimating causal effects described above (i.e., instrumental variables, RDD, and natural experiments) are unlikely to produce more than a tiny fraction of the research in top accounting journals.

This raises the question: What should researchers do? Do we stop doing research? Do we need to give up on causal inference? The objective of the second part of this paper is to provide some guidance to research on possible answers to these questions.

\subsection{Deeper institutional understanding}

We believe that accounting research could benefit greatly from increased emphasis on research that enhances our understanding of real-world phenomena and institutions. There are several benefits that would accrue to such efforts.

\begin{enumerate}
\item Better hypothesis development.  
\item % Too many papers in accounting research test ``armchair" hypotheses with no basis in real-world phenomena.
\item Enhanced identification of causal effects.
\item More relevant research results.
\end{enumerate}

One alternative approach to empirical research in economics is labeled the structural approach (Wolpin 2013).

The structural approach ``requires that a researcher explicitly specity a model of economic behavior, that is, a theory." (Wolpin, 2013, p.2).


\subsubsection{Better hypothesis development}
Many papers examine hypotheses that are not motivated by prior theory,  observations of real-world phenomena, or beliefs of practitioners. Instead, researchers often propose and empirically test hypotheses in the same paper.

\subsubsection{Enhanced identification of causal effects}
One point that is often overlooked by researchers seeking to make causal inferences is that a deep understanding of the treatment assignment mechanism is necessary to support claims that such assignment is as-if random. Such understanding is not statistical, but relates to the facts of assignment itself. \textbf{Get some examples from Rubin on this.}

\subsection{Increased emphasis on measurement}
It is often claimed that accounting researchers have a ``comparative advantage in measurement.'' This claim is presumably based on the notion that accounting in practice relates to measurement of the performance and financial position of organizations. However, other disciplines have created extensive literatures studying measurement issues. For example, psychologists have long grappled with issues of measurement of various constructs such as intelligence. This work has given rise to deep statistical techniques.

\section{How to enhance institutional understanding}

\subsection{Greater emphasis on description}
A typical paper in accounting research will include tables of descriptive statistics consisting of summary statistics of dependent and independent variables used in subsequent regression analyses. The paper may also include statistics on sample composition (e.g., split by industry and year). But it seems that there are opportunities to enrich our understanding of the phenomena being studied by description that extends beyond merely providing data for understanding subsequent regression analyses.
% Discuss role of graphical analysis. Good illustrations? RDD?

\subsection{Increase the use of field evidence}

The use of archival data obtained from public databases dominates empirical research in accounting. relies \cite{Soltes:2014gr} discusses the pitfalls of exclusive reliance of archival data. 


\begin{itemize}
\item Measurement
\item Institutional grounding
\item Descriptive
\item Structural
\end{itemize}
