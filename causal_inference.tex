\documentclass[11pt,reqno,titlepage]{amsart}

\usepackage[marginratio=1:1,margin=1in]{geometry}  % See geometry.pdf to learn the layout options. There are lots.
%\geometry{letterpaper} % ... or a4paper or a5paper or ... 
%\geometry{landscape}  % Activate for for rotated page geometry
% \usepackage[parfill]{parskip}    % Activate to begin paragraphs with an empty line rather than an indent
%\usepackage{amsfonts}
\usepackage{palatino}
\usepackage[longnamesfirst]{natbib}
\usepackage{hyperref} 
% \usepackage{paralist}
\usepackage{tikz}
\usepackage{pgf}
\usepackage{attrib}
\usepackage{caption}
\usepackage{subcaption}
\usepackage{pdflscape}
\usepackage{setspace}
\usepackage{ragged2e}
\raggedbottom

\setlength\parindent{1cm}

\newtheorem{theorem}{Theorem}
\newtheorem{lemma}[theorem]{Lemma}
\theoremstyle{definition}
\newtheorem{definition}{Definition}

\tikzset{every node/.style = 
		    	{shape = rectangle, rounded corners, fill = black!30!white,
		   		text width = 3cm, minimum height = 1.5cm, align = center, text = black},
		    every edge/.style = {draw, ->, line width=2pt, black}}

% The following are for Peter's tables		    
\newcommand\T{\rule[0em]{0pt}{1.5em}} % Top strut
\newcommand\B{\rule[-1em]{0pt}{0pt}} % Bottom strut

% To eliminate unnecessary space before bullet points
\usepackage{enumitem}
\setlist{nolistsep}

\title[Causal Inference in Accounting]{Causal Inference in Accounting Research}

\author{Ian D. Gow}
\author{David F. Larcker}
\author{Peter C. Reiss}

%\date{}   % Activate to display a given date or no date

\begin{document}
\usetikzlibrary{automata, shapes, calc, positioning}

\bibliographystyle{chicago}
% Quick LaTeX Guide for Dave (originally for Suraj).

% - Percent signs (%) mark comments. To get a percent sign, escape it by putting a backslash in front.
%  & is another special character in LaTeX. Use \& to get &.
% Note that each part of the document is in a separate file (so we can edit in parallel).
% Citations are automatic with the correct key. 
% LaTeX doesn't pay attention to multiple spaces. Also adjacent lines get collapsed into single paragraphs.
% Insert a blank line between lines that are part of two separate paragraphs.
% It's actually helpful to put every sentence on a separate line. 
% You need two line breaks to indicate a paragraph.
% \section, \subsection, and \subsubsection have the obvious meanings.
% Note that there is a file jar_methods.bib in the list of files to the right that this pulls bibliographic information from.
\begin{titlepage}
  \centering
  	\begin{large}
  	\textbf{Causal Inference in Accounting Research\footnote{We thank seminar participants at London Business School, Karthik Balakrishnan, Philip Berger, Bob Kaplan, Alexander Ljungqvist, Eugene Soltes, Dan Taylor, Ro Verrecchia, Charlie Wang, and Anastasia Zakolyukina for helpful discussions and feedback.}} \\	
  	\end{large}
  	\vspace{60pt}
	\textbf{Ian D. Gow} \\
	Harvard Business School \\
	email: igow@hbs.edu

  	\vspace{30pt}
	\textbf{David F. Larcker} \\
	Stanford Graduate School of Business \\
	Rock Center for Corporate Governance \\
	email: dlarcker@stanford.edu \\
		
	\vspace{30pt}
	\textbf{Peter C. Reiss} \\
	Stanford Graduate School of Business \\
	email: preiss@stanford.edu \\

	\vspace{30pt}
	\today
	
	\vspace{30pt}
	\centerline{\bf Rough Draft}
	

\end{titlepage}


\begin{abstract}
	This paper examines the approaches accounting researchers use to draw causal inferences using observational (or non-experimental) data. 
	The vast majority of accounting research papers draw causal inferences notwithstanding the well-known difficulties with doing so with observational data.
	While a minority of papers seek to use quasi-experimental methods to draw inferences, these methods are usually applied inappropriately.
	We believe that accounting research would benefit from: 
		a greater focus on the study of causal mechanisms (or causal pathways); 
		increased emphasis on structural modeling of the phenomena of interest; 
		and, more in-depth descriptive research. 
	We argue these changes are possible and offer a practical path forward. 
\end{abstract}

\maketitle
\clearpage

\section{Introduction}

\begin{quotation}\begin{singlespace} 
There is perhaps no more controversial practice in social and biomedical research than drawing inferences from observational data.
Despite \dots problems, observational data are widely available in many scientific fields and are routinely used to draw inferences about the causal impact of interventions.
The key issue, therefore, is not whether such studies should be done, but how they may be done well.
\attrib{\citealt{Berk:1999uz}}
\end{singlespace}
\end{quotation}

\begin{doublespace} 
Most empirical research in accounting relies on observational data (i.e., data produced by processes outside the control of the researcher).
This paper evaluates the different approaches accounting researchers use to draw causal inferences from observational data. 
Our discussion draws on developments in 
fields such as statistics, econometrics and epidemiology. Our goal is to identify areas for improvement and suggest how empirical accounting research can increase its impact.

The importance of causal inference in accounting research is clear from the research questions that accounting researchers seek to answer. 
Most long-standing questions in accounting research are causal: 
Does conservatism affect the terms of loan contracts?
Do higher quality earnings reports lead to lower information asymmetry? 
Did IFRS cause an increase in liquidity in the jurisdictions that adopted it?
Do managerial incentives lead to more misstatements of financial reports?
That accounting researchers focus on causal inference is consistent with the view that ``the most interesting research in social science is about questions of cause and effect" \cite[p. 3]{Angrist:2008vk}.
Simply documenting descriptive correlations provides no basis for understanding what would happen should circumstances change, 
whereas using data to make inferences that support or refute broader theories could facilitate these kinds of predictions.
Accounting researchers are aware of problems that can arise from the use of observational data.
% refer to Heckman-style methods as "treatment effect models?  Maybe cite Francis paper in TAR here?
For instance, many researchers who estimate so-called treatment effect models are aware that random assignment is usually required to ensure that differences between the treatment and control samples is due to the treatment. 
Despite this awareness, many accounting ``treatments" are not randomly assigned, and the researcher then is in the position of arguing that they can account for all other reasons that might lead to a difference between the treatment and control samples.

To understand what is actually done in empirical accounting research, we examined all empirical papers published in the leading accounting journals in 2014. 
We found that overwhelmingly these papers used observational data and descriptive statistical models to estimate the causal impact of $X$ on $y$. 
This approach makes strong assumptions about the functional relationship between $X$ and $y$, and requires that sufficient controls are present to account for changes in conditions related to changes in $X$.
Section \ref{sec:causal} examines these assumptions in detail.
We introduce causal graphs, a device that can clearly communicate the causal reasoning underlying an empirical model.
These graphs should prove useful to accounting researchers who wish to communicate the cause-and-effect logic underlying their regression analyses.
%Nonetheless, the fact that treatment is not randomly assigned leads many researchers to be skeptical of any efforts to use regression analyses of observational data for causal inference.


Recently, some social scientists have held out hope that better research designs and statistical methods can increase the credibility of causal inferences.
For example, \citet{Angrist:2010jv} claim that ``empirical microeconomics has experienced a credibility revolution, with a consequent increase in policy relevance and scientific impact.''  
\citet[p. 26]{Angrist:2010jv} argue that such ``improvement has come mostly from better research designs, either by virtue of outright experimentation or through the well-founded and careful implementation of quasi-experimental methods."
These quasi-experimental methods are used to some degree in accounting research.\footnote{
We use the term ``quasi-experimental" methods to refer to those methods that have a plausible claim to ``as if" random assignment to treatment conditions.
The term ``as if" is used by \citet{Dunning:2012tt} to acknowledge the fact that assignment is not random is such settings, but is claimed to be \emph{as if} random assignment had occurred.}
Our survey of research published in 2014 finds five studies claiming to study natural experiments (or ``exogenous shocks") and ten studies using instrumental variables.

In Section \ref{sec:quasi}, we examine and evaluate the use of quasi-experimental methods in accounting research.
We find that these approaches generally do not enhance the credibility of the causal inferences drawn by the authors.
We argue that more transparent reasoning, perhaps with the aid of causal diagrams, would have revealed implicit maintained assumptions that are difficult to defend.
While regression discontinuity designs might seem to rely on weaker assumptions than other methods such as instrumental variables, they too are based on similar maintained assumptions. Moreover, even when such designs satisfy the maintained assumptions, the estimated effects may not be generalizable outside the immediate application.
In summary, we argue that the assumptions underlying quasi-experimental methods are unlikely to apply in the vast majority of accounting research settings that rely on observational data.
Variations in treatments are rarely random and exogenous instruments are hard to come by.

% Dave: "One especially promising path is use of field experiments with randomized treatments to measure causal effects (a good example -- Roberts QJE paper)." % \citet{Roberts:2013cz} 
% Ian: "I think we could discuss these as interesting and a path forward, but scope them out as our focus is on observational data."
Ultimately we believe that accounting research needs to lean less heavily on statistical models applied to observational data.
Statistical methods alone cannot solve the inference issues that arise in observational data. 
The second part of the paper identifies approaches that can provide a path forward for accounting research.
In particular, we discuss work on causal models in economics, statistics, and other fields.
% Structural model does not solve endogeneity, but makes it explicit and gives it a theoretical and institutional basis.

We see at least three useful paths forward:

\vskip -10pt
\begin{itemize}
\item At a minimum, there should be an increased emphasis on the study of causal mechanisms.
Here we believe that causal diagrams can help clarify what assumptions an accounting researcher must make about putative causes and putative effects. 
Progress can occur even in the absence of knowing the exact process that generated the (observational) data.
\item We believe that there should be an increased use of structural modeling methods. Structural models provide a more complete characterization of the behavior and institutions that produce a phenomenon of interest. They also allow researchers to reason about what would happen if some features of the model change. However, we certainly acknowledge that these advantages are also associated with important disadvantages. 	%TODO: Flesh this out somewhat. What are the benefits of doing structural modeling?
\item There are many questions in accounting that have not yet be addressed by formal models.  In these settings, it is important to conduct descriptive research aimed at developing hypotheses, particularly when it comes to causes and effects. At present, most hypotheses that are tested
are only loosely tied to the accounting institutions and phenomena of interest. One goal of this approach is to entice theorists to build models
that empiricists can actually "take to data."
	%TODO: The 2017 JAR call for papers is arguably evidence of this concern being shared by others. Considering adding a footnote saying this.  DL -- this is a good idea
	%A better approach would be to draw on detailed studies that better \emph{describe} 
	%Such studies likely will have a greater chance of identifying causal pathways and entice theorists to build better models of the phenomena of interest.
	%TODO: Trim the "descriptive research" bullet point.
\end{itemize}

\vskip 10pt
The balance of the paper is structured as follows.
Section \ref{sec:causal} provides an overview of the issues observational data pose for drawing causal inferences in accounting research; 
it also suggests frameworks for identifying and analyzing these issues.
Section \ref{sec:quasi} evaluates the use and misuse of quasi-experimental methods.
We then turn to outlining a path forward for accounting research.
Section \ref{sec:mech} sketches the idea of mechanism-based causal inference.
Section \ref{sec:struct} illustrates how structural modeling approaches might be used by accounting researchers; it also discusses structural models strengths and weaknesses.
In Section \ref{sec:desc} we argue for richer descriptive research that can shed light on causal issues.
Section \ref{sec:conclude} concludes.

\section{Causal inference: An overview} \label{sec:causal}

\subsection{Causal inference in accounting research}
%TODO: Flag experimental research, then say we're focused on studies using observational data.
%TODO: Add a footnote to the guy doing something similar is AOS -- claims that only 3% of papers are causal.
%TODO: Add a footnote *somewhere* discussing other work on "causal diagrams" in accounting research. I think we want to do no more than mention that these exist and suggest that what we're doing is a bit more formal.


To get a sense for the importance of causal questions in accounting research,
we examined all papers published in 2014 in the \textit{Journal of Accounting Research}, \textit{The Accounting Review}, and the \textit{Journal of Accounting and Economics}.
We counted 139 papers, of which 125 are original research papers. Another 14 papers survey or discuss other papers.
We classify each of the 125 research papers into one of four categories:  ``Theoretical'' (7); ``Experimental'' (12); ``Field" (3); or ``Archival Data" (103). 
For our discussion below, we collect the field and archival data papers into a single category, "Observational".

For each non-theoretical paper, we determine whether the primary or secondary research questions are ``causal". One often does not have to look far to find causal claims. Words
such as ``effect of \dots" or ``impact of \dots" in titles often signal causal inferences   
\citep[e.g.][]{Cohen:2014jl,Clorproell:2014cv}. 
Abstracts and conclusions also reveal that authors have causal inferences as a goal. 
For example, \citet{deFranco:2014ct} asks ``how the tone of sell-side debt analysts' discussions about debt-equity conflict events \emph{affects} the informativeness of debt analysts' reports in debt markets.''

We recognize that some authors might disagree with our classifications.
We know some would say their paper only claimed that ``theory predicts $X$ is associated $Y$ and, consistent with that theory, we show $X$ is associated with $Y$".
However in the abstracts, introductions and conclusions of papers, these qualifications are rarely present. 
Instead the writing suggests the evidence in the paper tilts the scale in the direction of confirming a causal hypothesis.\footnote{
Papers that seek to estimate a causal effect of $X$ on $Y$ are a subset of papers we classify as causal.
A paper that argues that $Z$ is a common cause of $X$ and $Y$ and claims to find evidence of this is still making causal inferences (i.e., that $Z$ causes $X$ and $Z$ causes $Y$.
However, we do not find this kind of reasoning to be common in our survey.}
%TODO: Find an example of this kind of paper. I know there are some.

Of the 106 original papers using observational data, we coded 91 as about causality.\footnote{While we exclude research papers using experimental methods, all of these papers also seek to draw causal inferences.}
Of the remaining empirical papers, we coded 7 papers as having a goal of ``description'' (including two of the three field papers). 
For example, \citet{Soltes:2013ba} uses data collected from one firm to describe analysts' private interactions with management. Understanding how these interactions take place is key to understanding whether and how they transmit information to the market.
We coded 5 papers as having a goal of ``prediction.'' 
For example, \citet{Czerney:2014bv} examine whether the inclusion of ``explanatory language" in unqualified audit reports can be used to predict the detection of financial misstatements in the future.
We coded 3 papers as having a goal of ``measurement.'' 
For example, \citet{Cready:2014ji} examine whether inferences about traders based on trade size are reliable and suggest improvements to the measurement of variables used by accounting researchers.

In summary, we find that most original research papers use observational data and that about 90\% of these papers seek to draw causal inferences.
The most common estimation methods used in these studies include ordinary least-squares (OLS) regression, difference-in-difference estimates, and propensity-score matching.
While it is widely understood that OLS regressions that use observational data produce unbiased estimates of causal effects only under very strong assumptions, the credibility of these assumptions is rarely explicitly addressed.\footnote{
There are settings where difference-in-difference and fixed effect estimators may deliver causal estimates.
For example, if assignment to treatment is random, then it is possible for a difference-in-difference estimate using pre- and post-treatment data to yield unbiased estimates of causal effects.
But in this case, it is detailed understanding of the setting, not the method \emph{per se}, that makes these estimates credible.}
%TODO: So why do we see so many papers using OLS for causal inference? My conjecture: If the results is the "right" one, no-one cares about endogeneity. So WTF are we doing when we do research? (Even by the standards of this paper, this point is a little "heavy" I think.)

\subsubsection{Difference-in-difference and fixed effect estimators}
Accounting researchers have come to view some statistical methods as requiring fewer assumptions and thus being less subject to problems when it comes to drawing causal inferences. 
\citet[p.\,12]{Angrist:2010jv} include so-called difference-in-difference (DD) estimators on their list of such quasi-experimental methods, along with ``instrumental variables and regression discontinuity methods."\footnote{As \citet[p.\,228]{Angrist:2008vk} argue that ``DD is a version of fixed effects estimation," we discuss these methods together.}
Enthusiasm for DD designs perhaps stems from a belief that these are ``quasi-experimental" methods in the same sense as the other two approaches cited by \citet[p.\,12]{Angrist:2010jv}.
But the essential feature that instrumental variables and regression discontinuity methods rely on is the ``as if" random treatment assignment mechanism.
If treatment assignment is driven by unobserved confounding variables, then DD and fixed-effect estimates will be biased and inconsistent. 
We are not confident that there are many settings in accounting that are likely to satisfy random treatment assignment.
This then means that there is a (heavy) burden on the users of these methods to explain why they believe that DD or fixed-effect estimators allow them to recover unbiased estimates of causal effects.

%Even when assignment to treatment is random, care needs to be taken in interpreting estimates as causal effects.
%For example, \citet[p.\,1305]{Cadman:2014cr} conjecture that ``VCs have strong incentives to design compensation schemes that provide CEOs with short-horizon incentives in the fiscal years after the IPO." 
%The main analysis in support of this hypothesis is a regression DD analysis \citep[pp.\,233--241]{Angrist:2008vk} using the pre-IPO year as the pre-treatment period, and the two years after IPO as the post-treatment period. 
%However, one treatment of interest (i.e., being VC-backed) is likely implemented well before the ``pre-treatment" period, making it difficult to consider the pre-treatment values of the outcome variable as not being caused by the treatment.

\subsubsection{Propensity score matching}
Another method that has become popular in accounting research is propensity score matching (PSM).
Regression methods can be viewed as making model-based adjustments to address confounding variables.  
Stuart and Rubin (2007) argue that 

\begin{quote}
``[M]atching methods are preferable to these model-based adjustments for two key reasons. 
First, matching methods do not use the outcome values in the design of the study and thus preclude the selection of a particular design to yield a desired result.
Second, when there are large differences in the covariate distributions between the groups, standard model-based adjustments rely heavily on extrapolation and model-based assumptions.
Matching methods highlight these differences and also provide a way to limit reliance on the inherently untestable modeling assumptions and the consequential sensitivity to those assumptions."
\end{quote}
For these reasons, PSM methods can prove useful when faced with observational data.
However, PSM does \emph{not} provide ``the closest archival approximation to a true random experiment" and does \emph{not} not represent "the most appropriate and rigorous research design for testing the effects of an ex ante treatment" \citep[p.\,1429]{Kirk:2014gx}.
\citet[pp.\,73-75]{Rosenbaum:2009ul} points out that matching is ``a fairly mechanical  task," and when assignment to treatment is driven by unobservable variables, PSM-based estimates may be biased as much as regression estimates.
We agree with \citet{MinuttiMeza:2014fn} who argues that ``matching does not necessarily eliminate the endogeneity problem resulting from unobservable variables driving [treatment] and [outcomes]."

%TODO: Add a general evaluation of the use of PSM for causal inferences in accounting 
%TODO: Expand survey to discuss use of PSM in accounting.
%TODO: Discuss papers matching on post-treatment variables.
%TODO: Find a home for discussion of bounding. I don't think it belongs with PSM.

\subsection{Causal inference: A brief overview}
In recent decades, the definition and logic of causality has seen renewed interest from researchers in such diverse fields as epidemiology, sociology, statistics, and computer science. 
Work by \citet{Rubin:1974im,Rubin:1977dv} and \citet{Holland:1986p7458} formalizing ideas from the potential-outcome framework of \citet{Neyman:1923aa} have led to the so-called the Rubin causal model. 
Other fields have used path analysis, as initially studied by geneticist Sewell Wright \citep{Wright:1921aa}, as an organizing framework.
In economics and econometrics, early proponents of structural models were quite clear about how causal statements must be tied to theoretical economic models.
As discussed by \citet{Heckman:2015ez}, \citet{Haavelmo:1943cl,Haavelmo:1944jq} promoted structural models ``based on a system of structural equations that define causal relationships among a set of variables."
%standard econometric texts generally avoid explicit discussion of causation.
%For example, Greene (2003) does not discuss causality except for Granger causality, which is widely recognized as a purely statistical notion quite distinct from notions of one variable causing another.
% However, some economists have explic
\citet[p.\,979]{Goldberger:1972cq} promoted a similar notion: ``By structural equation models, I refer to stochastic models in which each equation represents a causal link, rather than a mere empirical association \dots
Generally speaking the structural parameters do not coincide with coefficients of regressions among observable variables, but the model does impose constraints on those regression coefficients."
\citet{Goldberger:1972cq} focuses on linking such approaches to the path analysis of Wright.

An important point worth emphasizing is that model-based causal reasoning is distinct from statistical reasoning. 
This point seems to be understood, but regularly forgotten.
Suppose we observed data on $x$ and $y$ and know only that one causes the other. How do we distinguish between whether $x$ causes $y$ or $y$ causes $x$? 
Statistics can help us determine whether $x$ and $y$ are correlated, but correlations do not establish causality.
Only with assumptions about causal relations between $x$, $y$, and other variables (i.e., a theory) can we infer causality.
While theories may be informed by evidence (e.g., prior research may suggest a given theory is more or less plausible), they also encode our understanding of causal mechanisms (e.g., barometers do not cause rain) and economic and behavioral assumptions.

\subsection{Causal diagrams: A primer}
Computer and decision scientists, as well as researchers in other disciplines, have recently sought to develop an analytical framework for thinking about causal models and their connection to probability statements \citep{Pearl:2009kh}.
Pearl's framework, which he calls the structural causal model, uses causal diagrams to describe causal relationships. 
These diagrams encode causal assumptions and visually communicate how a causal inference is being drawn from a given research design.
Given a \emph{correctly specified} causal diagram, these criteria can be used to verify conditioning strategies, instrumental variable designs, and mechanism-based causal inferences.\footnote{While \citet[p.248]{Pearl:2009kh} defines an instrument in terms of causal diagrams, additional assumptions (e.g., linearity) are often needed to estimate causal effects using an instrument \citep{Angrist:1996p7456}.}

We use Figure \ref{fig:basic} to illustrate the basic ideas of causal diagrams and how they can be used to facilitate causal inference.
Figure \ref{fig:basic} depicts three variants of a simple causal graph with three variables, all of which are observable.
In each case, we are interested in estimating the causal effect of $X$ on $Y$ in the presence of a third variable, $Z$ that is related to $X$ and $Y$ in some fashion.
The only difference between the three graphs is in the direction of arrows linking either $X$ and $Z$ or $Y$ and $Z$.
The boxes (or ``nodes") represent random variables and the arrows (or ``edges") connecting boxes represent hypothesized causal relations, with each arrow pointing from a cause to a variable assumed to be affected by it.\footnote{
That arrows have a direction accounts for the ``D" in DAG (Directed Acyclic Graph). 
The acyclic (``A") component means that there must be no cycles in the graph. 
Cycles cause obvious problems in causal reasoning.
 An example would be $X \rightarrow Y \rightarrow Z \rightarrow X$. 
In this graph there is no ultimate cause. 
Graphs make a distinction between observed and unobserved random variables.
In some cases, an unobserved joint determinant of two random variables will not be represented explicitly, but replaced by a dashed, ``undirected" edge between those two random variables.}

The criterion developed by \citet{Pearl:2009vo} implies that very different conditioning strategies are needed for each of the causal diagrams (see Appendix \ref{append} for a more formal treatment). 
\citet{Pearl:2009vo} shows that, if we are interested in assessing the causal effect of $X$ on $Y$, that we may be able to do so by conditioning on a set of variables, $Z$, that satisfies what \citet{Pearl:2009vo} labels the ``back-door criterion" \citep[p.79]{Pearl:2009vo}.\footnote{
Intuitively, the back-door criterion requires that $Z$ blocks (and does not open) ``back-door" paths.
A back-door path can be thought of as a way for $X$ to be associated with $Y$ due to associations with other variables rather than causal links from $X$ to $Y$.
See Appendix \ref{append} for a more formal discussion.}
While conditioning is much like the standard notion of ``controlling for" variables by including them as additional regressors in OLS regression, there are critical differences.
First, reflecting the non-parametric nature of causal diagrams in their most general form, conditioning in principle means estimating effects for each distinct level of the set of variables in $Z$.
Second, as we discuss below, the inclusion of a variable in $Z$ may actually result in biased estimates of causal effects.
We next apply these criteria to each of the three figures in Figure \ref{fig:basic}.	

Figure \ref{fig:confound} is straightforward. 
Here it is apparent that we need to condition on $Z$ and, if we do so, we can estimate the causal effect of $X$ on $Y$.
This situation is a generalization of a linear model in which $Y = X \beta + Z \gamma + \epsilon_Y$ and $\epsilon_Y$ is independent of $X$ and $Z$, but $X$ and $Z$ are correlated.
In this case, it is well known that omission of $Z$ would result in a biased estimate of $\beta$, the causal effect of $X$ on $Y$, but by including $Z$ in the regression, we get an unbiased estimate of $\beta$.\footnote{
Inclusion of $Z$ blocks the back-door path from $Y$ to $X$ via $Z$.}
In this situation, $Z$ is called a \emph{confounder}.

Figure \ref{fig:mech} is a bit different. Here we have $Z$ acting as a \emph{mediator} of the effect of $X$ on $Y$.
No conditioning is required in this setting to obtain an unbiased estimate of the effect of $X$ on $Y$.
But, the back-door criterion not only implies that we need not condition on $Z$ to obtain an unbiased estimate of the causal effect of $X$ on $Y$, but that we should not condition of $Z$ to get such an estimate.

Finally in Figure \ref{fig:collider}, we have $Z$ acting as what is referred to as a ``collider" variable \citep{Glymour:2008aa,Pearl:2009kh}.\footnote{
The two arrows from $X$ and $Y$ ``collide" in $Z$.} 
The back-door criterion not only implies that we need not condition on $Z$, but that we \emph{should} not condition of $Z$ to get an unbiased estimate of the causal effect of $X$ on $Y$.
While in epidemiology, the issue of ``collider bias \dots can be just as severe as confounding" \citep[p.\,186]{Glymour:2008aa}, it appears to receive less attention in accounting research than confounding.\footnote{
Many intuitive examples of collider bias involve selection or stratification.
Admission to a college could be a function of combined test scores and interview performance exceeding a threshold, i.e., $T + I \geq C$. Even if $T$ and $I$ are unrelated unconditionally, a regression of $T$ on $I$ conditioned on admission to college is likely to show a negative relation between these two variables.}

\subsubsection{Causal diagrams: Applications in accounting}
A typical paper in accounting research will include many variables  to ``control for" potential confounding of causal effects.
But why many of these variables should be considered confounders, in which case they should be controlled for, rather than mediators or colliders, in which case ``controlling for" these variables may lead to bias, is often unclear.

One paper that does discuss this distinction is \citet{Larcker:2007aa}, who use a multiple regression (or logistic) model of the form:\footnote{We alter the mathematical notation of  \citet{Larcker:2007aa} to conform with notation we use here.}
\begin{equation}
Y = \alpha + \sum_{r \in R} \gamma _r Z_r + \sum_{s \in S} \beta_s X_s + \epsilon \label{eqn:lrt1}
\end{equation}

\citet{Larcker:2007aa} suggest that 
\begin{quote}
``One important feature in the structure of Equation \ref{eqn:lrt1} is that the governance factors [$X$] are assumed to have no impact on the controls (and thus no indirect impact on the dependent variable). 
As a result, this structure may result in conservative estimates for the impact of governance on the dependent variable. Another approach is to only include governance factors as independent variables, or:
\begin{equation}
Y = \alpha + \sum_{s \in S} \beta_s X_s + \epsilon \label{eqn:lrt2}
\end{equation}
The structure in Equation \ref{eqn:lrt2} would be appropriate if governance impacts the control variables and both the governance and control variables impact the dependent variable (i.e., the estimated regression coefficients for the governance variables will capture the total effect or the sum of the direct effect and the indirect effect through the controls).''
\end{quote}

But there are some subtle issues here.
If some elements of $Z_r$ are mediators and others are confounders, then both equations will be subject to bias. 
Equation \ref{eqn:lrt2} will be biased due to omission of confounders, while Equation \ref{eqn:lrt1}  will be biased due to inclusion of mediating variables.
Additionally, the claim that the estimates are ``conservative" is only correct if the indirect effect via mediators is of the same sign as the direct (i.e., unmediated) effect. 
If this is not the case, then the relation between the magnitude (and even the sign) of the direct effect and the indirect effect is unclear.

Additionally, this discussion does not allow for the possibility of colliders.
For example, governance plausibly affects leverage choices, while performance is also likely to affect leverage.
If so, ``controlling for" leverage might induce associations between governance and performance even absent a true relation between these variables.\footnote{
Note that \citet{Larcker:2007aa} do not in fact use leverage as a control when performance is a dependent variable.}
For example, \citet{Cadman:2014cr} study, \emph{inter alia}, the effect of being a venture capital (VC)-backed firm on CEO incentive horizons after IPO.
In their analysis they include as controls variables such as \emph{Toptier Underwriter} and \emph{R\&D/Assets}.\footnote{See Table 5 of \citet{Cadman:2014cr}.}
But, given the timing of events, being VC-backed affecting the choice of underwriter is the most plausible causal relation between these two variables.
Also, given that \citet{Cadman:2014cr} include pre-IPO observations, it is plausible that the CEO's incentive horizons would affect variables such as \emph{Toptier Underwriter} and \emph{R\&D/Assets}.
If these assertions are correct,  \emph{Toptier Underwriter} and \emph{R\&D/Assets} are colliders.
While these variables may have little impact on the results of \citet{Cadman:2014cr}, we argue that more discussion about why researchers include controls is warranted.
While the with-and-without-controls approach used by \citet{Larcker:2007aa} has intuitive appeal, a more robust approach would involve careful thinking about the plausible causal relations between the treatment variables, the outcomes of interest, and the candidate control variables.

\section{Quasi-experimental methods in accounting research} \label{sec:quasi}
While most studies in accounting use methods of conditioning on confounding variables in some kind of regression or matching framework, a number of studies use quasi-experimental methods that rely on ``as if" random assignment to identify causal effects \citep{Dunning:2012tt}.
Of the 91 papers in accounting research in 2014 seeking to draw causal inference from observational data, we identify 14 that use quasi-experimental methods for inferences. Despite the low count, we believe that papers using these methods are considered stronger contributions to empirical accounting research and there seems a clear trend toward the use of quasi-experimental approaches.
% Do we want " are pressures from editors and reviewers to incorporate quasi-experimental approaches into research designs."?
In this section, we discuss and evaluate the use of these methods in accounting research.

\subsection{Natural experiments}
Natural experiments occur when observations are assigned by nature (or some other force outside the control of the researcher) to treatment and control groups in a way that is random or ``as if'' random \citep{Dunning:2012tt}. 
Truly (as if) random assignment to treatment and control provides a sound basis for causal inference, enhancing the appeal of natural experiments for social science research.
However, \citet[\,p.3, emphasis added]{Dunning:2012tt} argues that this appeal ``may provoke \emph{conceptual stretching}, in which an attractive label is applied to research designs that only implausibly meet the definitional features of the method.'' 

Our survey of accounting research in 2014 identified five papers that exploited either a ``natural experiment'' or a ``exogenous shock'' to identify causal effects.\footnote{These are \citet{Lo:2013jk,Aier:2014ii,Kirk:2014gx,Houston:2014hv} and \citet{Hail:2014fq}.}
An examination of these papers reveals how difficult it is to find a clear natural experiment in observational data.

First, some studies using exogenous shocks plausibly suffer from issues of confounding, as treatment assignment is non-random.
For example, \citet{Hail:2014fq} study the effect of the ``exogenous shocks" of mandatory IFRS adoption and enforcement of insider trading laws on firms' dividend payments.
While these shocks sort firms into treatment and control groups, they clearly do not do so randomly, as they apply to firms in specific countries and a variety of time-varying country-level effects plausibly exist.\footnote{
According to Table 1 of \citet{Hail:2014fq}, about 60\% of treatment firms are European firms that adopted IFRS in 2005. Figure 1, Panel A of \citet{Hail:2014fq} suggests that most of the impact of IFRS adoption occurs three years after adoption; i.e., for European firms in 2008, when there may have been other reasons for reducing dividends that applied to those firms more than controls.}

Second, because ``exogenous shocks" often do not directly sort firms into treatment and control groups, they rely on assumptions analogous to those required for instrumental variables.
That is, the exogenous shock should not only be random, but should only affect the outcome through its effect on the treatment.\footnote{
In some cases, if the necessary assumptions apply, it would be more appropriate to use instrumental variable methods to estimate causal effects.
In other cases, \citep[e.g.][]{Aier:2014ii}, the treatment of interest is unobserved, making such an approach unfeasible.}
For example, \cite{Aier:2014ii} exploit a 1991 Delaware court ruling as a ``natural experiment'' for the purpose of understanding the causal effect of debtholders' demand for conservatism (the treatment variable) on financial reporting conservatism (the outcome of interest).\footnote{
The court ruling ``expanded the scope of directors' fiduciary duties to include creditors when a Delaware incorporated firm is in the `vicinity of insolvency.'"}
For the court ruling to be a valid instrument for debtholders' demand for conservatism, it must only affect the outcome through its effect on the treatment of interest.
But if the 1991 Delaware court ruling  ruling in question caused directors to dispose of assets leading to recognition of losses and affecting measures of conservatism, as it plausibly did, then the identification strategy is invalid.\footnote{
Similar issues plausibly affect \citet{Houston:2014hv}, which uses the ``exogenous shock" of the 2008 financial crisis and the ``natural experiment" of midterm elections to study ``whether the political connections of listed firms in the United States affect the cost and terms of loan contracts,'' and \citet{Kirk:2014gx}, who ``exploit the natural experiment setting created by the exogenous shock of Reg FD" to examine ``the effect of investments in internal investor relations (IR) departments on firm outcomes."} 
% Random assignment is also dubious here.

The evidence from research published in 2014 suggests that accounting researchers apply the term ``natural experiment" to circumstances where it is not clear that it applies.
While ``exogenous shocks" may provide interesting settings for research, if random assignment does not apply, then researchers should exercise caution in giving causal explanations to associations observed in the data.
Readers should be alert to the fact that terms like ``exogenous shock" and ``natural experiment" are often used when ``as if" random assignment is not plausible.

Extending our survey beyond research published in 2014, we find some papers with credible natural experiments.
One such paper is \citet{Michels:2015aa}, who exploits the difference in disclosure requirements for significant events that occur before financial statements are issued that differ according to whether the event occurs before or after the balance sheet date.
He finds evidence that the market reacts more strongly to recognized events.
But even with such a credible natural experiment, there are concerns that can arise.
As recognized by \citet{Michels:2015aa}, there are possibly different materiality criteria that affect the relation been underlying events and the disclosures he relies on, and \citet{Michels:2015aa} takes care to address this concern.

Another credible natural experiment is examined in \citet[p.\,80]{Li:2015he}, who study the experiment whereby the SEC ``mandated temporary suspension of short-sale price tests for a set of randomly selected pilot stocks."
\citet[p.\,79]{Li:2015he} conjecture ``that managers respond to a positive exogenous shock to short selling pressure \dots by reducing the precision of bad news forecasts."
But if the exogenous shock affects the properties of the forecast (i.e., makes it endogenous), the ``natural experiment" aspect of the research design is undone by the decision to include such properties in the regression analysis.\footnote{\citet{Li:2015he} include the magnitude of the forecast surprise (\textit{MFSURP}) in their regressions (e.g., regressions in Table 2 where abnormal returns around the forecast is the dependent variable).
See \citet[p.\,116]{Imbens:2015aa} for discussion of ``the dangers of using a post-treatment variable \dots as a covariate."}

Our view is that true natural experiments are likely to be rare.
Certainly researchers should exploit these natural experiments when they occur \citep[e.g.][]{Michels:2015aa,Li:2015he}, but care is needed in doing so.
% Are there any good examples of natural experiments in accounting?  The SHO experiment is one, but the one JAR paper on it is very messed up. I could add discussion of this later.
% Add something about what can be learning from "shocks" -- in my email to you:  I think that we need to include something about "shocks" (not necessarily exogenous) -- this is an interesting place to look for effects.  Not really causal, but you can get some insights into forcing a company to change something (we discussed staggered board last week).  Not telling us about the causal impact of staggered board, but perhaps the cost or benefit to the company when they make this decision.

\subsection{Instrumental variables}
\citet[p.114]{Angrist:2008vk} describe instrumental variables (IV) as ``the most powerful weapon in the arsenal of [statistical tools]" in econometrics. 
Accounting researchers have long used instrument variables to address concerns about endogeneity \citep{Larcker:2010fq} and continue to do so.
Our survey of research published in 2014 identifies 10 papers using instrumental variables.\footnote{
These are \citet{Cannon:2014im,Cohen:2014jl,Kim:2014fm,Vermeer:2014bs,Fox:2014io,Guedhami:2013cj,Houston:2014hv,deFranco:2014ct,Erkens:2014hj} and \citet{Correia:2014fp}.}
However, much has been written on the challenges for researchers in using instrumental variables (IV) as the basis for causal inference \citep[e.g.,][]{Roberts:2013cz}. 

\subsubsection{Evaluating IVs requires careful causal (not statistical) reasoning}

With respect to accounting research, \citet{Larcker:2010fq} lament that ``some researchers consider the choice of instrumental variables to be a purely statistical exercise with little real economic foundation'' and call for 
``accounting researchers \dots to be much more rigorous in selecting and justifying their instrumental variables.'' 
\citet[p.117]{Angrist:2008vk} argue that ``good instruments come from a combination of institutional knowledge and ideas about the process determining the variable of interest."
One study that illustrates this is \citet{Angrist:1990dk}.
In that setting, the draft lottery is well understood as random and the process of mapping from the lottery to draft eligibility is well understood.
Furthermore, there are good reasons to believe that the draft lottery does not affect anything else directly except for draft eligibility.

%\footnote {Of course, this seemingly "ideal" instrument has been subject to considerable criticism. I think the idea is that even if you had a low draft number it was not clear that you actually went to the army.  In fact, upper class kids did not go (and the probably had much better skills) than the white hillbilly trash and minorities that ended up going to VN}
%TODO: Get reference for criticism of Angrist instrument

It is evident that many researchers in accounting view causal inference as a purely statistical exercise.
Several papers using IV methods  provide little or no justification for the validity of their instruments.
For example, to address endogeneity \citet{Cohen:2014jl} use ``two instrumental variables. The first is the natural log of industry size, measured as the number of companies within each two-digit SIC. The second measures industry competition using the Herfindahl-Hirschman index, which is well-established as a measure of competitive industries. Our untabulated results using this approach are qualitatively similar to our main analysis, thus indicating that endogeneity is not a concern when assessing the reliability of our findings.''\footnote{
Three other studies used a similar approach.
 \citet{Vermeer:2014bs} ``use Maddala's (1988) two-stage procedure'' in order to ``control for endogeneity'' without providing any explanation at all and in fact seem to be assuming the each of three endogeneous variables can used as an instrument for the other two.
\citet[p.48]{Fox:2014io} state in a footnote that they ``instrumented for the price index employing a two stage least squares estimator'' without further details, simply noting that their ``conclusions are robust with respect to these concerns.''
\citet{Cannon:2014im} uses ``industry-level capacity unit cost and selling price changes'' as instruments for firm-level capacity unit cost changes with no more justification than the fact that these ``are outside management's control.'' But being outside management control does not make a variable an adequate instrument.}

In most remaining cases, the reasoning in support of the validity of an instrument is evidently flawed. \citet{Kim:2014fm} use director age as an instrument for director tenure in a study examining the effect of the latter on firm performance. 
But their arguments to justify this instrument seem instead to provide reasons to believe that it is not valid. 
``Importantly, research finds little or no association between age and performance \dots and a small negative association between age and executive functions \dots. 
Related to directors, Ferris et al. (2003) suggest that any positive effects from director experience increasing with age may be offset by older directors having less energy, posing a last-period risk, and viewing directorships as lucrative part-time jobs for their retirement years.'' 
But these arguments seem to invalidate age as an instrument for tenure. 
For age to be a valid instrument, there should be no unblocked causal path between age and performance except for the path via tenure.
That possible positive effects \emph{may} be offset by negative effects is not a valid basis for claiming age to be a valid instrument.\footnote{
We omit discussion of  \citet{Erkens:2014hj,Houston:2014hv} and \citet{deFranco:2014ct} for reasons of space. But in each case, the instruments have obvious flaws and no convincing arguments for their validity are offered (details available on request).}

\subsubsection{There are no simple (statistical) tests for the validity of instruments}
Although perhaps obvious, the standard statistical tests applied by authors using instrumental variables provide little insight into the quality of the chosen instruments. 
% \citet{Guedhami:2013cj} use $\textit{CAPITAL}$, an indicator for a firm being located in a capital city, as an instrument for political connectivity in a study looking at the effect of political connections on the use of a Big 4 auditor ($\textit{BIG 4}$).
For example, the only justification \citet{Guedhami:2013cj} provide for their instrument is that ``importantly, the correlation between $\textit{CAPITAL}$ and $\textit{BIG 4}$ is small in our data set $(\rho = 0.05)$, helping to justify the validity of this exclusion restriction.''\footnote{
 \citet{Guedhami:2013cj} cite \citet{Larcker:2010fq} as a reference for this approach, even though \citet{Larcker:2010fq} carefully explain why simple tests like this cannot be used to justify instruments.}

%\citet{Correia:2014fp} is relatively thorough. \citet{Correia:2014fp} 
Even more sophisticated tests of weak instruments and tests of over-identifying restrictions are not obviously helpful, as can be demonstrated with a simple simulation exercise.
Suppose that we are interested in a model such as $y = X \beta + \epsilon$, but with $X$ and $\epsilon$ having correlation $\rho(X, \epsilon) > 0$ (i.e., $X$ is endogenous) and $\beta = 0$ (i.e., there is no causal relation between $X$ and $y$). 
Now, suppose we \emph{construct} the following three instruments 
$z_1 = x +\eta_1$, $z_2 = \eta_2$, and $z_3 = \eta_3$, with $\eta_1, \eta_2,  \eta_3 \sim N(0, \sigma_{\eta}^2)$ and independent. 
That is, $z_1$ is $X$ plus noise (e.g., industry averages or lagged values of $X$ would seem to approximate $z_1$), while $z_2$ and $z_3$ are random noise (many variables could be candidates here).\footnote{A paper using instruments with apparently similar properties is \citet{Correia:2014fp}. 
\citet{Correia:2014fp} uses ``average level of political contributions made by the other firms in the same industry'' as an instrument for political contributions by a firm, as well as two additional instruments: ``the percentage of sales made to the government, and the number of years in the previous five years in which there was a close election involving two candidates in the firm's state.''  \citet{Reiss:2007ej} suggest that there is no reason to view industry averages as valid instruments.}
Obviously, these ``instruments" are silly choices and completely inappropriate.

Assuming that $X$ and $\epsilon$ are bivariate-normally distributed with variance of $1$ and $\rho(X, \epsilon)=0.2$ and that $\sigma_{\eta}=0.03$, we run 1000 simulations and  estimate the IV regression using these instruments on the simulated data in each case.
Doing so, we find a mean estimated coefficient on $X$ of $0.201$, which is statistically significant at the 5\% level 100\% of the time.\footnote{Note that this coefficient is close to $\rho(X, \epsilon) = 0.2$, which is to be expected given how the data were generated.} 
Based on a test statistic of 30, which easily exceeds the thresholds suggested by \citet{Stock:2002aa}, the null hypothesis of weak instruments is rejected 100\% of the time. 
The test of overidentifying restrictions fails to reject a null hypothesis of valid instruments (at the 5\% level) 95.7\% of the time.

In other words, it is quite possible for completely spurious instruments to deliver bad inferences, yet easily pass tests for weak instruments and tests of overidentifying restrictions.
%\footnote{It is also common for accounting researchers to claim that they have established the validity of their instruments by implementing some type of Hausman test.  It is very clear that these types of overidentifying tests require the researcher to actually have one valid IV. 
In general, there is no test that enables a researcher to verify that their IVs satisfy the exclusion restriction.

\subsubsection{Causal diagrams and instruments}
To illustrate the application of causal diagrams to the evaluation of instrumental variables, we consider \citet{Armstrong:2013io}.
%
\citet{Armstrong:2013io} studies the effect of shareholder voting ($\textit{Shareholder support}_{t}$) on future executive compensation ($\textit{Compensation}_{t+1}$) .
Because of the plausible existence of unobserved confounding variables that affect both future compensation and shareholder support, a simple regression of $\textit{Compensation}_{t+1}$ on $\textit{Shareholder support}_{t}$ and controls would not allow \citet{Armstrong:2013io} to obtain an unbiased estimate of the causal relation.
Among other analyses, \citet{Armstrong:2013io} use an instrument variable to estimate the causal relation of interest.
\citet{Armstrong:2013io} claim that their instrument is valid based on reasoning that can be expressed as the  Figure \ref{fig:agl}.
By conditioning on $\textit{Compensation}_{t-1}$ and using ISS recommendations as an instrument, \citet{Armstrong:2013io} argue that they can identify a consistent estimate of the causal effect of shareholder voting on $\textit{Compensation}_{t+1}$, even though there is an unobserved confounder, namely determinants of future compensation observed by shareholders, but not the researcher.\footnote{
In Figure \ref{fig:agl}, we depict the unobservability of this variable (to the researcher) by putting it in a dashed box.
Note that we have omitted the controls included by \citet{Armstrong:2013io} for simplicity, though a good causal analysis would consider these carefully.}

However, a critical assumption for this analysis is, as the authors note, that the ``validity of this instrument depends on ISS recommendations not having an influence on future compensation decisions conditional on shareholder support (i.e., firms listen to their shareholders, with ISS having only an indirect impact on corporate policies through its influence on shareholders' voting decisions)" \citep[p.\,912]{Armstrong:2013io}.

In other words, the IV analysis in \citet[p.\,912]{Armstrong:2013io} requires that the causal diagram in Figure \ref{fig:agl} is correct and that there is no arrow from $\textit{ISS recommendation}_t$ to $\textit{Compensation}_{t+1}$. 
Unfortunately, this assumption seems inconsistent with the findings of \citet{Gow:2013aa}, who provide evidence that firms are carefully calibrating compensation plans (i.e., factors that directly affect $\textit{Compensation}_{t+1}$) to comply with the requirements of ISS's policies so as to get a favorable $\textit{ISS recommendation}_t$, implying a path from $\textit{ISS recommendation}_t$ to $\textit{Compensation}_{t+1}$ that does not pass through $\textit{Shareholder support}_{t}$.
Thus this new evidence suggests that the instrument of \citet[p.\,912]{Armstrong:2013io} is not credibly valid for the causal effect they seek to estimate.

% BBKL discussion will go between here ...
\citet{Balakrishnan:2014js} seek to understand the effects of information asymmetry on disclosure practice of firms, and in turn, the effect of disclosure on information asymmetry.\footnote{
\citet{Balakrishnan:2014js} does not appear in an accounting journal, but we examine it because it examines a topic of interest to accounting researchers (i.e., disclosure) and because the reasoning for the identification strategy is explicit and detailed.}
While changes in analyst coverage are generally not random and may be correlated with information asymmetry due to omitted correlated variables, 
\citet{Kelly:2012ih} treat closure of brokerage firms as an exogenous (i.e., ``as if" random) source of variation in analyst coverage, which they argue will affect liquidity only through the its effect on information asymmetry.
 
The identifying assumption that $\textit{Brokerage closure}_t$ affects $\textit{Information asymmetry}_t$ (via its effect on $\textit{Analyst coverage}_t$) but otherwise has no effect on $Disclosure$ or $Liquidity_{t+1}$. \citep{Balakrishnan:2014js} or $Liquidity_t$ or other outcomes .\footnote{
Note that \citet{Kelly:2012ih} do not estimate an instrumental variable regression, and instead treat ``coverage shocks" as an exogenous event. 
However, as it is changes in coverage driven by brokerage closures that they claim to be exogenous, Figure \ref{fig:kl} seems to represent the logic of  \citet{Kelly:2012ih}.}
They then examine the effect of such changes in analyst coverage on information asymmetry.

\citet{Balakrishnan:2014js} seek to build on the identification strategy of  \citet{Kelly:2012ih} so  as to estimate the causal effect of disclosure on liquidity.
The critical identifying assumptions are, as \citet{Balakrishnan:2014js}  note, that ``lagged coverage shocks a) lead to more disclosure, b) not affect liquidity directly, and c) not correlate with some omitted variable that in turn affects liquidity."
These links can be expressed by augmenting the causal graph in Figure \ref{fig:kl} to add a node for \textit{Disclosure}, driven by $\textit{Information asymmetry}_t$, which is linked to a new node for $\textit{Liquidity}_{t+1}$. 
The result of these additions is the causal diagram in  Figure \ref{fig:bbkl}.\footnote{
 Again \citet{Balakrishnan:2014js} view ``coverage shocks" as exogenous and use these shock as instruments. However, it is the shift in analyst coverage driven by brokerage closures that is expected to be exogenous.
The thrust of the analysis here is essentially the same if it is the ``coverage shocks" that are treated as exogenous.}
Again, if Figure \ref{fig:bbkl} correctly represents the relevant causal relations, then \citet{Balakrishnan:2014js} have a valid instrument.

However, there are reasons to believe that Figure \ref{fig:bbkl} does not represent the relevant causal relations.
First, it seems likely that coverage shocks are be persistent. \citet{Balakrishnan:2014js} seem to  assume as much when they argue that ``firms [unable to respond  to the coverage shock through increased disclosure] suffer a \emph{permanent} reduction in liquidity" (p.\,2239), which presumably requires the shocks to analyst coverage to be persistent.
We represent this persistence in Figure \ref{fig:bbkl_alt} by adding a node for $\textit{Analyst coverage}_{t+1}$ and an arrow into this node from $\textit{Analyst coverage}_t$.

But we know from the main result in \citet{Kelly:2012ih} that analyst coverage affects contemporaneous information asymmetry, so we add a node for \textit{Information asymmetry}$_{t+1}$ and, similar to the relation in $t$, add an arrow from $\textit{Analyst coverage}_{t+1}$ into this node.
Finally, it is well known that information asymmetry is a major driver of contemporaneous liquidity \citep{Glosten:1988gd}, which implies an arrow from \emph{Information asymmetry}$_{t+1}$ into $\textit{Liquidity}_{t+1}$.

The end result of these additions is the causal diagram depicted in Figure \ref{fig:bbkl_alt}.
But from this diagram, it is clear that $\textit{Analyst coverage}_t$ affects $\textit{Liquidity}_{t+1}$ through a channel other than disclosure, implying that the instrument of \citet{Balakrishnan:2014js} is not credibly valid for the causal effect they seek to estimate. 
The point of the discussion above is not to impugn the precise findings of \citet{Balakrishnan:2014js}, but rather to illustrate the merit of more careful analysis of a paper's identification strategy and, we argue, the value of causal diagrams in doing so.
% ... and here. Don't edit between these two lines.

\subsubsection{IV in accounting research: An evaluation}
A review of IV in published research in accounting in 2014 suggests that researchers have paid little heed to the suggestions and warnings of  \citet{Larcker:2010fq} and \citet{Roberts:2013cz}.
We find no case of an instrumental variable analysis in our survey of accounting research that can withstand rigorous scrutiny.
This is perhaps not unsurprising, as plausible instruments have tended to rely on some explicit randomization, which is likely to be extremely rare in accounting research settings.
While IV is a classic textbook approach for credible causal inference, its applicability in actual research settings seems very limited and it seems unlikely that IV will provide a sound basis for causal inference in accounting research for the vast majority of research questions.
 
 % \citet{Houston:2014hv} use variables variables that are related to the location of the company's headquarters as instruments for political connection and argue that ``these instruments should not be conceptually related to loan spreads. The key insight here is that the geographic locations of headquarters for companies are predetermined and are unlikely to affect banks' financing decision on loan costs. In summary, our identification assumption is that the costs of bank loans are not directly related to the companies' geographic locations, after controlling for a series of firm and loan characteristics'' (p.228). In justifying the relevance of the instrument, the authors seem eager to justify a connection, suggesting that ``the presumption is that the company's geographic location affects the company's ability to attract politically connected directors.'' But it far from clear why a company's geographic location would not also affect the its ability to attract directors with connections to \emph{financial institutions}, which plausibly affects financing terms directly \citep{Guner:2008tp}.\footnote{\citet{Houston:2014hv} also use firm age as an instrument, arguing that ``firm age affects a firm's incentive and capability in building up political connections''; but it is not clear why firm age would not also affect a firm's ``incentive and capability in building up'' financial connections.} 
 
% Researchers tend to very unclear about the determinants of their selected instruments.  In many cases, it seems that the instruments are also endogenous.  This makes it very difficult to to rule out the possibility that the instrument directly affects (or is correlated with) variables other than the endogenous variable of interest.
 % \citet{Erkens:2014hj} ``use the following three instrumental variables that capture the extent to which lenders are more likely to serve on a firm's board, which is studied for its potential effect on accounting conservatism. We use \emph{Industry importance to primary lender} because industry specialization increases the importance of acquiring information about a firm's industry, \emph{Primary lender within 50 mile radius} because physical proximity to lenders' headquarters reduces the cost of serving on the board, and \emph{Number of commercial banks within 50 mile radius} because the close proximity of multiple banks increases competition for board seats from other lenders.'' If industry specialization affects information-acquisition incentives, it seems it would do so through channels outside of board membership. With respect to the second instrument, it's quite likely that proximity affects information-gathering independent of service on the board. With respect to the third instrument, it is also implausible that the only direct effect of this variable is one on the service of bankers on the board (for example, this may lead to lower search costs in choosing potential lenders).

 %\citet{deFranco:2014ct} ``find that the number of covenants is positively related to the interest rate, likely due to endogeneity between the interest rate and covenants.'' To address this using they use ``the number of covenants by calendar year indicators as the instrument'' for the number of covenants. Apart from the issues with using an average as an instrument discussed in \citet{Reiss:2007ej}, the authors justify their instrument by suggesting that ``the strictness of covenant packages significantly deteriorated during the years of the credit boom that preceded the financial crisis.'' But it seems likely that the credit boom would have a direct effect on interest rates on bond issues. 

\subsection{Regression discontinuity designs}
In discussing the recent ``flurry of research" using regression discontinuity (RD) designs, \citet[p.\,282]{Lee:2010hya} point out that they ``require seemingly mild assumptions compared to those needed for other nonexperimental approaches \dots and that causal inferences from RD designs are potentially more credible than those from typical `natural experiment' strategies."
% I would add a reference to the "first application" -- Thistlewaite, D.; Campbell, D. (1960). "Regression-Discontinuity Analysis: An alternative to the ex post facto experiment". Journal of Educational Psychology 51 (6): 309–317

%Imbens, G.; Lemieux, T. (2008). "Regression Discontinuity Designs: A Guide to Practice". Journal of Econometrics 142 (2): 615–635. 
Recently, RD designs have attracted the interest of accounting researchers, as a number of phenomena of interest to accounting researchers involve discontinuities. For example, whether an executive compensation plan is approved is a discontinuous function of shareholder support \citet{Armstrong:2013io} and whether a firm had to comply with provisions of Sarbanes-Oxley Act in 2004 \citep{Iliev:2010ic} is a discontinuous function of market float.

While RD designs make relatively mild assumptions, in practice these assumptions may be violated.
In particular, manipulation of the running variable (or the variable that determines whether an observation is assigned to a treatment)  may occur.
%TODO: Add discussion of Listokin, McCrary, etc.  
Another issue with RD designs is that the causal effect estimated is a local estimate (i.e., it relates to observations close to the discontinuity.
This effect may be very different from the effect at points away from the discontinuity.
For example, in designating a public float of \$75 million, the SEC may have reasoned that at that point the benefits of Sarbanes-Oxley, which may have been increasing in firm size, were approximately equal to the approximately fixed costs of complying with the law.
If true, we would expect to see an estimate of approximately zero effect, even if there benefits of the law for shareholders of firms well about is positive.
Similarly, a vote that receives approximately 50\% support may reflect the fact that costs and benefits are approximately balanced, while measures that receive much greater support may have very different levels of benefits.

It is also important to note that so-called ``quasi-RD" designs have only a superficial resemblance to RD designs.
For example, as he cannot observe ``the specific covenant thresholds in [his] primary dataset," \citet{Tan:2013ce} is constrained to estimate a ``quasi-RD" design like that estimated in \citet{Roberts:2009ka}.
But this ``regression discontinuity design" is essentially ordinary-least squares with an indicator for covenant violation and thus does not represent a method for estimating unbiased causal effects with observational data.
%TODO: Add discussion of Ertimur, Ferri and Oesch (2014).

%TODO: Seems like we want to include in this section:  plot the data and if you can't see it in the data, it probably is not actually there (Imbens), do not use the high level polynomial approach, and other similar issues.  Probably something on whether the magnitude of the results is actually believable -- Yonca's, the prize winning JF paper, and others results seem implausibly large for governance topics.

\subsection{Quasi-experimental methods: An evaluation}
We agree that the revolution in econometric methods for causal inference has been an exciting development.
However, we have serious concerns regarding the value of these methods in accounting research. 
First, it is evident that accounting researchers often apply these methods poorly and inappropriately.
Second, it is far from clear that these methods, properly applied, can support more than a small fraction of accounting research.

\section{Mechanisms and causal inference} \label{sec:mech}

\begin{quotation}
\begin{singlespace} 
	\addtolength{\leftmargin}{.25in}
	\addtolength{\rightmargin}{.25in}
In complex fields like the social sciences and epidemiology, there are only few (if any) real life situations where we can make enough compelling assumptions that would lead to identification of causal effects.
\attrib{Judea Pearl, cited in \citealt[p.\,287]{Freedman:2004ix}}
\end{singlespace}
\end{quotation}

In the first half of the paper, we have argued that, while causal inference is the goal of most accounting research using observational data, research designs that yield output that can be viewed as unbiased estimates of causal effects using such data likely do not exist most research settings.
So, what should researchers do? Do we stop doing research? Do we need to give up on causal inference? 
We believe that this is too pessimistic and that there are viable paths forward that do not rely on researchers identifying ``clever" identification strategies to answer questions of interest.
The objective of the second part of this paper is to discuss these paths forward.
The first path we discuss is an increased focus on causal mechanisms.
%TODO: Define the term "mechanism"

\subsection{Causal mechanisms: Some examples}
Accounting research is not alone in relying primarily on observational data.
Other fields also seek to draw causal inferences, but need to grapple with the reality of observational data. 
Yet in many cases, these fields have successfully drawn causal inferences.
In the following, we briefly discuss case studies of plausible causal inference in other fields and highlight features that enhanced the credibility of inference.

\subsubsection{John Snow and cholera}
A widely cited case of causal inference involves John Snow's work on cholera.
As there are many excellent accounts of Snow's work, we will focus on the barest details.
As discussed in  \citet[p.\,339]{Freedman:2009ur}
``John Snow was a physician in Victorian London.
 In 1854, he demonstrated that cholera was an infectious disease, which could be prevented by cleaning up the water supply. 
The demonstration took advantage of a natural experiment.
 A large area of London was served by two water companies. 
 The Southwark and Vauxhall company distributed contaminated water, and households served by it had a death rate`between eight and nine times as great as in the houses supplied by the Lambeth company, ' which supplied relatively pure water."

But there was much more to Snow's work than the use of a convenient natural experiment.
First, Snow's reasoning (much of which was surely done before ``the arduous task of data collection" began) was about the  mechanism through which cholera spread. Existing theory suggested ``odors generated by decaying organic material."
Snow reasoned qualitatively that such a mechanism was implausible.
Instead, drawing on his medical knowledge and the facts at hand, Snow conjectured that ``A living organism enters the body, as a contaminant of water or food, multiplies in the body, and creates the symptoms of the disease. Many copies of the organism are expelled with the dejecta, contaminate water or food, then infect other victims" \citep[p.\,342]{Freedman:2009ur}.
With a hypothesis at hand, Snow then needed to collect data to prove it.
His data collection involved a house-to-house survey in the area surrounding the Broad Street pump operated by  Southwark and Vauxhall.
As part of his data collection, Snow needed to account for anomalous cases (such as the brewery workers who drank beer, not water).
It is important to note that this qualitative reasoning and diligent data collection were critical elements establishing (to a modern reader) the ``as if" random nature of the treatment assignment mechanism provided by the Broad Street pump.
This contrasts with the speculative guesses often used to justify natural experiments by modern researchers.

But another important feature of the case of John Snow and cholera is that widespread acceptance of Snow's hypothesis did not occur until compelling evidence of the mechanism was provided.
``However, widespread acceptance was achieved only when Robert Koch isolated the causal agent (\emph{Vibrio cholerae}, a comma-shaped bacillus) during the Indian epidemic of 1883"  \citep[p.\,342]{Freedman:2009ur}.
Only once persuasive evidence of a plausible mechanism was provided (i.e., direct observation of microorganisms now known to cause the disease) did Snow's ideas become widely accepted.

\subsubsection{Smoking and heart disease}
A more recent illustration of plausible causal inference is discussed by \citet{Gillies2011-GILTRT-3}.
\citet{Gillies2011-GILTRT-3} points discusses the paper by \citet{Doll:1976aa}, which studies the mortality rates of male doctors between 1951 and 1971.
The data that \citet{Doll:1976aa} had showed ``a striking correlation between smoking and lung cancer" \citep[p.\,111]{Gillies2011-GILTRT-3}.
\citet{Gillies2011-GILTRT-3} argues that ``this correlation was accepted at the time by most researchers (if not quite all!) as establishing a causal link between smoking and lung cancer. Indeed Doll and Peto themselves say explicitly (p.\,1535) that the excess mortality from cancer of the lung in cigarette smokers is caused by cigarette smoking."
In contrast, while \citet{Doll:1976aa} also had highly statistically significant evidence of an association between smoking and heart disease, they were cautious about drawing inferences of a direct causal explanation for the association.
\citet[p.\,1528]{Doll:1976aa} say ``To say that these conditions were related to smoking does not necessarily imply that smoking caused \dots them. The relation may have been secondary in that smoking was associated with some other factor, such as alcohol consumption or a feature of the personality, that caused the disease.''
 
\citet{Gillies2011-GILTRT-3} then discusses extensive research into atherosclerosis between 1979 and 1989 and concludes that ``by the end of the 1980s, it was established that the oxidation of LDL was an important step in the process which led to atherosclerotic plaques."
Later research established evidence of much higher levels of a new measure (levels of $F_2$-isoprostanes in blood samples) of the relevant oxidation in the body ``provides compelling evidence that smoking causes oxidative modification of biologic components in humans. 
This conclusion is greatly strengthened by the finding that levels of $F_2$-isoprostanes in the smokers fell significantly after two weeks of abstinence from smoking" \citep[pp.\,1201--2]{Morrow:1995gz}.  
\citet[p.\,120]{Gillies2011-GILTRT-3} points out that this evidence did not establish a confirmed mechanism linking smoking with heart disease, because the required oxidation needs to occur in the artery wall, not in the blood stream. 
But later research established a link: ``Smoking produced oxidative stress. 
This increased the adhesion of leukocytes to the \dots artery, which in turn accelerated the formation of atherosclerotic plaques" \citep[p.\,123]{Gillies2011-GILTRT-3}.
Thus, a causal link or mechanism between smoking and atherosclerosis was established.
%TODO: Get an example from the social sciences, e.g., political science.

\subsubsection{Implications of cases on mechanism}
 \citet{Gillies2011-GILTRT-3} avers that the process by which a causal link between smoking and atherosclerosis was established illustrates the ``Russo-Williamson thesis."
 \citet[p.\,159]{Russo:2007iz} suggest that ``mechanisms allow us to generalize a causal relation: while an appropriate dependence in the sample data can warrant a causal claim `$C$ causes $E$ in the sample population,' a plausible mechanism or theoretical connection is required to warrant the more general claim `$C$ causes $E$.' Conversely, mechanisms also impose negative constraints: if there is no plausible mechanism from $C$ to $E$, then any correlation is likely to be spurious. Thus mechanisms can be used to differentiate between causal models that are underdetermined by probabilistic evidence alone."

The Russo-Williamson thesis was arguably also at work in the case of Snow and cholera, where the establishment of a mechanism (\emph{Vibrio cholerae}) was essential before the causal explanation offered by Snow was widely accepted and also in the case of smoking and lung cancer, which was initially conjectured based on associational evidence, but was widely accepted by 1976.\footnote{
The persuasive force of Snow's natural experiment, coming decades before the work of Neyman and Fisher, might be considered greater today.}

Our view is that accounting researchers can learn from fields such as epidemiology and political science. 
These fields grapple with the reality of observational data.
While randomized controlled trials are a gold standard of sorts in epidemiology, in many cases it is unfeasible or unethical to use such trials.
And in political science, it is not possible to randomly assign countries to treatment conditions such as \emph{democracy} or \emph{socialism}.
Yet these fields have often been able to draw plausible causal inferences by establishing clear mechanisms, or causal pathways, from putative causes to putative effects.
%TODO: Get better examples from fields other than epidemiology.

We argue that the value of the study of mechanisms is perhaps underestimated, perhaps due to a belief that the path to credible causal inferences involves clever identification strategies.

One paper that has a fairly compelling identification strategy is \citet{Brown:2015ik}, which examines ``the influence of mobile communication on local information flow and local investor activity using the enforcement of state-wide distracted driving restrictions."
The authors find that ``these restrictions \dots inhibit local information flow and \dots the market activity of stocks headquartered in enforcement states."
\citet[p.\,9]{Miller:2015ec} suggest that ``given the authors' setting and research design, it is difficult to imagine a story under which the types of reverse causality or correlated omitted variables explanations that we normally worry about in disclosure research are at play."\footnote{
One issue with the study is the fact the authors do not adjust their standard errors for the well-known cross-sectional dependence in the dependent variable they focus on, trading volume, in addition to the time-series dependence they do adjust for.}
However, notwithstanding the apparent robustness of the research design, it seems that the results would be more compelling with detailed evidence of a causal mechanism through which the estimated  effect occurs.
For example, if evidence were provided of trading activity by local investors while driving prior to, but not after, the implementation of distracted driving restrictions, this would seem to quite persuasive even incremental to a compelling identification strategy.\footnote{
Note that the authors disclaim reliance on trading while driving: ``our conjectures do not depend on the presumption that local investors are driving when they execute stock trades \dots [as] we expect such behavior to be uncommon."
However, even if not \emph{necessary}, given the small effect size documented in the paper (approximately 1\% decrease in volume), a small amount of such activity could be \emph{sufficient} to account for their results.}

As another example, many published papers have suggested that managers adopt conditional conservatism as a reporting strategy to obtain benefits such as reduced debt costs \citep{Ahmed:2002aa,Zhang:2008bc}.
But, as \citet[p\,317]{Beyer:2010cj} point out, an ex ante commitment to such a reporting strategy ``requires a mechanism that allows managers to credibly commit to withholding good news or to commit to an accounting information system that implements a higher degree of verification for gains than for losses," yet research has only recently begun to focus on the mechanisms through which such commitments are made \citep[e.g.,][]{Erkens:2014hj}.

Better understanding of mechanisms also allows researchers to identify gaps in research based on archival data and verbal theorizing.
\cite{Soltes:2014gr} provides an insightful discussion of the pitfalls from exclusive reliance on archival data. 

%If a plausible causal mechanism for their empirical research question (using theory and/or institutional observation), the next step in the research process is to assess the ability of this mechanism to explain real-world data. 
% As we have argued above, the traditional quasi-experimental methods used by accounting researchers have a number of serious limitations for conducting this type of data analysis exercise. 
% As an alternative, structural modeling methods that are becoming popular in economics (especially industrial organization work) and marketing represent an approach that might be used to provide an improved understanding of causal mechanisms for many accounting research questions.

% Structural material here
\section{Structural Modeling}
Structural empirical models consist of two main elements:
% One alternative approach to empirical research in economics is labeled the structural approach (Wolpin 2013).
% The structural approach ``requires that a researcher explicitly specify a model of economic behavior, that is, a theory." (Wolpin, 2013, p.2).

\begin{enumerate}
\item a theoretical (economic) model of the phenomena
of interest; and,
\item a stochastic model that links the theoretical model to the observed
data.
\end{enumerate}
The theoretical model minimally describes who makes decisions, the objectives of decisions makers,
and constraints on decisions and behavior. 
In developing and analyzing the theoretical model, the researcher decides what conditions (variables) matter, 
what is endogenous and exogenous, and what conditions impact behavior. 
The goal in formulating the theoretical model is to derive a set of equations or inequalities that describe the determinants of decisions. 
Not all these determinants may be observed by the researcher.
The stochastic component of the structural model accounts for determinants the researcher does not observe, as well as why the theoretical model may not perfectly explain the data. 
Additionally, the unobservables may also rationalize why the process that generated the data differs from that envisioned by the theoretical model.

Empiricists employ structural models for several nonexclusive reasons. 
First, structural models are a means by which a researcher can understand what determines behavior and market outcomes. 
Second, structural models make it clear what data are needed to identify unobserved theoretical quantities or objects, such as an individual's degree of risk aversion. 
Third, structural models provide a foundation for estimation and
inference. 
Finally, structural models facilitate counterfactual analyses.
Counterfactual analyses predict what might happen under conditions not observed in the data. 
For example, they might show how accounting reports would change if a new accounting standards were adopted. 
 
We now explore these benefits of structural models in more detail, as well as discuss some of the costs that arise in formulating and estimating structural models.

\subsection{The Structural Approach}

The term structural model originated with economists and statisticians working at
the Cowles Foundation in the 1940s and 1950s.
The earliest structural models showed how price and quantity data could be used to recover unobserved demand and supply curves. 
This literature introduced the issue of identification, and more specifically the idea that (instrumental) variables beyond price and quantity were needed to identify demand and supply schedules. 
The impact of these early models on empirical work in economics encouraged and other social scientists to begin using theoretical models to help interpret data.
Indeed, structural models have been used to study: educational choices, voting, contraception, addiction, and financing decisions. 
More modern definitions and applications of structural models can be found in Reiss and Wolak (2007) and Reiss (2011). 
%TODO: Add references.

Generically, structural empirical models consist of equations or inequalities that describe the optimizing behavior of individuals or organizations.
These equations come from decision-making models that describe what factors (variables) impact agents' decisions.
Let $y^*$ denote the endogenous choices of agents. 
The researcher may observe these choices, or some transformation of them.
Let $y=y(y^*)$ denote the choices the researcher observes. 
The exogenous factors affecting decisions observed by the researcher are denoted by the vector $x$. 
Unobserved factors generally are divided into two types: those that vary across sample observations, $\xi$, and those that are constant across observations, $\theta$. 
The $\theta$s are referred to as parameters.

Mathematically, the theoretical model delivers either equalities (``structural equations")
$$  y = g(y, x,\xi ,\theta)$$ 
or inequalities, e.g.,
$$  y \le g(y, x,\xi ,\theta)$$
that relate these quantities. 
In most structural models, the functional form of $g(\cdot)$ is known and is a consequence of specific functional forms used in the theoretical model. 
Usually the ultimate goal of the structural model is to show how the unknown parameters $\theta$ can be recovered from data on $y$ and $x$. 

To illustrate these ideas, under specific economic and behavioral
assumptions, the behavior of demanders and suppliers can be described by a demand and supply system of the form:
$$\begin{array}{lcl}
q^D & = & \theta_1 + \theta_2 \, p + \theta_3 \, x_1 \\[.5em]
q^S & = & \theta_4 + \theta_5\, p + \theta_6 \, x_2 \\[.5em]
q^D  & = & q^S 
\end{array}
$$
Here $y^*$ equals: quantity demanded, $q^D$;  quantity supplied, 
$q^S$; and price $p$.
The exogenous variables in $x$ include demand and cost shifters such as consumer incomes ($x_1$) and suppliers' input prices ($x_2$).
It is important to notice that this structural model consists of three equations. The first two equations are a consequence of the optimizing behavior by individual demanders and suppliers. 
The third condition is an equilibrium condition (i.e., ``demand equals supply"), one that is imposed by the modeler to close the model. 
That is, map the unobserved quantities in $y^*$ to an observed $y$, which is simply ``quantity" ($q$) and price.

In principle these theoretical relations can be taken directly to data.
The problem with taking them directly to data is that the data will prove them flawed. 
For example, in the above demand and supply system the
only object besides $x$ that explain variation in $y$ is the vector $\theta$. 
With many observations on prices and quantities, it would be highly unusual for the six unknown parameters to explain the variation in price and quantity not rationalized by $x$.
For this reason, the researcher typically adds unobservables, such as $\xi$ and $\epsilon$, whose variation across sample observations (along with the other variables) is capable of rationalizing 
all values of $y$.
It is important to realize that these unobservables need not be motivated by the model or related to agent behavior.\footnote{Measurement errors are but one example.}

With the addition of an $\epsilon$, the above demand and supply model becomes 
$$\begin{array}{lcl}
q & = & \theta_1 + \theta_2 \, p + \theta_3 \, x_1 + \epsilon_1 \\[.5em]
q & = & \theta_4 + \theta_5\, p + \theta_6 \, x_2 + \epsilon_2\\
\end{array}
$$
The critical challenge for the researcher at this point is to explain how the observational data on $x$ and $y$ identify $\theta$. 
It is important to realize that just because the structural model contains distinct parameters or quantities, this does not mean that they can be unambiguously estimated. 
When they can, the quantities are ``identified." 
For example, $y$ might be CEO compensation and one
element of $\theta$ might be a coefficient representing a corporate CEO's degree of relative risk aversion.
In order to recover the degree of relative risk aversion from data, the researcher must be prepared to show how risk aversion is recovered from variations in CEO compensation and the other variables.

\subsection{Are Structural Models Relevant to Accounting Research? A Misstatement Illustration}

This high-level discussion of structural modeling may not seem directly relevant to accounting
research. 
Indeed, in general there is a divide between theoretical and empirical research in accounting.
While there are exceptions, few theoretical accounting researchers explain how to map the specifics of their models to data. Similarly, few empirical researchers have attempted to take theoretical models directly to data.
Does this mean that structural models are not applicable to accounting questions? Or, is there unrealized potential? Our view is that structural models have more of a role to play.
To be clear, we do not believe that all accounting researchers need  estimate structural models. 
Indeed, no structural modeling exercise should go forward unless the researcher is highly convinced that the benefits of a structural model would outweigh the substantial costs entailed in developing and estimating a structural model. 

While we have already discussed some of the advantages of structural models, it is 
important to keep in mind the challenges researchers face in developing structural models. 
To begin, structural models can be both technically demanding and time-intensive to develop. 
Additionally, when constructing a theoretical model that can be taken the data, the
empirical researcher will typically be forced to make simplifications that a pure theorist would never make.
Similarly, in order to have an empirical model that is linked to a theory, the empiricist may have to live with assumptions that other empiricists criticize as
unrealistic. 
Finally, just because a researcher can write down a theoretical model and estimate it does not make the empirical model ``right."  There is no guarantee that the causal connections contemplated are correct. 
Further, there is no guarantee that after all the effort that went into developing the model, that the estimates will make sense or that the model will otherwise be validated. 
Despite these challenges, we believe that the benefits of structural models can outweigh their costs.
The purpose of this section is to illustrate how a structural model can deliver accounting insights. 

To illustrate how one might go about developing a structural model for accounting data, we turn to the topic of accounting misstatements. 
There is now an extensive accounting literature exploring why misstatements are made and how easy they are to detect.
This literature is of importance to investors, managers and boards.
Many predictive or explanatory model of misstatements regress indicators for misstatements on a host of accounting, firm and market variables. 
These variables include measures of the complexity of the accounting statements, accrual quality, off-balance sheet activities, firm performance, firm and auditor characteristics, and manager compensation variables. 

If we take this literature as starting point, empirically we would like to have a model that explains why misstatements occur. 
The first problem a structural model must confront is the empirical reality that we do not observe misstatements, but typically only restatements. 
Restatements occur after the firm and its auditor agree on how to report results, and as such are triggered by investigations by third parties or new 
audits.
For simplicity we shall refer to these collectively as ``investigations." 
Thus, any inferences about misstatements have to be seen through the lens of what is detected, or restatements.
If investigations are perfect and detect all misstatements that initially get past the auditor, then there is a one-to-one correspondence of misstatements and restatements. 
Realistically, investigations are not perfect, meaning that we will have to recognize the difference between the two in the structural model.

\subsection{A First Model: Nonstrategic Auditing}

To simplify the initial model, we imagine that misstatements  are
deliberate. 
Though this is clearly a strong assumption, we make it here because it 
allows us to deliver clear predictions about the unobserved rate of misstatements.\footnote{
While this strong assumption can be seen as a weakness of the model, it also can be
seen as a strength. 
If we, or others, do not like the end model, we know what assumption(s)
to revisit.} 
We also simplify matters by assuming that a single agent, the `CEO', is responsible for deciding whether or not to misstate results.
The CEO is assumed to be rational in the sense that they trade off the expected benefits and costs of misstatements when deciding whether to misstate.

Suppose that the CEO receives a benefit of $B^*$ from the \emph{successful} manipulation of earnings, 
i.e., from a misstatement that is not detected by the firm's auditors or subsequent investigations. 
Manipulations are successful only if they are not caught initially by the firm's auditors before a report is released and if they are not caught subsequently during further scrutiny.
We assume the firm's auditors independently catch misstatements at a constant rate $p_A$ and that the (conditional) probability of subsequent investigations catching a misstatement is $p_I$.
Given these assumptions, the probability of misstatement getting past the firm's 
auditor and subsequent investigation is $(1-p_A) \times (1 - p_I)$.
The CEO's benefit from a successful misstatement is then
$$ B^* = (1-p_I) \times (1-p_A) \times B$$
where $B$ is a gross benefit to the manager of a misstatement. 

To misstate performance, the CEO must exert costly effort, which is a fixed $C_M$. 
Putting together the manager's benefits of misstating with their costs gives
\begin{equation}\label{bencost}
y_M^* = \begin{cases}\mbox{Misstate} & \mbox{if }\, (1-p_I) \times (1-p_A) \times B - C_M \ge 0\\
\mbox{Don't Misstate} & \mbox{Otherwise}.\end{cases}.\end{equation}
This (structural) inequality describes the unobserved misstatement process. 
In general, researchers will not observe $B$ or $C_M$. 
In some instances, they may not observe $p_A$ or $p_I$.

To complete the structural model, the researcher must relate these objects to the observed data.
Because we observe a zero-one indicator variable for restatements $y$ and not misstatements, we need to link the two. 
In this model, restatements are the result of three stochastic processes:

\begin{enumerate}
\item The manager decides to misstate (or not).
\item The firm auditor randomly audits and detects (or not) .
\item A post-report investigation happens and detects (or not).
\end{enumerate}

Mathematically, this sequence can be modeled as
\begin{equation}\label{restateeqn}
 y = I(\mbox{Restate}) = I(y^*_M \ge 0) \, \times\, (1 - I(y^*_A \ge 0)) \, \times\, I(y^*_I \ge 0)
\end{equation}
where $I(\cdot)$ is a zero-one indicator function equaling one when the condition in parentheses is true.
The unobserved variables $y^*_A$ and $y^*_I$ respectively reflect the likelihood that the firm's
auditor and the investigation process detect a misstatement. Notice equation (\ref{restateeqn})
uses $(1 - I(y^*_A \ge 0))$, the indicator that the firm's auditor misses the misstatement.

Equation (\ref{restateeqn}) somewhat resembles a traditional binary discrete choice model. The easiest
way to see this is to take expectations (from the researcher's standpoint)
\begin{equation} \label{equilpr}
\begin{array}{lcl}
 E(y) & = & E\, \left[\; I(y^*_M \ge 0) \, \times\, (1 - I(y^*_A \ge 0)) \, \times\, I(y^*_I \ge 0) \; \right]\\[1em]
 & = &  \mbox{Pr(Misstate)} \times \mbox{Pr(Auditor Misses)} \times
\mbox{Pr(Investigation Finds)}\\[1em]
& = & \beta^* \times (1-p_A) \times p_{I} = \mbox{Pr(Restate)}
\end{array}\end{equation}
From equation (\ref{bencost}), $\beta^*$ is the (researcher's) forecasted probability that a misstatement occurs, or
\begin{equation}\label{betaplus}
\beta^*= \mbox{Pr}\left(\, (1 - p_A)(1 - p_I) B - C_M \ge 0 \,\right)
\end{equation}

At this point, the theory has delivered a structure for relating the \emph{unobserved} probability of a misstatement, $\beta^*$, to the potentially estimable probability of a restatement.
Now we face a familiar structural modeling problem, which is that the equations delivered by theory do not necessarily anticipate all the reasons why
in practice why CEO behavior, auditor behavior or outside scrutiny might vary across accounting reports.
For example, the theory so far does not point to different reasons why CEO's might differ in their cost-benefit analyses of misstatements. 
To move theoretical relations closer to the data, researchers typically introduce observable reasons into them. Often there is some ad hoc or ``nonstructural" element to these
additions. 
Empiricists are willing to do this, however, because they believe that it is important to account for heterogeneity they believe the theory does not explicitly recognize.

To illustrate this approach, here we assume that CEO's unobserved costs and benefits do vary systematically with observables.
In addition, because these observables do not perfectly represent the observed and unobserved benefits, it is important to allow for unobservable differences in the costs and benefits of misstatements. 
One specification that does this is to assume
\begin{equation}\begin{array}{lcl}\label{eqns1}
B & = & b_0 + b_1 \, \mbox{BONUS} + X_B\beta\\[.5em]
C_M & = & m_0 + m_1 \, \mbox{SALARY} + X_C\gamma + \xi\\[.5em]
\end{array}
\end{equation}
where BONUS is the fraction of a CEO's total pay that is stock-based compensation, 
the $X_B$ are other observable factors that impact the manager's benefits from misstatements,
SALARY is the CEO's annual base salary, and the $X_C$ are observable factors impacting the CEO's perceived costs of misstatements.

Why this linear specification and why these variables? 
We have no strong theoretical reason for the linear assumption. 
Instead, its motivation is practical.
We shall shortly see that it facilitates estimation of the model unknowns.
As for the variables in the benefit and cost specifications, here we rely in part on theoretical and empirical observations in the academic literature on misstatements. 
The BONUS variable is included in benefits because it is thought to capture a classic moral hazard problem: the more CEO are rewarded for performance, the
greater their incentive to misstate results so as to increase (perceived) 
performance.
%TODO: Get refs
Thus, we would expect the unknown coefficient $b_1$ to be positive.
Similarly, the SALARY variable is included in costs because previous studies have hypothesized that the higher a CEO's guaranteed base pay, the more the CEO perceives ex ante that it is risky to make misstatements.
Thus, we would expect the unknown coefficient $m_1$ also to be positive. For now, we leave the other $X$ variables unnamed.

One key variable in the above model is the unobserved cost $\xi$.
While it makes sense to say that the researcher cannot measure all misstatement
costs, why not also allow for unobserved benefits as well.
The answer here is that adding an unobserved benefit would not really add to the model as it is the net difference that the model is trying to capture.\footnote{
The sense in which it could matter is if we thought we observed the probabilities
$p_A$- and $p_I$.
In this case, we might be able to distinguish between the cost and benefit unobservables based on their variances.}
To see this, observe that the additive error in costs becomes the additive error in the net benefit to a misstatement. 
Further, the probability of a restatement becomes
\begin{equation}\label{restate1}
\mbox{Pr(Restate)} = \theta_0 \mbox{Pr}\left(\, \theta_1 + \theta_2 \mbox{BONUS}
+ \theta_3 \mbox{SALARY}  \ge \xi \,\right)
\end{equation}
where $\theta_0=(1-p_A) \times p_{E}, \theta_1 = (1 - p_A)(1 - p_I) b_0 - m_0, 
\theta_2 = (1 - p_A)(1 - p_I) b_1,$ and $\theta_3 = - m_1.$ 

Apart from the scalar multiple $\theta_0$, which can be absorbed into the probability statement (and is thus not identified), this probability model has the form of  a familiar binary choice (e.g., a probit or logit model).
Thus, the value of the structure imposed so far is that it can motivate the application of a familiar statistical model, as  well as explain how the estimated coefficients are potentially connected to quantities that impact the probability of a misstatement.

\subsection{Estimation}

To illustrate the application of this structural to data, we assembled a dataset containing 5,000 firm-year observations on whether or not financial results were restated in a given year.
Table \ref{tab:desc} describes the variables in our data set.
We discuss the source of the data in more detail after we estimate the structural model.

The data include variables that have previously been used to model restatements.%TODO: \footnote{[Refs to be included]} 
The variable BIG4 is included  because it is believed that Big 4 auditing firms have more expertise and are therefore more likely to catch misstatements. 
Similarly, the corporate governance literature suggests that board oversight from directors with accounting or finance backgrounds reduces the likelihood that CEOs will make misstatements. 
%TODO: Get refs.
Finally, the variables INT and 
SEG are included to capture the complexity and costs of audits. 
%TODO: Get reference for audit model.
Specifically, international companies and companies with more business segments are thought to raise the costs of auditing. 

Table \ref{tab:desc} reports descriptive statistics for the sample. CEOs are on average receive about three-quarters of a million dollars in base pay and their incentive-related pay averages 26\% of their total pay.
Three-quarters of the sample has a Big 4 accounting firm as its auditor. The fraction of directors with financial expertise is less than ten percent. 
The average firm has about four and one-half business SEG and is primarily based in the United States.

Absent a theoretical model of mis- or re-statements, most empirical
analyses of these data would summarize them by regressing the restatement
indicator on the list of predictors in Table 1. Such a regression can either be described
as showing how the probability of a restatement co-varies with the right hand side variables,
or it can be described as a prediction equation. Our structural model allows us to say more,
particularly when it comes to the signs of the SALARY and BONUS variables. By including
the other variables on the right hand side, we are in essence maintaining they belong in
$X_C$, $X_B$, or both sets of regressors.

Table \ref{tab:logit} reports the results of logit regressions in which the dependent
variable is the restatement indicator variable. 
The table contains both a simple 
specification containing an intercept along with the two CEO pay variables, 
and a more intricate specification involving the other variables we have in our data.
For each specification we report the estimated coefficients of the logit and the 
corresponding marginal effects evaluated at the sample averages of the exogenous
variables.%TODO: Explain marginal effects.
The results for the pay coefficients in both specifications run counter those the previous accounting literature might predict and counter to those predicted by the structural model.
Specifically,  more base pay is associated with more restatements, while more incentive pay is associated with fewer restatements.
Besides the intercepts and BONUS coefficients, the only other coefficients  that are statistically significant are INTional companies and companies with more SEG to audit. 
While we can say (descriptively) that they are associated
with higher restatement rates, unless we take a position on how they enter $X_C$ or $X_B$, it is difficult to interpret whether their signs make sense.

The question we now address is what to make of the fact that the coefficients
on the CEO pay variables did not turn out as either informal arguments or our
structural model would predict. 
We consulted colleagues in the profession for their opinions. 
%TODO: Really?
One set attributed the outcome to data errors or sampling issues. 
Another set said the logit model was obviously flawed because it did not include all relevant accounting variables. (High on the list were measures
of recent accrual activity, use of off balance sheet transactions and firm
performance variables.) 
Their thinking was, had we included these, the signs on the pay coefficients might be different. 
Mentioned less often was that the idea that an important difference between restatements and misstatements is the role of subsequent investigations. 
There is nothing in the model or variable list that would help predict  the intensiveness of outside scrutiny (or indeed of scrutiny by the firm's external auditors).

\subsection{An alternative model}

The point we would like to make here is that the structural model can help us understand the potential influence and importance of each of these points. 
To illustrate this, we will focus attention on the model's potentially overly simplistic view of the auditor's role in detecting misstatements.  
To make the model richer, suppose the firm's  auditors are more likely to look for misstatements when they perceive they are more likely to find misstatements. 
This reasoning naturally
leads to a strategic decision making model where the managers' and the auditors' decisions are interdependent.
%TODO: Get references (e.g. Tshibano (198X) and Bresnahan and Reiss (1991)).
In such a model, auditors too presumably trade off the costs of audit effort against the reputational losses they might incur should they miss a managerial misstatement and the misstatement is subsequently detected.\footnote{
Here we have in mind the findings of \citet{Dyck:2010kh} who show that many egregious forms of misstatements are detected subsequently by employees, directors, regulators, and the media.} 

In the previous model, the firm's auditor impacted the manager's misstatement benefits through $p_A$. 
Suppose that $p_A$ is in fact a choice variable for the firm's auditor. 
To make matters simple, suppose that the auditor detects manipulation with probability $p_{AH}$ if they exert high effort and  otherwise they detect manipulation with the lower probability $p_{AL}$. 
Let the cost of high effort be a fixed cost $C_A > 0$. 
Without loss of generality suppose the cost of low effort is zero. 
When deciding whether to audit with high or low effort, the auditor perceives a cost to its reputation, $C_R$, to not detecting a misstatement that is subsequently caught by an external investigations. 
This structure implies that the total cost of high effort to the auditor is $C_A + (1-p_{AH}) \times p_E \times C_R$ -- the cost of high effort plus the expected cost of missing a misstatement that is subsequently caught with probability $p_E$. 
The total expected cost of
low effort is similarly, $(1-p_{AL}) \times p_E \times C_R$. 

We complete this new model, we need to make an (equilibrium) assumption about how the CEO and firm auditor interact. Following the literature, we assume that the two simultaneously
and independently make decisions, and that their strategies form a Nash equilibrium.
That is, we assume the players' strategies are such that they optimize their objectives 
taking the actions of the other players as fixed. This means that in a Nash equilibrium, 
the players are taking actions that they cannot unilaterally improve upon.

It is well known that in this type of auditing game, that the CEO and the auditor 
best actions are to play a mixed (randomized) strategy.\footnote{Explain.}
That is, the auditor will independently exert high effort with probability $\alpha^*$ 
and the manager independently misstates with probability $\beta^*$. These probabilities 
are such that each has no incentive to change their randomized strategy; that is:\\
\begin{quote}
(i) the manager is indifferent between misstating and not misstating, or:\\
\begin{equation}\label{manager}
(1 - p_A^*)(1 - p_I) B - C_M = 0 
\end{equation}
where $p_A^* = \alpha^* p_{AH }+ (1-\alpha^*) p_{AL}$ is the equilibrium 
probability a misstatement is detected; and,\\
\end{quote}
\begin{quote} (ii) the auditor must be
indifferent between exerting high and low effort, or\\
$$ \beta^* (1-p_{AH}) p_G C_R + C_A = \beta^* (1-p_{AL}) p_I C_R .$$
\end{quote}

\vglue 5pt
Solving these two equations for the equilibrium probabilities $\alpha^*$ and $\beta^*$
yields:\\
\begin{equation}\label{equilstrat}
\begin{array}{lcl}
  \alpha^* &= & \dfrac{ ( 1 - p_{AL}) (1 - p_I) B- C_M}{ (1 - p_I) (p_{AH}-p_{AL}) B}\\[1.5em]
  \beta^* &= & \dfrac{C_A}{(p_{AH}-p_{AL}) p_I C_R}  
\end{array}
\end{equation}
From these equations, we can calculate the equilibrium probability of a restatement\footnote{
As part of the solution, we require $\alpha^*$ and $\beta^*$ to be probabilities between
zero and one. This is true provided $C_R$ and $B$ satisfy the inequality\\
$$  C_R > \frac{C_A}{(p_H-p_L)p_I} $$
and \\
$$ B > \frac{C_M}{1 - p_I}  $$}
\begin{equation} \label{equilpr1}
\begin{array}{lcl}
\mbox{Pr(Restate)} & = &  \mbox{Pr(Misstate)} \times \mbox{Pr(Auditor Misses)} \times
\mbox{Pr(Investigation Finds)}\\[1em]
& = & \beta^* \times (1-p_A^*) \times p_{E}
\end{array}\end{equation}
This equation tells us how the observed (or measurable) probability of a restatement is related to
the unobserved frequency of misstatements. In particular, if we knew the frequency with which auditors and investigations caught misstatements, we could easily link the two. Otherwise,
we would have to estimate these probabilities (or make assumptions about them).

Substituting the equilibrium strategies (\ref{equilstrat}) into (\ref{equilpr1}) yields
\begin{equation} \label{equilpr2}
\begin{array}{lcl}
\mbox{Pr(Restate)}& = &  \dfrac{C_AC_M(1-p_{IL})}{(p_{IH}-p_{IL})(1-p_I)C_RB}.
\end{array}\end{equation}
We now are in a position to use the theory to help interpret the conflicting logistic regression results
in Table 3. 

 
Equation (\ref{equilpr2}) shows that the presence of a strategic external auditor
changes how the CEO's incentives impact the probability of a restatement.\footnote{Notice that
the probability statement in equation  (\ref{equilpr2})  differs from that in equation (\ref{restate1}).
The probability statement in equation  (\ref{equilpr2}) reflects the randomness of the
strategies, whereas in equation (\ref{restate1}) it reflects unobservables the researcher 
does not observe.} Partial derivatives of equation (\ref{equilpr2}) show that the restatement probability is:\\

\begin{itemize}
\item Decreasing in the benefit $B$ that the manager enjoys from misstatement.
\item Increasing in the personal cost of manipulation $C_M$ incurred by the manager.
\item Decreasing in the reputational cost $C_R$ incurred by the external auditor.
\item Increasing in the cost of high effort $C_A$ incurred by the external auditor.
\end{itemize}

Thus, in contrast to the nonstrategic model, increasing the benefit that managers enjoy from misstatement,
or decreasing the misstatement cost, leads to fewer restatements being observed by researchers.
These two effects explain the negative sign on BONUS and the positive sign on SALARY observed in the previous logit results. Thus, this structural model has the potential to rationalize patterns observed in the data.

To have a better sense of how one might connect the strategic auditor theory to the
logistic models in Table \ref{tab:logit}, suppose, similar to ways we motivated (\ref{eqns1}), that 
\begin{equation}\begin{array}{lcl}\label{eqns2}
B & = & b_0 + b_1 \, \mbox{BONUS} \\[.5em]
C_M & = & m_0 + m_1 \, \mbox{SALARY} \\[.5em]
C_A & = & a_0 + a_1 \, \mbox{INT} + a_2 \, \mbox{SEG}\\[.5em]
C_R & = & r_0, \quad B  =  b_0, \quad p_{AH}   =  p_0, \; \mbox{ and } \; p_{AL}  =  v_0 \\[.5em]
\end{array}
\end{equation}
where $ a_0, a_1, a_2, r_0, b_0, p_0$ and $v_0$ are constant parameters. 
Inserting these expressions into the expected restatement rate (\ref{equilpr2}) gives
\begin{equation*} \label{equilpr3}
\begin{array}{lcl}
\mbox{Pr(Restate)}& = &  \dfrac{C_AC_M(1-p_{IL})}{(p_{IH}-p_{IL})(1-p_I)C_RB}\\[2em]
& = & \dfrac{(1-v_0)(a_0 + a_1 \, \mbox{INT} + a_2 \, \mbox{SEG})(m_0 + m_1 \, \mbox{SALARY})}
{(p_0-v_0)r_0(b_0 + b_1 \, \mbox{BONUS})}\\[2em]
\end{array}
\end{equation*}
\begin{equation}\label{equilpr4}
 =  \dfrac{\theta_0 + \theta_1\mbox{\small INT} + \theta_2\mbox{\small SEG} + \theta_3\mbox{\small SALARY}
+ \theta_4\mbox{\small INT} \times \mbox{\small SALARY}+ \theta_5\mbox{\small SEG} \times \mbox{\small SALARY}}
{1 +  \theta_6\mbox{\small BONUS}}
\end{equation}
Notice that the $\theta$s absorb unknown quantities such as $r_0$ and $p_0$, and that the denominator intercept
is normalized to one. This last restriction is required to identify the ratio of the two linear functions.

Although this model does not have a logit form, it is potentially estimable using 
generalized method of moments (GMM).
This method attempts to match so-called sample moments to what the structural model implies they should be. 
For example, an obvious sample moment would be the average restatement rate in the sample.
The corresponding theoretical moment would be the probabilty expression in equation (\ref{equilpr4}).
Because we need at least as many moments as we have $\theta$ parameters to estimate (there are seven $\theta$s in the model), we use seven sample moments, each of the form:
$$ \mathcal{M}_j = \sum_{i=1}^{5,000} \; X_{ji}^\prime\left[\; \mbox{RESTATE}_A - \mbox{Pr(Restate)}_A \; \right]. $$
where Pr(Restate) comes from equation (\ref{equilpr4}).\footnote{
To ensure that the model parameters imply restatement probabilities between zero and one, we add a penalty function to the GMM objective function.
This penalty increases with the number of estimated probabilities below zero or above one.
For most replications this penalty is immaterial to the results obtained.} 
The $X_j$ used in the moments include all explanatory variables. 
Thus, because $X$ includes  a dummy variable for whether the firm is an international company, the corresponding moment equation seeks to match the sample international companies average restatement rate to the model's prediction for that rate.

Table \ref{tab:gmm} reports the results of estimating the new (strategic auditor) structural model on the sample of 5,000 firms. 
The results show that in this particular case, even without sample information on the unobserved probabilities $p_A$ and $p_I$, we can recover estimates of the model parameters up to a normalization.\footnote{
A simple way to see this might be the case is to observe that there are seven $\theta$ coefficients and eight underlying structural parameters.}
For instance, the coefficient ratio $\theta_3/\theta_0$ estimates the ratio of cost parameters $m_1/m_0$.
Since $m_1$ is the cost coefficient on SALARY and $m_0>0$ for costs to make sense, the sign of $\theta_3/\theta_0$ reveals the sign of $m_1$.
From the theory, we expect the sign to be positive, and in the estimates it is. 
Similarly, $\theta_6$ equals the (scaled) misstatement benefit coefficient on the BONUS variable.
Although the descriptive regression coefficients in Table 3 suggest BONUS has a negative effect on restatements, here, because we model misstatements as part of restatements, we find it has a positive effect, as expected.

The one sign that does not make sense given the other coefficient estimates is the negative sign on $\theta_6$, however this coefficient is insignificantly different from zero.
This is indeed true of most of the coefficients, and is perhaps not surprising given the participants use of randomized strategies.
Further, even with a sample size of 5,000, restatements are relatively rare, thus making it difficult for the model to predict them with much accuracy. 

While the coefficient magnitudes do not allow us to estimate the underlying benefits and costs to managers from misstatements, we can illustrate the value of the model by performing a counterfactual calculation.
There are many different counterfactuals that could be considered. 
Here, for illustrative purposes we can ask what would happen to misstatements and restatements if we  do away with incentive pay and nothing else changes.
The value of having a model to analyze this
change is that we can see how the auditing process would adjust to the removal of 
CEO incentives to misstate. 
From the equilibrium strategies in equation (\ref{equilstrat}), we see that removing bonus pay does not
change the equilibrium frequency of misstatements, but does change the frequency of high effort auditing.
From (\ref{equilpr}), the model and the data we find
$$ \dfrac{\mbox{Pr(Restate }\vert \mbox{ No Bonus)}}{\mbox{Pr(Restate }\vert \mbox{ Bonus)}}=\dfrac{\beta^* \times (1-p_A^{**}) \times p_{E}}
{\beta^* \times (1-p_A^{*}) \times p_{E}} = \dfrac{(1-p_A^*)}{(1-p_A^*)} = 1.10.$$
What this says is that the restatement rate increases by 10\% (from 10.24\% to 11.25\%) when the bonuses
are withdrawn. 
The fact that the restatement rate goes up may at first seem somewhat odd given that the benefits to the CEOs have fallen. 
The model, however, shows that the increase  comes about because the auditors exert \emph{less} effort in detecting misstatements, thereby catching fewer, leaving more for outsiders to subsequently catch.  

The discussion above illustrates some of the ways in which a structural modeling exercise might help understand accounting data. 
In particular, the comparative statics of the model shed light on the difference between restatements and misstatements, and what assumptions (e.g., strategic versus nonstrategic auditor) and data were needed to draw  inferences about misstatements from restatements. 
Additionally, we were able to recover some of the primitive parameters impacting incentives for managers to misstate results, as well as perform counterfactual analyses.

While there is the potential for disappointment in the simplistic theory we used, we see room for improving models as an opportunity rather than a defining limitation.

\subsection{Limitations of structural models}
% This discussion may belong near the end. When we talk about structural modeling, we should make clear that some questions are difficult to fit into a structural model, etc.
The vast majority of empirical research papers in accounting do not rely on a formal theoretical model to motivate their hypotheses.
But in many cases, extant theory would not support estimation of a structural model.
For example, \citet{Huang:2014cs} study the effect ``tone management'' on capital market outcomes.
Developing a formal theory of the relation between firm performance, managerial psychological states, and measures of tone would be a complex undertaking involving economics, psychology, and linguistics.
Building on such a (hypothetical) foundation to solve the complex game involving managers and capital markets would be extremely ambitious.
Instead, \citet{Huang:2014cs} does what almost all empirical research papers in accounting do and resorts to more verbal approaches to hypothesis development. 
% Can we find some critique of this approach? Pfleiderer?  YES -- let's review Paul's paper and add some of this and include him in the references

To those who would see the ``glass as half full," the exercise can be seen as yielding insight into an important accounting issue and how accountants might better take advantage of data (and indeed what extra data they might like to collect).

\subsection{Structural models in accounting research}

The use of structural models in accounting research has been very modest to date.  There are certainly examples of regression models being derived (or perhaps influenced) by a theoretical model. For example, Lambert and Larcker (1985) use the traditional Holmstrom model (and a variety of simplifying assumptions) to help specify a regression function linking CEO compensation to firm performance. While somewhat structural in orientation, the approach suppresses the fundamental causal mechanism associated with compensation decisions and does little to actually estimate the primary structural model of Holmstrom.

% This stuff is in Peter's structural section (i.e., before Section 6).
Gerakos and Kovrijnykh (2013) and Nikolaev (2014a,b) provide an analysis misreporting and accounting quality that resembles some features of structural models (e.g., Nikolaev implements method of moment estimation). Both papers develop a dynamic stochastic model for the accounting process and use this structure to separately identify quality earnings from manipulated earnings. While interesting empirical studies, these papers simply assume the fundamental stochastic process for earnings and abstract away from all the underlying (optimal) managerial decisions and accounting choices that ultimately produce observed accounting numbers. 

The recent papers by Zakolyukina (2014) and Bertomeu et al (2015) are more consistent with traditional structural modeling.  Similar to our simple model above, Zakolyukina is concerned with how equity incentives motivate managers to manipulate earnings.  Bertomeu et al examine management forecasts using a formal disclosure model to estimate whether managers strategically withhold information from shareholders.

These two papers model an institutionally rich problem, estimate the derived model, provide estimates for important structural parameters, and also give interesting counterfactuals based on their theoretical models.  
We view these papers as useful initial steps in applying structural approaches to accounting research questions.\footnote{Maybe review some structural finance papers -- Luke Taylor on CEO labor markets, Whited on debt?, Terry jmp about "meet or beat" and macroeconomic research and development.} % Is this footnote for the editors or for us?



\section{Descriptive studies} \label{sec:desc}

Accounting is essentially an applied discipline and it would seem that most empirical research studies should be solidly grounded in the details of how institutions operate.
Unfortunately, there are very few studies published in top accounting journals that focus on providing deep description of institutions relevant to accounting research settings.
Part of this likely reflects the perception that research that pursues causal questions (i.e., tests of theories) is more highly prized and thus more likely to be published in top accounting journals.\footnote{
The \emph{Journal of Accounting Research} used to publish some papers in a section entitled ``Capsules and Comments."
The editor at the time (Nicholas Dopuch) would seem to place papers into this section if ``did not fit" as a main article, but examined new institutional data or ideas. 
Such a journal section might have provided a credible signal of a willingness to publish descriptive studies of institutionally interesting settings.}

We believe that accounting research could benefit substantially from more in-depth descriptive research.
As we discuss below, this type of research is essential for those who seek to develop structural models or improve our understanding of causal mechanisms.\footnote{
There are many ``classic" descriptive studies that have had a major impact on subsequent theoretical and empirical research in organizational behavior and strategy such as \citep{Cyert:1956fd,Bower:1986vd,Mintzberg1973nature}.
\citet{Cyert:1956fd} argue that ``a realistic description and theory of the decision-making process are of central importance to business administration and organization theory. Moreover, it is extremely doubtful whether \dots economics does in fact provide a realistic account of decision-making in large organizations operating in a complex world."}

One reason to value descriptive research is that it can uncover structures and mechanisms that exist, but which would be exceedingly difficult to arrive at from basic economic theory.
For example, using proxy statements, \citet{Healy:1985jg} studies the bonus contracts of 94 large US companies and identifies a common structure of these bonus plans, including the existence of caps and floors \citep[p.\,89]{Healy:1985jg}. The paper also suggests hypotheses worth investigating regarding the effects of these plan features on accounting decisions.
It seems highly unlikely that a model derived from fundamental economic theory would arrive at these plan features actually used by firms.
These institutional features can be used to identify precise mechanisms and also as elements of structural models in which other features might be motivated more directly by economic theory.

Recent published research suggests that increased recognition of the value of descriptive research.
\citet{Soltes:2013ba} examines the interactions between sell-side analysts and company management in one firm that granted him proprietary access to ``offer insights into which analysts privately meet with management, when analysts privately interact with management, and why these interactions occur."  
By comparing private interaction to observed interaction between analysts and managers on conference calls and highlighting that private interaction with management is an important communication channel for analysts, \citet{Soltes:2013ba} provides a plausible mechanism through which information transfers hypothesized in more traditional empirical papers actually occur.

That private communication with management is an important source of information is confirmed by  \citet{Brown:2015kd}. \citet{Brown:2015kd} survey and interview financial analysts to understand how they think about a variety of issues. 
Their findings suggest that analysts' views on earnings quality differ from those researchers focus on. 
For instance, analysts do not use the ``red flags" used by academics to identify manipulation; and analysts generally are not attempting to uncover manipulation and use forecasts, not as ends in themselves, but figure out the stock price target.
These insights should shape research seeking to develop hypotheses and models of accounting information and analyst behavior.

Other fields provide interesting examples likely to be of interest to accounting researchers.
For example, \citet{Ahern:2014id} examines 183 illegal insider networks using primary source documents from the SEC, DOJ, and various public records. 
It provides rich insights into investor networks and it suggests questions for future work.
For example, network relationships can be divided into familial (23\%), business-related (35\%), friendships (35\%), or ``not clear" (21\%).
Insiders are more likely to be an accountant or lawyer, less likely to be a Democrat, and more likely to have a ``criminal record."
These results provide new institutional insights have have the potential to identify causal mechanisms regarding information transfer and disclosure.
In economics, \citet{Bloom:2007ed} provide a descriptive study of 732 medium-sized firms to assess whether management practices, such as lean manufacturing and use of incentives, are related to productivity. These descriptive results have lead to the development of theoretical models of innovating and productivity along with increasingly sophisticated empirical studies.

% Although it is a matter of taste, a number of the papers in our review seem to ask research questions and use causal mechanisms that are far removed from real world issues.

%Select a topic or setting of real interest to accounting researchers - analysts, loan officers, executives making strategic decisions, compensation committee members, etc. 
% How do managers and boards make decisions on ``disclosure quality" (an aside - do they even know what disclosure quality means and do they care?  Does this construct have any real relevance in the real world?)  

% You need to go out and actually interview these people using structured and semi-structured interviews.  Not just one per company, but several at varying levels.
% Need to understand the setting, economics aspects, behavioral aspects.
% Map out the mechanism by which decisions of interest are actually made.
% Maybe there are multiple mechanisms depending on the situation.
% What are the contextual variables?

%Use these qualitative insights to justify research question selection.  Would conditional conservatism show up in conversations with real world managers?  If you had an unstructured conversation, what accounting topics would be ``top of mind" for managers, board members, bankers, etc.?
%
%No doubt we would find many cases where present accounting research substantially departs from known institutional details - do executive really behave like Black-Scholes would imply?
%How are compensation plans really designed and why?  Why don’t companies take advantage of ``academically obvious" tax changes?
%
%Ultimately uUse these mechanisms to develop structural models with a plausible causal mechanism.
%
%Punchline: if you want to explain something observed, maybe it is a good idea to understand the phenomenon of interest first.

%TODO: somewhere we want to note that there are some real field experiments -- John Roberts with the impact of consulting firms in India and a variety of related studies in Economics.  
%TODO: There was also the SEC decision to randomly assign short selling restrictions.

\section{Concluding remarks} \label{sec:conclude}
In this paper, we examined the approaches used by accounting researchers to draw causal inferences from analyses of observational (or non-experimental) data. 
The vast majority of empirical papers using such data seek to draw causal inferences, notwithstanding the well-known difficulties with doing so.
While some papers seek to use quasi-experimental methods to develop unbiased estimates of causal effects, we find that the assumptions required to deliver such estimates
are not often credible. We believe that clearer communication of research questions and design choices would help researchers avoid some of the conceptual traps that affect 
accounting research. One tool that may help in this regard are causal diagrams.

We also argued that  accounting research could benefit from a more complete understanding of causal pathways through the use of rigorous theory. 
In particular, we believe that structural models will see greater use in the coming years. Finally we see great value to in-depth descriptive studies that inform 
causal issues and deepen our knowledge of the behavior and institutions we seek to model. Although our suggestions do not completely resolve controversies 
surrounding causal inferences drawn from observational data, we believe they offer a viable and exciting path forward.

\end{doublespace}

\clearpage
\bibliography{jar_methods}

\clearpage

\begin{figure}
	
    \centering
    \caption{Three basic causal diagrams} \label{fig:basic}
    \addtocounter{figure}{-1}
    \begin{subfigure}{\textwidth}
        \caption{$Z$ is a confounder} \label{fig:confound}
        \centering
		\begin{tikzpicture}
		    \node[rectangle] (X) {Treatment variable (X)}; 
		    \node [right = 1 of X] (Y) {Outcome variable (Y)};
		    \node [below = 1 of X] (Z) {``Control" (Z)};
		    \draw (X) edge[->] (Y);
		    \draw (Z) edge[->] (X);
		    \draw (Z) edge[->] (Y);
		\end{tikzpicture}
    \end{subfigure}

    \begin{subfigure}[b]{\textwidth}
        \caption{$Z$ is mediator} \label{fig:mech}
        \centering
		\begin{tikzpicture}
		    %create X and Y node
		    \node[rectangle] (X) {Treatment variable (X)}; 
		    \node [right = 1 of X] (Y) {Outcome variable (Y)};
		    \node [below = 1 of X] (Z) {``Control" (Z)};
		    \draw (X) edge[->] (Y);
		    \draw (Z) edge[<-] (X);
		    \draw (Z) edge[->] (Y);
		\end{tikzpicture}
    \end{subfigure}

    \begin{subfigure}[b]{\textwidth}
    	\caption{$Z$ is a collider} \label{fig:collider}
		\centering
		\begin{tikzpicture}
		    \node[rectangle] (X) {Treatment variable (X)}; 
		    \node [right = 1 of X] (Y) {Outcome variable (Y)};
		    \node [below = 1 of X] (Z) {``Control" (Z)};
		    \draw (X) edge[->] (Y);
		    \draw (Z) edge[<-] (X);
		    \draw (Z) edge[<-] (Y);
		\end{tikzpicture} 

    \end{subfigure}

\end{figure}

\clearpage
\begin{figure} 
    \caption{Causal graph for \citet{Armstrong:2013io}} \label{fig:agl}
    \centering
	\begin{tikzpicture}
		\node[rectangle] (C0) {Compensation$_{t-1}$}; 
		\node[draw,dashed,fill=white] at ([shift={(3,-2.5)}] C0) (EC1) {Shareholder-observable determinants of compensation$_{t+1}$};
		\node [right = 3 of C0] (C1) {Compensation$_{t+1}$};
		\node [below = 3 of C1] (SS0) {Shareholder support$_t$};
		\node [below = 3 of C0] (ISS) {ISS recommendation$_t$};
				
		\draw (C0) edge[->] (ISS);
		\draw (C0) edge[->] (EC1);
		\draw (SS0) edge[->] (C1);
		\draw (C0) edge[->] (C1);
		\draw (ISS) edge[->] (SS0);
		\draw (EC1) edge[->] (C1);
		\draw (EC1) edge[->] (SS0);
	\end{tikzpicture}
\end{figure}


\clearpage
\begin{figure} 
    \centering
	\caption{Identifying effects of analyst coverage changes} 
    \addtocounter{figure}{-1}
    \begin{subfigure}{\textwidth}
        \caption{Causal graph for \citet{Kelly:2012ih}} \label{fig:kl}
        \centering
		\begin{tikzpicture}
			\node[rectangle] (BC) {Brokerage closure$_t$}; 
			\node [right = 1 of BC] (AC0) {Analyst coverage$_t$};
			\node [below = 1 of AC0] (IA0) {Information asymmetry$_t$};
			\draw (BC) edge[->] (AC0);
			\draw (AC0) edge[->] (IA0);
			\draw (IA0) edge[bend right=60, dashed, <->] (AC0);
		\end{tikzpicture}
    \end{subfigure}

    \begin{subfigure}[b]{\textwidth}
        \caption{Causal graph for \citet{Balakrishnan:2014js}} \label{fig:bbkl}
        \centering
		\begin{tikzpicture}
						\node[rectangle] (BC) {Brokerage closure$_t$}; 
			\node [right = 1 of BC] (AC0) {Analyst coverage$_t$};
			\node [right = 1 of AC0] (AC1) {Analyst coverage$_{t+1}$};
			\node [below = 1 of AC0] (IA0) {Information asymmetry$_t$};
			\node [right = 1 of IA0] (IA1) {Information asymmetry$_{t+1}$};
			\node [below = 1 of IA0] (D) {Disclosure};
			\node [right = 1 of D] (LQ) {Liquidity$_{t+1}$};
			\draw (BC) edge[->] (AC0);
			\draw (AC0) edge[->] (IA0);
			% \draw (AC0) edge[->] (AC1);
			\draw (AC1) edge[->] (IA1);
			\draw (IA0) edge[->] (D);
			\draw (IA1) edge[->] (LQ);
			\draw (D) edge[->] (LQ);
			\draw (IA0) edge[bend right=60, dashed, <->] (AC0);
			\draw (IA1) edge[bend right=60, dashed, <->] (AC1);
		\end{tikzpicture}
    \end{subfigure}
    
        \begin{subfigure}[b]{\textwidth}
        \caption{Alternative causal graph for \citet{Balakrishnan:2014js}} \label{fig:bbkl_alt}
        \centering
		\begin{tikzpicture}
						\node[rectangle] (BC) {Brokerage closure$_t$}; 
			\node [right = 1 of BC] (AC0) {Analyst coverage$_t$};
			\node [right = 1 of AC0] (AC1) {Analyst coverage$_{t+1}$};
			\node [below = 1 of AC0] (IA0) {Information asymmetry$_t$};
			\node [right = 1 of IA0] (IA1) {Information asymmetry$_{t+1}$};
			\node [below = 1 of IA0] (D) {Disclosure};
			\node [right = 1 of D] (LQ) {Liquidity$_{t+1}$};
			\draw (BC) edge[->] (AC0);
			\draw (AC0) edge[->] (IA0);
			\draw (AC0) edge[->, line width=2pt] (AC1);
			\draw (AC1) edge[->] (IA1);
			\draw (IA0) edge[->] (D);
			\draw (IA1) edge[->] (LQ);
			\draw (D) edge[->] (LQ);
			\draw (IA0) edge[bend right=60, dashed, <->] (AC0);
			\draw (IA1) edge[bend right=60, dashed, <->] (AC1);
		\end{tikzpicture}
    \end{subfigure}
\end{figure}

\begin{landscape}

\begin{figure}

\caption{Causal diagram for strategic auditor model} \label{fig:audit}

% Figure depicts causal graph for model with strategic auditor discussed in the text. 

\begin{tikzpicture}
    
    % Exogenous stuff
    \node[rectangle] (BONUS) {BONUS};  
    \node [right = 1 of BONUS] (SALARY) {SALARY};  
    \node [left = 1 of BONUS] (SEG) {SEG};    
    \node [left = 1 of SEG] (INT) {INT};    
	
    \node[below = 1 of BONUS,draw,dashed,fill=white] (B) {Managerial incentives ($B$)}; 
    
    % Equilibrium
    \node [right = 1 of B, below = 2 of B,draw,dashed,fill=white] (alpha) {Audit effort};
    \node [right = 1 of alpha,draw,dashed,fill=white] (beta) {Attempted misstatement};
    \node [below = 1 of SALARY,draw,dashed,fill=white] (C_M) {Cost of\\manipulation ($C_M$)};
    \node [below = 1 of beta,draw,dashed,fill=white] (M) {Misstatement};
    \node [left = 1 of B,,draw,dashed,fill=white] (C_A) {Cost of audit effort ($C_A$)};
    \node [below = 4 of C_A,draw,dashed,fill=white] (p_I) {$\mathrm{Pr}$(Detection by subsequent investigation)\\ ($p_I$)};
    \node [below = 1.5 of INT,draw,dashed,fill=white] (C_R) {Auditor\\reputational concerns ($C_R$)};  
    
    % Unobservable outcome

    % Observable outcome
    \node [below = 2 of alpha] (R) {Restatement};
    
    \draw (INT) edge[->] (C_A);
	\draw (SEG) edge[->] (C_A);
    \draw (BONUS) edge[->] (B);
    
    \draw (BONUS) edge[->] (B);
    \draw (SALARY) edge[->] (C_M);
    \draw (B) edge[->] (beta);
    
    \draw (C_R) edge[->] (beta);
    \draw (C_A) edge[->] (beta);
    \draw (C_M) edge[->] (beta);
    \draw (C_M) edge[->] (beta);
    \draw (p_I) edge[->] (beta);
    
    \draw (B) edge[->] (alpha);
	\draw (C_R) edge[->] (alpha);
    \draw (C_A) edge[->] (alpha);
    \draw (C_M) edge[->] (alpha);
    \draw (p_I) edge[->] (alpha);
    
    \draw (alpha) edge[->] (M);
    \draw (beta) edge[->] (M);
    \draw (M) edge[->] (R);
    \draw (p_I) edge[->] (R);
    
    % \draw (IA) edge[bend right=60, dashed, <->] (AC);
\end{tikzpicture}
\end{figure}
\end{landscape}



\clearpage
\begin{table}[t]

\caption{Descriptive statistics.} \label{tab:desc}

\justify{
{\bf RESTATE} is a zero-one indicator for whether the firm made a restatement in a particular year.
{\bf SALARY} is the CEO's annual base salary.  
{\bf EQUITY} is the fraction of a CEO's total pay that is incentive compensation.  
{\bf BIG4} is a zero-one indicator for whether the firm uses a Big 4 auditor.
{\bf FINDIRECT} is the fraction of the board of directors with a finance background. 
{\bf INT} is a zero-one indicator for non-US corporation.  
{\bf SEG} is number of the firm's business segments.\\}

\begin{center}
\begin{tabular}{|l|c|}
\hline
             & {\bf Sample }   \\
{\bf Variable}  &   {\bf Mean }   \\ 
          &  (Std Dev)   \\ \hline
\T {\bf RESTATE} &   0.102  \\
                           &   0.303  \\[.6em]
    {\bf SALARY} &     0.95  \\
                           &     0.15  \\[.6em]
    {\bf EQUITY} &    0.26  \\
                           &     0.29  \\[.6em]
    {\bf BIG4} &   0.76  \\
                           &     0.43  \\[.6em]
    {\bf FINDIRECT} &    0.08  \\
                           &     0.08  \\[.6em]
    {\bf SEG} &    4.41  \\
                           &    3.02  \\[.6em]
    {\bf INT} &     0.30  \\
                           &     0.46  \\[.6em]
\hline
\end{tabular}
\end{center}
\end{table}

\begin{table}[t]
\caption{Logit Regression Results} \label{tab:logit}

\justify{This table presents results from logistic regressions of 

{\bf RESTATE}, a zero-one indicator for whether the firm made a restatement in a particular year, on a proxy for managerial incentives ({\bf EQUITY}) and controls.
{\bf SALARY} is the CEO's annual base salary.  
{\bf EQUITY} is the fraction of a CEO's total pay that is incentive compensation.  
{\bf BIG4} is a zero-one indicator for whether the firm uses a Big 4 auditor.
{\bf FINDIRECT} is the fraction of the board of directors with a finance background. 
{\bf INT} is a zero-one indicator for non-US corporation.  
{\bf SEG} is number of the firm's business segments.\\}

\begin{center}
\begin{tabular}{|l|cc|cc|}
\hline
             & \multicolumn{2}{c|}{\bf Specification 1}  &  \multicolumn{2}{c|}{\bf Specification 2}   \\
 {\bf Coefficient on}  &  {\bf Coefficient} & {\bf Marginal Effect} & {\bf Coefficient} & {\bf Marginal Effect}  \\ 
          &  (Std Dev)  & (Std Dev) &  (Std Dev)  & (Std Dev) \\ \hline
\T {\bf Intercept} &   -2.210*   &   & -2.786*   &    \\
                   &   0.308   &   & 0.349   &   \\[.6em]
 {\bf SALARY}    &   0.118    &  0.010  & 0.010    &  0.001   \\
                   &   0.317   &  0.027 & 0.327   &  0.027 \\[.6em]
 {\bf EQUITY}  &   -0.644*   &  -0.055* & -0.673*   &  -0.056*  \\
                   &   0.185   &  0.016 & 0.201   &  0.017 \\[.6em]
 {\bf BIG4 }  &       &    & 0.104    &  0.009   \\
                   &      &   & 0.130   &  0.011 \\[.6em]
 {\bf FINDIRECT}  &       &    & -0.058    &  -0.005   \\
                   &      &   & 0.740   &  0.062 \\[.6em]
 {\bf INT }  &       &    & 0.630*   &  0.052*  \\
                   &      &   & 0.120   &  0.010 \\[.6em]
 {\bf SEG}  &       &    & 0.086*   &  0.007*  \\
                   &      &   & 0.021   &  0.002 \\[.6em]
\hline
\end{tabular}
\end{center}
\end{table}

%TODO: Add standard errors and table header to Table 3.
%TODO: Include marginal effects.
\begin{table}[t]
\caption{Logit GMM Estimates for the Strategic Auditor Model} \label{tab:gmm}

\justify{This table presents results for GMM estimates of the strategic auditor model of Section \ref{sec:struct}.\\}

\begin{center}
\begin{tabular}{|c|cc|}
\hline
\T  &  {\bf Estimated}  & {\bf Bootstrap Std} \\ 
\B  & {\bf Coefficient} & {\bf Error} \\ \hline
\T $\theta_0= \dfrac{(1-v_0)a_0m_0}{(p_0-v_0)r_0b_0}$ & 0.0342 &\\[1.5em]
 $\theta_1=\dfrac{(1-v_0)a_1m_0}{(p_0-v_0)r_0b_0}$  &  0.0022 & \\[1.5em]
$\theta_2=\dfrac{(1-v_0)a_2m_0}{(p_0-v_0)r_0b_0}$  &  0.0145   & \\[1.5em]
$\theta_3=\dfrac{(1-v_0)a_0m_1}{(p_0-v_0)r_0b_0}$  &  0.0108  & \\[1.5em]
$\theta_4=\dfrac{(1-v_0)a_1m_1}{(p_0-v_0)r_0b_0}$ &  0.1015  &  \\[1.5em]
$\theta_5=\dfrac{(1-v_0)a_2m_1}{(p_0-v_0)r_0b_0}$ &  -0.0064  & \\[1.5em]
\B$\theta_6=\dfrac{b_1}{b_0}$ &  0.4557  & \\[1.6em]
\hline
\end{tabular}
\end{center}
\end{table}


\appendix
\section{Causal diagrams: Formalities} \label{append}
 
 \subsection{Definitions and a result}
 We first introduce some basic definitions and a key result.
 
\begin{definition}[$d$-separation, block, collider]
A path $p$ is said to be \emph{$d$-separated} (or \emph{blocked}) by a set of nodes $Z$ if and only if
\begin{enumerate}
	\item $p$ contains a chain $i \rightarrow m \rightarrow j$ or a fork $i \leftarrow m \rightarrow j$ such that the middle node $m$ is in $Z$, or
	\item $p$ contains an inverted fork (or \emph{collider}) $i \rightarrow m \leftarrow j$ such that the middle node $m$ is not in $Z$ and such that no descendant of $m$ is in $Z$
\end{enumerate}
\end{definition}

\begin{definition}[Back-door criterion]
A set of variables $Z$ satisfies the \emph{back-door criterion} relative to a an ordered pair of variables $(X, Y)$ in a 
	DAG $G$ if:
	\begin{itemize}
		\item no node in $Z$ is a descendant of $X$; and
		\item $Z$ blocks every path between $X$ and $Y$ that contains an arrow into $X$.\footnote{The ``arrow into $X$" is the portion of the definition that is explains the ``back-door" terminology.}
	\end{itemize}
\end{definition}%
Given this criterion, \citet[p.\,79]{Pearl:2009vo} proves the following result.
%
\begin{theorem}[Back-door adjustment]
	If a set of variables $Z$ satisfies the back-door criterion relative to $(X, Y)$, then the causal effect of $X$ on $Y$ is identifiable and is given by the formula 
	\[ P(y | x) = \sum_{z} P(y | x, z) P(z), \]
where $P(y|x)$ stands for the probability that $Y = y$, given that $X$ is set to level $X=x$ by external intervention.\footnote{
How the quantities $P(y|x)$ map into estimates of causal effects is not critical to the current discussion, it suffices to note that in a given setting, it can be calculated if the needed variables are observable.}
\end{theorem}
%

 \subsection{Application of back-door criterion to Figure \ref{fig:basic}}
Applying the back-door criterion to Figure \ref{fig:confound} is straightforward and intuitive.
The set of variables $\{Z\}$ or simply $Z$ satisfies the criterion, as $Z$ is not a descendant of $X$ and $Z$ blocks the back-door path $X \leftarrow Z \rightarrow Y$.
So by conditioning on $Z$, we can estimate the causal effect of $X$ on $Y$.
This situation is a generalization of linear model in which $Y = X \beta + Z \gamma + \epsilon_Y$ and $\epsilon_Y$ is independent of $X$ and $Z$, but $X$ and $Z$ are correlated.
In this case, it is well known that omission of $Z$ would result in a biased estimate of $\beta$, the causal effect of $X$ on $Y$, but by including $Z$ in the regression, we get an unbiased estimate of $\beta$.
In this situation, $Z$ is a \emph{confounder}.

Turning to Figure \ref{fig:mech}, we see that $Z$, which is a \emph{mediator} of the effect of $X$ on $Y$, does not satisfy the back-door criterion, because $Z$ is a descendant of $X$.
However, $\emptyset$ (i.e., the empty set) does satisfy the back-door criterion.
Clearly, $\emptyset$ contains no descendant of $X$.
Furthermore, the only path other than $X \rightarrow Y$ that exists is $X \rightarrow Z \rightarrow Y$, which does not have a back-door into $X$.
Note that the back-door criterion not only implies that we need not condition on $Z$ to obtain an unbiased estimate of the causal effect of $X$ on $Y$, but that we should not condition of $Z$ to get such an estimate.

Finally in Figure \ref{fig:collider}, we have $Z$ acting as what \citet[p.\,17]{Pearl:2009kh} refers to as a ``collider" variable.\footnote{
The two arrows from $X$ and $Y$ ``collide" in $Z$.} 
Again, we see that $Z$ does not satisfy the back-door criterion, because $Z$ is a descendant of $X$.
However, $\emptyset$ again satisfies the back-door criterion.
First, contains no descendant of $X$.
Second, the only path other than $X \rightarrow Y$ that exists is $X \rightarrow Z \leftarrow Y$, which does not have a back-door into $X$.
Again, the back-door criterion not only implies that we need not condition on $Z$, but that we should not condition of $Z$ to get an unbiased estimate of the causal effect of $X$ on $Y$.

 \subsection{Causal diagrams and instrumental variables}

\begin{definition}[Instrument]
Let $G$ denote a causal graph in which $X$ has an effect on $Y$. 
Let $G_{\overline{X}}$ denote the causal graph created by deleting all arrows emanating from $X$.
A variable $Z$ is an \emph{instrument} relative to the total effect of $X$ on $Y$ if there exists a set of nodes $S$, unaffected by $X$, such that
\begin{enumerate}
\item $S$ $d$-separates $Z$ from $Y$ in $G_{\overline{X}}$
\item $S$ does not $d$-separate $Z$ from $X$ in $G$
\end{enumerate}
\end{definition}

Applying this definition to Figure \ref{fig:agl}, we can evaluating the instrument used in \citet{Armstrong:2013io}.
There we have have $S = \textit{Compensation}_{t-1}$,
$X =\textit{Shareholder support}_{t}$, $Y = \textit{Compensation}_{t-1}$, and $Z = \textit{ISS recommendation}_{t}$.
We use $U$ to denote the observed variables depicted in the dashed box of Figure \ref{fig:agl}.
If create $G_{\overline{X}}$ by deleting the single arrow emanating from $\textit{Shareholder support}_{t}$, we can see that there are two back-door paths running from $Y$ to $Z$: 
$Z \leftarrow S \rightarrow U \rightarrow Y$ and $Z \leftarrow S \rightarrow Y$.
However, both of these paths are blocked by $S$ and the first requirement is satisfied.
The second requirement is clearly satisfied as $Z$ is directly linked to $X$.\footnote{
This is a necessary condition, but assumptions about functional form are also critical in using an instrument to estimate a causal effect.
However, this is not essential to our argument here.}
%
Note that analysis can be expressed intuitively as requiring that the ISS recommendation only affects \textit{Compensation}$_{t+1}$ through its effect on \textit{Shareholder support}$_{t}$, and that \textit{Compensation}$_{t+1}$ has an effect on \textit{Shareholder support}$_{t}$.

But this analysis presumes that the causal diagram Figure \ref{fig:agl} is correct.
\citet[p.\,912]{Armstrong:2013io} note that the ``validity of this instrument depends on ISS recommendations not having an influence on future compensation decisions conditional on shareholder support (i.e., firms listen to their shareholders, with ISS having only an indirect impact on corporate policies through its influence on shareholders' voting decisions)."
This assumption represented in Figure \ref{fig:agl} by the \emph{absence} of an arrow from \textit{ISS recommendation}$_t$ to \textit{Compensation}$_{t+1}$.

Unfortunately, this assumption seems inconsistent with the findings of \citet{Gow:2013aa}, who provide evidence that firms are carefully calibrating compensation plans (i.e., factors that directly affect \textit{Compensation}$_{t+1}$) to comply with the requirements of ISS's policies, implying a path from \textit{ISS recommendation}$_t$ to \textit{Compensation}$_{t+1}$ that does not pass through \textit{Shareholder support}$_{t}$.
This path is represented in Figure \ref{fig:agl2} and the plausible existence of this path suggests that the instrument of \citet[p.\,912]{Armstrong:2013io} is not credibly valid for the causal effect they seek to estimate.


\end{document}
	
