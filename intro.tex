% Quick LaTeX Guide for Dave (originally for Suraj).

% - Percent signs (%) mark comments. To get a percent sign, escape it by putting a backslash in front.
%  & is another special character in LaTeX. Use \& to get &.
% Note that each part of the document is in a separate file (so we can edit in parallel).
% Citations are automatic with the correct key. 
% LaTeX doesn't pay attention to multiple spaces. Also adjacent lines get collapsed into single paragraphs.
% Insert a blank line between lines that are part of two separate paragraphs.
% \section, \subsection, and \subsubsection have the obvious meanings.
% Note that there is a file jar_methods.bib in the list of files to the right that this pulls bibliographic information from.

\section{Introduction}

\begin{quotation}
	There is perhaps no more controversial practice in social and biomedical research than drawing inferences from observational data.
	Despite \dots problems, observational data are widely available in many scientific fields and are routinely used to draw inferences about the causal impact of interventions.
	The key issue, therefore, is not whether such studies should be done, but how they may be done well.
\attrib{\cite{Berk:1999uz}}
\end{quotation}

% \textbf{This section is awesome, Dave.}
% Dave: It's actually helpful to put every sentence on a separate line. 
% You need two line breaks to indicate a paragraph.

The dominant mode of research in accounting is empirical analysis of observational (archival) data with a focus on causal inference. A survey of current accounting research (i.e., papers published in 2014 in the \textit{Journal of Accounting Research}, \textit{Accounting Review}, or \textit{Journal of Accounting and Economics}) reveals that 90\% of empirical papers doing original research seek to draw causal inferences.\footnote{``Original" research excludes papers that are surveys or discussions of other papers and, in this context, ``empirical" papers excluded experiment- and field-based research papers, though most of these seek to make causal inferences also. 
We recognize that a goal of causal inference might be denied by the authors of some papers to which we have attributed that goal.

For example, a researcher might argue that a paper that claimed that ``theory predicts X is associated Y and, consistent with that theory, we show X is associated with Y'' is merely a descriptive paper that does not make causal inferences. 

However, by stating that ``consistent with \dots theory, X is associated with Y," the purpose clearly is to argue that the evidence tilts the scale, however slightly, in the director of believing the theory is a valid description of the real world: in other words, \emph{inference}. 
Additionally, because theory is essentially causal, such inference is inherently \emph{causal} inference.} 
But in general, the causal (or ``treatment'') variables studied by accounting researchers are unlikely to be assigned (as-if) randomly, as is necessary for simple comparisons or regressions to yield valid causal inferences.
Treatment variables like ``tone management'' \citep{Huang:2014cs}, ``occurrence of a material restatement'' \citep{Chen:2014ji}, or ``adoption of fair value reporting model'' \citep{Liang:2014ea} are likely to be the result of complex processes that are very imperfectly understood by researchers.

\cite{Angrist:2010jv} argue that ``empirical microeconomics has experienced a credibility revolution, with a consequent increase in policy relevance and scientific impact.''
They attribute much of the credit for this revolution to the emergence of design-based studies using quasi-experimental methods such as instrumental variables and regression discontinuity designs.
While such methods do appear in accounting research \citep{Larcker:2010fq}, we find that their use is relatively rare, with just X studies using instrumental variables and Y studies using regression discontinuity designs.

It is important to distinguish ``causal inference'' as we use the term from a narrower sense in which causal inference is based on empirical methods---such as those indicated by \cite{Angrist:2010jv}---that are argued to deliver causal parameters unaffected by endogeneity. First, the reality is that the majority of inference in accounting research does not employ these methods. Second, even when such methods are used, we argue that are applied very inexpertly and generally do not support warranted causal inference in the narrower sense.

In the first part of paper, we examine and critically evaluate the use of the methods discussed in \cite{Angrist:2010jv} in accounting research. We argue that so-called natural experiments and instrumental variables generally do not yield.
While regression discontinuity designs rely on weaker assumptions than these other methods, their applicability is inherently limited and, even when they are applicable, the estimates they provide will not be of the effects of primary interest to researchers.

Having argued that causal inference is the primary focus of accounting research, but that standard econometric approaches to warranted causal inference have little applicability in accounting research, the message of the first part of our paper might be perceived as pessimistic and as not offering a realistic path forward for accounting research. 
The goal of the second part of the paper is to offer such a path (or paths) forward. 

In this second part of the paper, we identify emerging approaches in accounting research and also draw on other disciplines to offer a vision for how accounting research might successfully address causal questions even when clever instruments or natural experiments do not provide sharp, easy answers to the questions of interest to the field.

In particular, we advocate the greater use of \emph{structural modeling}. 

%% Some stuff on other fields.
% Economics (IO), political sci, epidemiology, sociology/criminology.
% Sections to write (for later in the paper):
% - Mechanisms - conditional conservatism?
% - Measurement (machine learning)
