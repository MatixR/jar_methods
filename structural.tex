\section{Structural Modeling}
Structural empirical models consist of two main elements:
% One alternative approach to empirical research in economics is labeled the structural approach (Wolpin 2013).
% The structural approach ``requires that a researcher explicitly specify a model of economic behavior, that is, a theory." (Wolpin, 2013, p.2).

\begin{enumerate}
\item a theoretical (economic) model of the phenomena
of interest; and,
\item a stochastic model that links the theoretical model to the observed
data.
\end{enumerate}
The theoretical model minimally
describes who makes decisions, the objectives of decisions makers,
and constraints on decisions and behavior. In developing and analyzing the 
theoretical model, the researcher decides what conditions (variables) matter, 
what is endogenous and exogenous, and what conditions impact behavior. 
The goal in formulating the theoretical model is to derive a set of equations or inequalities
that describe the determinants of decisions. 
Not all these determinants may be observed by the researcher.
The stochastic component of the structural model accounts for determinants the researcher does not observe, as well as why the theoretical model may not perfectly explain the data. 
Additionally, the unobservables may also rationalize why the process that generated the data differs from that envisioned by the theoretical model.

Empiricists employ structural models for several nonexclusive reasons. First, structural models are a means by which a researcher can understand what determines behavior and market outcomes. 
Second, structural models make it clear what data are needed to identify
unobserved theoretical quantities or objects, such as an individual's degree of
risk aversion. 
Third, structural models provide a foundation for estimation and
inference. 
Finally, structural models facilitate counterfactual analyses.
Counterfactual analyses predict what might happen under conditions not observed in the data. 
For example, they might show how accounting reports would change if a new 
accounting standards were adopted. 
 
We now explore these benefits of structural models in more detail, as well as discuss some of the costs that arise in formulating and estimating structural models.

\subsection{The Structural Approach}

The term structural model originated with economists and statisticians working at
the Cowles Foundation in the 1940s and 1950s.
The earliest structural models showed how price and quantity data could be used to recover unobserved demand and supply curves. 
This literature
introduced the issue of identification, and more specifically the idea that (instrumental) variables beyond price and quantity were needed to identify demand and supply schedules. 
The impact of these early models on empirical work in economics encouraged and other social scientists to begin using theoretical models to help interpret 
data.
Indeed, structural models have been used to study: educational choices, voting, 
contraception, addiction, and financing decisions. 
More modern definitions and applications 
of structural models can be found in Reiss and Wolak (2007) and Reiss (2011). 
%TODO: Add references.

Generically, structural empirical models consist of equations or inequalities that describe the optimizing behavior of individuals or organizations. These equations come from decision making models that describe what factors (variables) impact agents' decisions.
Let $y^*$ denote the endogenous choices of agents. 
The researcher may observe these choices, or some transformation of them.
Let $y=y(y^*)$ denote the choices the researcher observes. 
The exogenous factors affecting decisions observed by the researcher are denoted by the vector $x$. 
Unobserved factors generally
are divided into two types: those that vary across sample observations, $\xi$, and
those that are constant across observations, $\theta$. 
The $\theta$'s are referred to as parameters.

Mathematically, the theoretical model delivers either equalities (``structural equations")
$$  y = g(y, x,\xi ,\theta)$$ 
or inequalities, e.g.,
$$  y \le g(y, x,\xi ,\theta)$$
that relate these quantities. In most structural models, the functional form of $g(\cdot)$ is known
and is a consequence of specific functional forms used in the theoretical model. 
Usually the ultimate goal of the structural model is to show how the unknown parameters 
$\theta$ can be recovered from data on $y$ and $x$. 

To illustrate these ideas, under specific economic and behavioral
assumptions, the behavior of demanders and suppliers can be described by a demand and supply system of the form:
$$\begin{array}{lcl}
q^D & = & \theta_1 + \theta_2 \, p + \theta_3 \, x_1 \\[.5em]
q^S & = & \theta_4 + \theta_5\, p + \theta_6 \, x_2 \\[.5em]
q^D  & = & q^S 
\end{array}
$$
Here $y^*$ equals: quantity demanded, $q^D$;  quantity supplied, 
$q^S$; and price $p$.
The exogenous variables in $x$ include demand and cost shifters such as consumer incomes ($x_1$) and suppliers' input prices ($x_2$).
It is important to notice that this structural model consists of three equations. The first two equations are a consequence of the optimizing behavior by individual demanders and suppliers. 
The third condition is an equilibrium condition (i.e., ``demand equals supply"), one that is imposed by the modeler to close the model. 
That is, map the unobserved quantities in $y^*$ to an observed $y$, which is simply ``quantity" ($q$) and price.

In principle these theoretical relations can be taken directly to data.
The problem with taking them directly to data is that the data will prove them flawed. 
For example, in the above demand and supply system the
only object besides $x$ that explain variation in $y$ is the vector $\theta$. 
With many observations on prices and quantities, it would be highly unusual for the six unknown parameters to explain the variation in price and quantity not rationalized by $x$.
For this reason, the researcher typically adds unobservables, such as $\xi$ and $\epsilon$, whose variation across sample observations (along with the other variables) is capable of rationalizing 
all values of $y$.
It is important to realize that these unobservables need not be motivated by the model or related to agent behavior.\footnote{Measurement errors are but one example.}

With the addition of an $\epsilon$, the above demand and supply model becomes 
$$\begin{array}{lcl}
q & = & \theta_1 + \theta_2 \, p + \theta_3 \, x_1 + \epsilon_1 \\[.5em]
q & = & \theta_4 + \theta_5\, p + \theta_6 \, x_2 + \epsilon_2\\
\end{array}
$$
The critical challenge for the researcher at this point is to explain how the observational data on $x$ and $y$ identify $\theta$. 
It is important to realize that just because the structural model contains distinct parameters or quantities, this does not mean that they can be unambiguously estimated. 
When they can, the quantities are ``identified." 
For example, $y$ might be CEO compensation and one
element of $\theta$ might be a coefficient representing a corporate CEO's degree of relative risk aversion.
In order to recover the degree of relative risk aversion from data, the researcher must be prepared to show how risk aversion is recovered from variations in CEO compensation and the other variables.

\subsection{Are Structural Models Relevant to Accounting Research? A Misstatement Illustration}

This high-level discussion of structural modeling may not seem directly relevant to accounting
research. 
Indeed, in general there is a divide between theoretical and empirical research in accounting.
While there are exceptions, few theoretical accounting researchers explain how to map the specifics of their models to data. Similarly, few empirical researchers have attempted to take theoretical models directly to data.
Does this mean that structural models are not applicable to accounting questions? Or, is there unrealized potential? Our view is that structural models have more of a role to play.
To be clear, we do not believe that all accounting researchers need  estimate structural models. 
Indeed, no structural modeling exercise should go forward unless the researcher is highly convinced that the benefits of a structural model would outweigh the substantial costs entailed in developing and estimating a structural model. 

While we have already discussed some of the advantages of structural models, it is 
important to keep in mind the challenges researchers face in developing structural models. 
To begin, structural models can be both technically demanding and time-intensive to develop. 
Additionally, when constructing a theoretical model that can be taken the data, the
empirical researcher will typically be forced to make simplifications that a pure theorist would never make.
Similarly, in order to have an empirical model that is linked to a theory, the empiricist may have to live with assumptions that other empiricists criticize as
unrealistic. 
Finally, just because a researcher can write down a theoretical model and estimate it does not make the empirical model ``right."  There is no guarantee that the causal connections contemplated are correct. 
Further, there is no guarantee that after all the effort that went into developing the model, that the estimates will make sense or that the model will otherwise be validated. 
Despite these challenges, we believe that the benefits of structural models can outweigh their costs.
The purpose of this section is to illustrate how a structural model can deliver accounting insights. 

To illustrate how one might go about developing a structural model for accounting data, we turn to the topic of accounting misstatements. 
There is now an extensive accounting literature exploring why misstatements are made and how easy they are to detect.
This literature is of importance to investors, managers and boards.
Many predictive or explanatory model of misstatements regress indicators for misstatements on a host of accounting, firm and market variables. 
These variables include measures of the complexity of the accounting statements, accrual quality, off-balance sheet activities, firm performance, firm and auditor characteristics, and manager compensation variables. 

If we take this literature as starting point, empirically we would like to have a model that explains why misstatements occur. 
The first problem a structural model must confront is the empirical reality that we do not observe misstatements, but typically only restatements. 
Restatements occur after the firm and its auditor agree on how to report results, and as such are triggered by investigations by third parties or new 
audits.
For simplicity we shall refer to these collectively as ``investigations." 
Thus, any inferences about misstatements have to be seen through the lens of what is detected, or restatements.
If investigations are perfect and detect all misstatements that initially get past the auditor, then there is a one-to-one correspondence of misstatements and restatements. 
Realistically, investigations are not perfect, meaning that we will have to recognize the difference between the two in the structural model.

\subsection{A First Model: Nonstrategic Auditing}

To simplify the initial model, we imagine that misstatements  are
deliberate. 
Though this is clearly a strong assumption, we make it here because it 
allows us to deliver clear predictions about the unobserved rate of misstatements.\footnote{
While this strong assumption can be seen as a weakness of the model, it also can be
seen as a strength. 
If we, or others, do not like the end model, we know what assumption(s)
to revisit.} 
We also simplify matters by assuming that a single agent, the `CEO', is responsible for deciding whether or not to misstate results.
The CEO is assumed to be rational in the sense that they trade off the expected benefits and costs of misstatements when deciding whether to misstate.

Suppose that the CEO receives a benefit of $B^*$ from the \emph{successful} manipulation of earnings, 
i.e., from a misstatement that is not detected by the firm's auditors or subsequent investigations. 
Manipulations are successful only if they are not caught initially by the firm's auditors before a report is released and if they are not caught subsequently during further scrutiny.
We assume the firm's auditors independently catch misstatements at a constant rate $p_A$ and that the (conditional) probability of subsequent investigations catching a misstatement is $p_I$.
Given these assumptions, the probability of misstatement getting past the firm's 
auditor and subsequent investigation is $(1-p_A) \times (1 - p_I)$.
The CEO's benefit from a successful misstatement is then
$$ B^* = (1-p_I) \times (1-p_A) \times B$$
where $B$ is a gross benefit to the manager of a misstatement. 

To misstate performance, the CEO must exert costly effort, which is a fixed $C_M$. 
Putting together the manager's benefits of misstating with their costs gives
\begin{equation}\label{bencost}
y_M^* = \begin{cases}\mbox{Misstate} & \mbox{if }\, (1-p_I) \times (1-p_A) \times B - C_M \ge 0\\
\mbox{Don't Misstate} & \mbox{Otherwise}.\end{cases}.\end{equation}
This (structural) inequality describes the unobserved misstatement process. 
In general, researchers will not observe $B$ or $C_M$. 
In some instances, they may not observe $p_A$ or $p_I$.

To complete the structural model, the researcher must relate these objects to the observed data.
Because we observe a zero-one indicator variable for restatements $y$ and not misstatements, we need to link the two. 
In this model, restatements are the result of three stochastic processes:

\begin{enumerate}
\item The manager decides to misstate (or not).
\item The firm auditor randomly audits and detects (or not) .
\item A post-report investigation happens and detects (or not).
\end{enumerate}

Mathematically, this sequence can be modeled as
\begin{equation}\label{restateeqn}
 y = I(\mbox{Restate}) = I(y^*_M \ge 0) \, \times\, (1 - I(y^*_A \ge 0)) \, \times\, I(y^*_I \ge 0)
\end{equation}
where $I(\cdot)$ is a zero-one indicator function equaling one when the condition in parentheses is true.
The unobserved variables $y^*_A$ and $y^*_I$ respectively reflect the likelihood that the firm's
auditor and the investigation process detect a misstatement. Notice equation (\ref{restateeqn})
uses $(1 - I(y^*_A \ge 0))$, the indicator that the firm's auditor misses the misstatement.

Equation (\ref{restateeqn}) somewhat resembles a traditional binary discrete choice model. The easiest
way to see this is to take expectations (from the researcher's standpoint)
\begin{equation} \label{equilpr}
\begin{array}{lcl}
 E(y) & = & E\, \left[\; I(y^*_M \ge 0) \, \times\, (1 - I(y^*_A \ge 0)) \, \times\, I(y^*_I \ge 0) \; \right]\\[1em]
 & = &  \mbox{Pr(Misstate)} \times \mbox{Pr(Auditor Misses)} \times
\mbox{Pr(Investigation Finds)}\\[1em]
& = & \beta^* \times (1-p_A) \times p_{I} = \mbox{Pr(Restate)}
\end{array}\end{equation}
From equation (\ref{bencost}), $\beta^*$ is the (researcher's) forecasted probability that a misstatement occurs, or
\begin{equation}\label{betaplus}
\beta^*= \mbox{Pr}\left(\, (1 - p_A)(1 - p_I) B - C_M \ge 0 \,\right)
\end{equation}

At this point, the theory has delivered a structure for relating the \emph{unobserved} probability of a misstatement, $\beta^*$, to the potentially estimable probability of a restatement.
Now we face a familiar structural modeling problem, which is that the equations delivered by theory do not necessarily anticipate all the reasons why
in practice why CEO behavior, auditor behavior or outside scrutiny might vary across accounting reports.
For example, the theory so far does not point to different reasons why CEO's might differ in their cost-benefit analyses of misstatements. 
To move theoretical relations closer to the data, researchers typically introduce observable reasons into them. Often there is some ad hoc or ``nonstructural" element to these
additions. 
Empiricists are willing to do this, however, because they believe that it is important to account for heterogeneity they believe the theory does not explicitly recognize.

To illustrate this approach, here we assume that CEO's unobserved costs and benefits do vary systematically with observables.
In addition, because these observables do not perfectly represent the observed and unobserved benefits, it is important to allow for unobservable differences in the costs and benefits of misstatements. 
One specification that does this is to assume
\begin{equation}\begin{array}{lcl}\label{eqns1}
B & = & b_0 + b_1 \, \mbox{BONUS} + X_B\beta\\[.5em]
C_M & = & m_0 + m_1 \, \mbox{SALARY} + X_C\gamma + \xi\\[.5em]
\end{array}
\end{equation}
where BONUS is the fraction of a CEO's total pay that is stock-based compensation, 
the $X_B$ are other observable factors that impact the manager's benefits from misstatements,
SALARY is the CEO's annual base salary, and the $X_C$ are observable factors impacting the CEO's perceived costs of misstatements.

Why this linear specification and why these variables? 
We have no strong theoretical reason for the linear assumption. 
Instead, its motivation is practical.
We shall shortly see that it facilitates estimation of the model unknowns.
As for the variables in the benefit and cost specifications, here we rely in part on theoretical and empirical observations in the academic literature on misstatements. 
The BONUS variable is included in benefits because it is thought to capture a classic moral hazard problem: the more CEO are rewarded for performance, the
greater their incentive to misstate results so as to increase (perceived) 
performance.
%TODO: Get refs
Thus, we would expect the unknown coefficient $b_1$ to be positive.
Similarly, the SALARY variable is included in costs because previous studies have hypothesized that the higher a CEO's guaranteed base pay, the more the CEO perceives ex ante that it is risky to make misstatements.
Thus, we would expect the unknown coefficient $m_1$ also to be positive. For now, we leave the other $X$ variables unnamed.

One key variable in the above model is the unobserved cost $\xi$.
While it makes sense to say that the researcher cannot measure all misstatement
costs, why not also allow for unobserved benefits as well.
The answer here is that adding an unobserved benefit would not really add to the model as it is the net difference that the model is trying to capture.\footnote{
The sense in which it could matter is if we thought we observed the probabilities
$p_A$- and $p_I$.
In this case, we might be able to distinguish between the cost and benefit unobservables based on their variances.}
To see this, observe that the additive error in costs becomes the additive error in the net benefit to a misstatement. 
Further, the probability of a restatement becomes
\begin{equation}\label{restate1}
\mbox{Pr(Restate)} = \theta_0 \mbox{Pr}\left(\, \theta_1 + \theta_2 \mbox{BONUS}
+ \theta_3 \mbox{SALARY}  \ge \xi \,\right)
\end{equation}
where $\theta_0=(1-p_A) \times p_{E}, \theta_1 = (1 - p_A)(1 - p_I) b_0 - m_0, 
\theta_2 = (1 - p_A)(1 - p_I) b_1,$ and $\theta_3 = - m_1.$ 

Apart from the scalar multiple $\theta_0$, which can be absorbed into the probability statement (and is thus not identified), this probability model has the form of  a familiar binary choice (e.g., a probit or logit model).
Thus, the value of the structure imposed so far is that it can motivate the application of a familiar statistical model, as  well as explain how the estimated coefficients are potentially connected to quantities that impact the probability of a misstatement.

\subsection{Estimation}

To illustrate the application of this structural to data, we assembled a dataset containing 5,000 firm-year observations on whether or not financial results were restated in a given year.
Table \ref{tab:desc} describes the variables in our data set.
We discuss the source of the data in more detail after we estimate the structural model.

The data include variables that have previously been used to model restatements.%TODO: \footnote{[Refs to be included]} 
The variable BIG4 is included  because it is believed that Big 4 auditing firms have more expertise and are therefore more likely to catch misstatements. 
Similarly, the corporate governance literature suggests that board oversight from directors with accounting or finance backgrounds reduces the likelihood that CEOs will make misstatements. 
%TODO: Get refs.
Finally, the variables INT and 
SEG are included to capture the complexity and costs of audits. 
%TODO: Get reference for audit model.
Specifically, international companies and companies with more business segments are thought to raise the costs of auditing. 

Table \ref{tab:desc} reports descriptive statistics for the sample. CEOs are on average receive about three-quarters of a million dollars in base pay and their incentive-related pay averages 26\% of their total pay.
Three-quarters of the sample has a Big 4 accounting firm as its auditor. The fraction of directors with financial expertise is less than ten percent. 
The average firm has about four and one-half business SEG and is primarily based in the United States.

Absent a theoretical model of mis- or re-statements, most empirical
analyses of these data would summarize them by regressing the restatement
indicator on the list of predictors in Table 1. Such a regression can either be described
as showing how the probability of a restatement co-varies with the right hand side variables,
or it can be described as a prediction equation. Our structural model allows us to say more,
particularly when it comes to the signs of the SALARY and BONUS variables. By including
the other variables on the right hand side, we are in essence maintaining they belong in
$X_C$, $X_B$, or both sets of regressors.

Table \ref{tab:logit} reports the results of logit regressions in which the dependent
variable is the restatement indicator variable. 
The table contains both a simple 
specification containing an intercept along with the two CEO pay variables, 
and a more intricate specification involving the other variables we have in our data.
For each specification we report the estimated coefficients of the logit and the 
corresponding marginal effects evaluated at the sample averages of the exogenous
variables.%TODO: Explain marginal effects.
The results for the pay coefficients in both specifications run counter those the previous accounting literature might predict and counter to those predicted by the structural model.
Specifically,  more base pay is associated with more restatements, while more incentive pay is associated with fewer restatements.
Besides the intercepts and BONUS coefficients, the only other coefficients  that are statistically significant are INTional companies and companies with more SEG to audit. 
While we can say (descriptively) that they are associated
with higher restatement rates, unless we take a position on how they enter $X_C$ or $X_B$, it is difficult to interpret whether their signs make sense.

The question we now address is what to make of the fact that the coefficients
on the CEO pay variables did not turn out as either informal arguments or our
structural model would predict. 
We consulted colleagues in the profession for their opinions. 
%TODO: Really?
One set attributed the outcome to data errors or sampling issues. 
Another set said the logit model was obviously flawed because it did not include all relevant accounting variables. (High on the list were measures
of recent accrual activity, use of off balance sheet transactions and firm
performance variables.) 
Their thinking was, had we included these, the signs on the pay coefficients might be different. 
Mentioned less often was that the idea that an important difference between restatements and misstatements is the role of subsequent investigations. 
There is nothing in the model or variable list that would help predict  the intensiveness of outside scrutiny (or indeed of scrutiny by the firm's external auditors).

\subsection{An alternative model}

The point we would like to make here is that the structural model can help us understand the potential influence and importance of each of these points. 
To illustrate this, we will focus attention on the model's potentially overly simplistic view of the auditor's role in detecting misstatements.  
To make the model richer, suppose the firm's  auditors are more likely to look for misstatements when they perceive they are more likely to find misstatements. 
This reasoning naturally
leads to a strategic decision making model where the managers' and the auditors' decisions are interdependent.
%TODO: Get references (e.g. Tshibano (198X) and Bresnahan and Reiss (1991)).
In such a model, auditors too presumably trade off the costs of audit effort against the reputational losses they might incur should they miss a managerial misstatement and the misstatement is subsequently detected.\footnote{
Here we have in mind the findings of \citet{Dyck:2010kh} who show that many egregious forms of misstatements are detected subsequently by employees, directors, regulators, and the media.} 

In the previous model, the firm's auditor impacted the manager's misstatement benefits through $p_A$. 
Suppose that $p_A$ is in fact a choice variable for the firm's auditor. 
To make matters simple, suppose that the auditor detects manipulation with probability $p_{AH}$ if they exert high effort and  otherwise they detect manipulation with the lower probability $p_{AL}$. 
Let the cost of high effort be a fixed cost $C_A > 0$. 
Without loss of generality suppose the cost of low effort is zero. 
When deciding whether to audit with high or low effort, the auditor perceives a cost to its reputation, $C_R$, to not detecting a misstatement that is subsequently caught by an external investigations. 
This structure implies that the total cost of high effort to the auditor is $C_A + (1-p_{AH}) \times p_E \times C_R$ -- the cost of high effort plus the expected cost of missing a misstatement that is subsequently caught with probability $p_E$. 
The total expected cost of
low effort is similarly, $(1-p_{AL}) \times p_E \times C_R$. 

We complete this new model, we need to make an (equilibrium) assumption about how the CEO and firm auditor interact. Following the literature, we assume that the two simultaneously
and independently make decisions, and that their strategies form a Nash equilibrium.
That is, we assume the players' strategies are such that they optimize their objectives 
taking the actions of the other players as fixed. This means that in a Nash equilibrium, 
the players are taking actions that they cannot unilaterally improve upon.

It is well known that in this type of auditing game, that the CEO and the auditor 
best actions are to play a mixed (randomized) strategy.\footnote{Explain.}
That is, the auditor will independently exert high effort with probability $\alpha^*$ 
and the manager independently misstates with probability $\beta^*$. These probabilities 
are such that each has no incentive to change their randomized strategy; that is:\\
\begin{quote}
(i) the manager is indifferent between misstating and not misstating, or:\\
\begin{equation}\label{manager}
(1 - p_A^*)(1 - p_I) B - C_M = 0 
\end{equation}
where $p_A^* = \alpha^* p_{AH }+ (1-\alpha^*) p_{AL}$ is the equilibrium 
probability a misstatement is detected; and,\\
\end{quote}
\begin{quote} (ii) the auditor must be
indifferent between exerting high and low effort, or\\
$$ \beta^* (1-p_{AH}) p_G C_R + C_A = \beta^* (1-p_{AL}) p_I C_R .$$
\end{quote}

\vglue 5pt
Solving these two equations for the equilibrium probabilities $\alpha^*$ and $\beta^*$
yields:\\
\begin{equation}\label{equilstrat}
\begin{array}{lcl}
  \alpha^* &= & \dfrac{ ( 1 - p_{AL}) (1 - p_I) B- C_M}{ (1 - p_I) (p_{AH}-p_{AL}) B}\\[1.5em]
  \beta^* &= & \dfrac{C_A}{(p_{AH}-p_{AL}) p_I C_R}  
\end{array}
\end{equation}
From these equations, we can calculate the equilibrium probability of a restatement\footnote{
As part of the solution, we require $\alpha^*$ and $\beta^*$ to be probabilities between
zero and one. This is true provided $C_R$ and $B$ satisfy the inequality\\
$$  C_R > \frac{C_A}{(p_H-p_L)p_I} $$
and \\
$$ B > \frac{C_M}{1 - p_I}  $$}
\begin{equation} \label{equilpr1}
\begin{array}{lcl}
\mbox{Pr(Restate)} & = &  \mbox{Pr(Misstate)} \times \mbox{Pr(Auditor Misses)} \times
\mbox{Pr(Investigation Finds)}\\[1em]
& = & \beta^* \times (1-p_A^*) \times p_{E}
\end{array}\end{equation}
This equation tells us how the observed (or measurable) probability of a restatement is related to
the unobserved frequency of misstatements. In particular, if we knew the frequency with which auditors and investigations caught misstatements, we could easily link the two. Otherwise,
we would have to estimate these probabilities (or make assumptions about them).

Substituting the equilibrium strategies (\ref{equilstrat}) into (\ref{equilpr1}) yields
\begin{equation} \label{equilpr2}
\begin{array}{lcl}
\mbox{Pr(Restate)}& = &  \dfrac{C_AC_M(1-p_{IL})}{(p_{IH}-p_{IL})(1-p_I)C_RB}.
\end{array}\end{equation}
We now are in a position to use the theory to help interpret the conflicting logistic regression results
in Table 3. 

 
Equation (\ref{equilpr2}) shows that the presence of a strategic external auditor
changes how the CEO's incentives impact the probability of a restatement.\footnote{Notice that
the probability statement in equation  (\ref{equilpr2})  differs from that in equation (\ref{restate1}).
The probability statement in equation  (\ref{equilpr2}) reflects the randomness of the
strategies, whereas in equation (\ref{restate1}) it reflects unobservables the researcher 
does not observe.} Partial derivatives of equation (\ref{equilpr2}) show that the restatement probability is:\\

\begin{itemize}
\item Decreasing in the benefit $B$ that the manager enjoys from misstatement.
\item Increasing in the personal cost of manipulation $C_M$ incurred by the manager.
\item Decreasing in the reputational cost $C_R$ incurred by the external auditor.
\item Increasing in the cost of high effort $C_A$ incurred by the external auditor.
\end{itemize}

Thus, in contrast to the nonstrategic model, increasing the benefit that managers enjoy from misstatement,
or decreasing the misstatement cost, leads to fewer restatements being observed by researchers.
These two effects explain the negative sign on BONUS and the positive sign on SALARY observed in the previous logit results. Thus, this structural model has the potential to rationalize patterns observed in the data.

To have a better sense of how one might connect the strategic auditor theory to the
logistic models in Table \ref{tab:logit}, suppose, similar to ways we motivated (\ref{eqns1}), that 
\begin{equation}\begin{array}{lcl}\label{eqns2}
B & = & b_0 + b_1 \, \mbox{BONUS} \\[.5em]
C_M & = & m_0 + m_1 \, \mbox{SALARY} \\[.5em]
C_A & = & a_0 + a_1 \, \mbox{INT} + a_2 \, \mbox{SEG}\\[.5em]
C_R & = & r_0, \quad B  =  b_0, \quad p_{AH}   =  p_0, \; \mbox{ and } \; p_{AL}  =  v_0 \\[.5em]
\end{array}
\end{equation}
where $ a_0, a_1, a_2, r_0, b_0, p_0$ and $v_0$ are constant parameters. 
Inserting these expressions into the expected restatement rate (\ref{equilpr2}) gives
\begin{equation*} \label{equilpr3}
\begin{array}{lcl}
\mbox{Pr(Restate)}& = &  \dfrac{C_AC_M(1-p_{IL})}{(p_{IH}-p_{IL})(1-p_I)C_RB}\\[2em]
& = & \dfrac{(1-v_0)(a_0 + a_1 \, \mbox{INT} + a_2 \, \mbox{SEG})(m_0 + m_1 \, \mbox{SALARY})}
{(p_0-v_0)r_0(b_0 + b_1 \, \mbox{BONUS})}\\[2em]
\end{array}
\end{equation*}
\begin{equation}\label{equilpr4}
 =  \dfrac{\theta_0 + \theta_1\mbox{\small INT} + \theta_2\mbox{\small SEG} + \theta_3\mbox{\small SALARY}
+ \theta_4\mbox{\small INT} \times \mbox{\small SALARY}+ \theta_5\mbox{\small SEG} \times \mbox{\small SALARY}}
{1 +  \theta_6\mbox{\small BONUS}}
\end{equation}
Notice that the $\theta$s absorb unknown quantities such as $r_0$ and $p_0$, and that the denominator intercept
is normalized to one. This last restriction is required to identify the ratio of the two linear functions.

Although this model does not have a logit form, it is potentially estimable using 
generalized method of moments (GMM).
This method attempts to match so-called sample moments to what the structural model implies they should be. 
For example, an obvious sample moment would be the average restatement rate in the sample.
The corresponding theoretical moment would be the probabilty expression in equation (\ref{equilpr4}).
Because we need at least as many moments as we have $\theta$ parameters to estimate (there are seven $\theta$s in the model), we use seven sample moments, each of the form:
$$ \mathcal{M}_j = \sum_{i=1}^{5,000} \; X_{ji}^\prime\left[\; \mbox{RESTATE}_A - \mbox{Pr(Restate)}_A \; \right]. $$
where Pr(Restate) comes from equation (\ref{equilpr4}).\footnote{
To ensure that the model parameters imply restatement probabilities between zero and one, we add a penalty function to the GMM objective function.
This penalty increases with the number of estimated probabilities below zero or above one.
For most replications this penalty is immaterial to the results obtained.} 
The $X_j$ used in the moments include all explanatory variables. 
Thus, because $X$ includes  a dummy variable for whether the firm is an international company, the corresponding moment equation seeks to match the sample international companies average restatement rate to the model's prediction for that rate.

Table \ref{tab:gmm} reports the results of estimating the new (strategic auditor) structural model on the sample of 5,000 firms. 
The results show that in this particular case, even without sample information on the unobserved probabilities $p_A$ and $p_I$, we can recover estimates of the model parameters up to a normalization.\footnote{
A simple way to see this might be the case is to observe that there are seven $\theta$ coefficients and eight underlying structural parameters.}
For instance, the coefficient ratio $\theta_3/\theta_0$ estimates the ratio of cost parameters $m_1/m_0$.
Since $m_1$ is the cost coefficient on SALARY and $m_0>0$ for costs to make sense, the sign of $\theta_3/\theta_0$ reveals the sign of $m_1$.
From the theory, we expect the sign to be positive, and in the estimates it is. 
Similarly, $\theta_6$ equals the (scaled) misstatement benefit coefficient on the BONUS variable.
Although the descriptive regression coefficients in Table 3 suggest BONUS has a negative effect on restatements, here, because we model misstatements as part of restatements, we find it has a positive effect, as expected.

The one sign that does not make sense given the other coefficient estimates is the negative sign on $\theta_6$, however this coefficient is insignificantly different from zero.
This is indeed true of most of the coefficients, and is perhaps not surprising given the participants use of randomized strategies.
Further, even with a sample size of 5,000, restatements are relatively rare, thus making it difficult for the model to predict them with much accuracy. 

While the coefficient magnitudes do not allow us to estimate the underlying benefits and costs to managers from misstatements, we can illustrate the value of the model by performing a counterfactual calculation.
There are many different counterfactuals that could be considered. 
Here, for illustrative purposes we can ask what would happen to misstatements and restatements if we  do away with incentive pay and nothing else changes.
The value of having a model to analyze this
change is that we can see how the auditing process would adjust to the removal of 
CEO incentives to misstate. 
From the equilibrium strategies in equation (\ref{equilstrat}), we see that removing bonus pay does not
change the equilibrium frequency of misstatements, but does change the frequency of high effort auditing.
From (\ref{equilpr}), the model and the data we find
$$ \dfrac{\mbox{Pr(Restate }\vert \mbox{ No Bonus)}}{\mbox{Pr(Restate }\vert \mbox{ Bonus)}}=\dfrac{\beta^* \times (1-p_A^{**}) \times p_{E}}
{\beta^* \times (1-p_A^{*}) \times p_{E}} = \dfrac{(1-p_A^*)}{(1-p_A^*)} = 1.10.$$
What this says is that the restatement rate increases by 10\% (from 10.24\% to 11.25\%) when the bonuses
are withdrawn. 
The fact that the restatement rate goes up may at first seem somewhat odd given that the benefits to the CEOs have fallen. 
The model, however, shows that the increase  comes about because the auditors exert \emph{less} effort in detecting misstatements, thereby catching fewer, leaving more for outsiders to subsequently catch.  

The discussion above illustrates some of the ways in which a structural modeling exercise might help understand accounting data. 
In particular, the comparative statics of the model shed light on the difference between restatements and misstatements, and what assumptions (e.g., strategic versus nonstrategic auditor) and data were needed to draw  inferences about misstatements from restatements. 
Additionally, we were able to recover some of the primitive parameters impacting incentives for managers to misstate results, as well as perform counterfactual analyses.

While there is the potential for disappointment in the simplistic theory we used, we see room for improving models as an opportunity rather than a defining limitation.

\subsection{Limitations of structural models}
% This discussion may belong near the end. When we talk about structural modeling, we should make clear that some questions are difficult to fit into a structural model, etc.
The vast majority of empirical research papers in accounting do not rely on a formal theoretical model to motivate their hypotheses.
But in many cases, extant theory would not support estimation of a structural model.
For example, \citet{Huang:2014cs} study the effect ``tone management'' on capital market outcomes.
Developing a formal theory of the relation between firm performance, managerial psychological states, and measures of tone would be a complex undertaking involving economics, psychology, and linguistics.
Building on such a (hypothetical) foundation to solve the complex game involving managers and capital markets would be extremely ambitious.
Instead, \citet{Huang:2014cs} does what almost all empirical research papers in accounting do and resorts to more verbal approaches to hypothesis development. 
% Can we find some critique of this approach? Pfleiderer?  YES -- let's review Paul's paper and add some of this and include him in the references

To those who would see the ``glass as half full," the exercise can be seen as yielding insight into an important accounting issue and how accountants might better take advantage of data (and indeed what extra data they might like to collect).

\subsection{Structural models in accounting research}

The use of structural models in accounting research has been very modest to date.  There are certainly examples of regression models being derived (or perhaps influenced) by a theoretical model. For example, Lambert and Larcker (1985) use the traditional Holmstrom model (and a variety of simplifying assumptions) to help specify a regression function linking CEO compensation to firm performance. While somewhat structural in orientation, the approach suppresses the fundamental causal mechanism associated with compensation decisions and does little to actually estimate the primary structural model of Holmstrom.

% This stuff is in Peter's structural section (i.e., before Section 6).
Gerakos and Kovrijnykh (2013) and Nikolaev (2014a,b) provide an analysis misreporting and accounting quality that resembles some features of structural models (e.g., Nikolaev implements method of moment estimation). Both papers develop a dynamic stochastic model for the accounting process and use this structure to separately identify quality earnings from manipulated earnings. While interesting empirical studies, these papers simply assume the fundamental stochastic process for earnings and abstract away from all the underlying (optimal) managerial decisions and accounting choices that ultimately produce observed accounting numbers. 

The recent papers by Zakolyukina (2014) and Bertomeu et al (2015) are more consistent with traditional structural modeling.  Similar to our simple model above, Zakolyukina is concerned with how equity incentives motivate managers to manipulate earnings.  Bertomeu et al examine management forecasts using a formal disclosure model to estimate whether managers strategically withhold information from shareholders.

These two papers model an institutionally rich problem, estimate the derived model, provide estimates for important structural parameters, and also give interesting counterfactuals based on their theoretical models.  
We view these papers as useful initial steps in applying structural approaches to accounting research questions.\footnote{Maybe review some structural finance papers -- Luke Taylor on CEO labor markets, Whited on debt?, Terry jmp about "meet or beat" and macroeconomic research and development.} % Is this footnote for the editors or for us?

