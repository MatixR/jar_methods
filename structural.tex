

\section{Structural modeling in accounting research}

\subsection{What is structural modeling?}
The methods for estimating causal effects described above (i.e., instrumental variables, RDD, and natural experiments) have been characterized as the experimentalist approach. 
We argue above that the experimentalist approach is unlikely to yield more than a tiny fraction of the research in top accounting journals. One alternative approach to empirical research in economics is labeled the structural approach (Wolpin 2013).

The structural approach ``requires that a researcher explicitly specify a model of economic behavior, that is, a theory." (Wolpin, 2013, p.2). As pointed out by \citet{Rust:2014im}, Wolpin (2013) does not require ``rational, optimizing agents," but includes ``models of agents who have `irrational' or subjective beliefs, or theories involving time-inconsistent or sub-optimal decision making." Nonetheless, it is clear that few empirical accounting studies use anything resembling structural models.

\subsection{Why do we need structural modeling?}
To understand the need for structural modeling, we 



While structural modeling is clearly not the norm in accounting research, there are precedents for its use.

\subsection{Residual income valuation}
The analytical work of Ohlson (1995), etc., spawned an literature that studied how well real-world data line up with the analytical models.

Note that the analytical framework of Ohlson (1995) examines the implications of three assumptions: 
\begin{itemize}
\item PVED: Price equals the present value of expected dividends
\[ P_t = \sum_{\tau=1}^{\infty} R_f^{-\tau} \mathbb{E} [\tilde{d}_{t+\tau} ], \]
\item CSR: The clean surplus relation (CSR), and
\item LID: Linear information dynamics (LID). 
\end{itemize}

One aspect of this model is that there are no optimizing agents, let alone rational ones. Thus this lacks elements found in many of the models studied by Wolpin (2013). Nonetheless, the framework of Ohlson (1995) provides a structure that seems might be of use to structuring empirical analyses. However, note that PVED is a topic much-studied in finance independent of accounting. Additionally, CSR is either trivially satisfied (by defining income appropriately) or easily seen not to hold (due to other comprehensive income). Put aside these two assumptions leaves just LID, which could be empirically tested by looking at the associated informations processes themselves.

\citep{Dechow:1998}

\subsection{Earnings quality}

\citet{Dechow:1998}

Dechow and Dichev (2002) Various earnings quality models: These are very much reduced form structural models, to abuse some terminology. I think “earnings = economic truth + noise” is a bad starting point.

\subsection{Microstructure models}

3. Verrecchia-style models are structural, but it’s not clear that these are very amenable to empirical testing. There are many non-nested versions of these models (earnings is complementary, earnings is a substitute).

4. Barron, O. E., Kim, O., Lim, S. C., \& Stevens, D. E. (1998) is a structural model. I’m not sure that the assumptions it uses get sufficient scrutiny. It seems to get used in circumstances where it is hypothesized not to hold.

I think we could describe these models to illustrate what is meant by a structural model and then suggest that using these models needs to be more thoughtful and to reflect gradual accumulation of knowledge (rather than “here’s a model that isn’t inconsistent with the data” without any comparison of whether it is better than another model).

I also think it would be helpful to illustrate that complete lack of modeling (i.e., in most accounting research), it is not clear what is being achieved. This isn’t the debate from the economics literature, as we generally aren’t choosing between plausible identification without theory and structural models; rather we have no identification and no theory as the status quo alternative. Conservatism could be a good topic for illustration of this.

\subsection{Application: Conservatism}
